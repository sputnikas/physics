Задача 1.
Найти множество мощности континуума не являющееся непрерывным множеством.
Задача 2.
Определить почему для размерности пространства 4 существуют 6 основных видов частиц.
Задача 3.
Выяснить физический смысл закона дисперсии для бесконечной цепочки гармонических осцилляторов в квантовом случае
Задача 4.
Решить волновое уравнение методом Фурье-преобразования найти функцию Грина волнового уравнения.  Объяснить наличие или отсутствие в функции Грина запаздывающего  и опережающего потенциалов.
Задача 5.
Найти решение для геона уравнений общей теории относительности доказать что геон может быть пространственно-ограничен и состоит из гравитационных волн показать что он может переносить энергию и является прообразом частицы всего.
Задача 6.
Найти оптические аналоги элементов компьютерной логики и, или, не.  Построить аналог оперативной памяти на оптических элементах.
Задача 7.
Вывести из принципа наименьшего действия Решение уравнения электродинамики  в предположении что фотон обладает массой эта масса постоянна и неизменна заряженная частица всего одна и движется произвольным образом.  Показать что для такой частицы решение в форме потенциалов Лиенара-Вихерта  не  выделяет направление времени.
Задача 8.
Найти условия ( плотность вещества , температура вид вещества) при которых возможен холодный термоядерный синтез. Предполагается известными сечение реакции термоядерного синтеза реакция протекает в веществе. На основе полученных данных сделать вывод о возможности или невозможности холодного термоядерного синтеза.
Задача 9.
Доказать что уравнение движения частицы в гравитационном поле может быть выведено из уравнения общей теории относительности.
Задача 10.
Для данной системы мировых линий построить гравитационное поле.  Определить какое распределение масс будет создавать такое поле.
Задача 11.
Для частицы, характеризуемой скаляром, существует всего три степени свободы для частицы, характеризуемой вектором, существует пять степеней свободы, если этот  вектор постоянен по модулю, определить какое количество степеней свободы существуют для частицы, характеризуемой тензором 2 ранга, тензором 3 ранга и так далее. Как все это связано с групповыми свойствами движений.
Задача 12.
Найти  над аппаратом бесконечных матриц операции, которые будут приводить к запоминанию  матрицами внешних объектов( других матриц,  но часто конечных) обработке внешних объектов, выводу результатов в определённую область бесконечной матрица.
Задача 13.
Записать нелинейные уравнения электродинамики с точностью до членов третьего порядка.
Задача 14.
Показать, что волновая функция фотона 6-компонентная, три компоненты которой представляют собой комплексные векторы действительная часть электрическое поле, мнимая часть магнитное поле. Выразить импульс фотона при таких обозначениях.
Задача 15.
Сформулировать принцип наименьшего действия для уравнения Дирака для свободной частицы распространить его на случай частицы в электромагнитном поле.  Записать уравнение движения частицы и уравнение электродинамики в случае таких полей. Найти решение такой системы в одном из простых случаев.