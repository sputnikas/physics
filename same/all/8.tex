\section{Гипергеометрические функции}

\subsection{Гипергеометрическая функция (функция Гаусса)}

Сумма бесконечной геометрической прогрессии:
\[
	S = b_0(1 + x + x^2 + \ldots)
\]
Её обобщением является следующая трёхпараметрическая функция:
\[
	F(\alpha, \beta; \gamma; x) = 
	1+
	\frac{\alpha \beta}{\gamma} \frac{x}{1!} + 
	\frac{\alpha(\alpha + 1) \beta (\beta + 1)}{\gamma (\gamma + 1)} \frac{x^2}{2!} +
	\frac{\alpha(\alpha + 1)(\alpha + 2) \beta (\beta + 1)(\beta + 2)}{\gamma (\gamma + 1)(\gamma + 2)} \frac{x^3}{3!} +
	\ldots
\]
Ряд сходится только при $|x| < 1$.
Некоторые очевидные соотношения:
\[
	F'(\alpha, \beta; \gamma; x) = 
	\frac{\partial F(\alpha, \beta; \gamma; x)}{\partial x} = 
	\frac{\alpha \beta}{\gamma} F(\alpha+1, \beta + 1; \gamma + 1; x)
\]

Она является частным решением дифференциального уравнения Эйлера:
\[
	x(x - 1) y'' + [(\alpha + \beta + 1)x - \gamma] y' + \alpha \beta y = 0
\]
Общее решение уравнения Эйлера:
\[
	y = C_1 F(\alpha, \beta; \gamma; x) + C_2 x^{1 - \gamma} F(\alpha - \gamma + 1, \beta - \gamma + 1; 2 - \gamma; x)
\]
Из определения гипергеометрической функции следует:
\[
	\gamma \ne 0, -1, -2, \ldots
\]
Интегральное представление:
\[
	F(\alpha, \beta; \gamma; x) =
	\frac{\Gamma(\gamma)}{\Gamma(\beta) \Gamma(\gamma - \beta)}
	\int\limits_0^1 t^{\beta-1}(1 - t)^{\gamma - \beta - 1} (1 - tx)^{-\alpha} dt
\]
Как следствие:
\[
	F(\alpha, \beta; \gamma; 1) = \frac{\Gamma(\gamma)\Gamma(\gamma - \alpha - \beta)}{\Gamma(\gamma - \alpha)\Gamma(\gamma - \beta)}
\]
Другая форма дифференциального уравнения Эйлера:
\[
	\frac{d}{dx}
	\left[x^\gamma(x - 1)^{\alpha + \beta + 1 - \gamma} y'\right] + \alpha \beta x^{\gamma - 1}(x - 1)^{\alpha + \beta - \gamma} y = 0
\]

\subsection{Преобразование уравнения Эйлера}

\[
	x(x - 1) y''_{xx} + [(\alpha + \beta + 1)x - \gamma] y'_{x} + \alpha \beta y = 0
\]
Выполняем замену:
\[
	x = f(t) \quad y = g(t) z(t)
\]
\[
	\begin{aligned}
	& y''_{xx} = \frac{1}{f'} \frac{d}{dt} \left(\frac{y'}{f'}\right) = 
	\frac{1}{f'^2} y'' - \frac{f''}{f'^3} y'
	= \frac{1}{f'^2} (g z'' + 2 g' z' + g'' z) - \frac{f''}{f'^3} (gz' + g'z)
	\\
	& y'_x = \frac{1}{f'} (gz' + g'z)
	\end{aligned}
\]
\[
	\begin{aligned}
	\frac{f(f - 1)}{f'^2} g z'' + &
	\left(
		\frac{2g'f(f - 1)}{f'^2} + \frac{(\alpha + \beta + 1)f - \gamma}{f'} g - \frac{f''f(f - 1)}{f'^3} g
	\right) z' + \\ & +
	\left(
		\frac{f(f - 1)}{f'^2} g'' + \frac{(\alpha + \beta + 1)f - \gamma}{f'} g' - \frac{f''f(f - 1)}{f'^3} g' + \alpha \beta
	\right) z = 0
	\end{aligned}
\]
\[
	z'' + 
	\left(
		\frac{2g'}{g} + \frac{(\alpha + \beta + 1)f - \gamma}{f(f - 1)} f' - \frac{f''}{f'}
	\right) z' +
	\left(
		\frac{g''}{g}  + \frac{(\alpha + \beta + 1)f - \gamma}{f(f - 1)} \frac{g'}{g} - \frac{f''}{f'} \frac{g'}{g} + \alpha \beta
	\right) z = 0
\]
Накладываем дополнительное условие:
\[
	\frac{2g'}{g} + \frac{(\alpha + \beta + 1)f - \gamma}{f(f - 1)} f' - \frac{f''}{f'} = 0
\]
\[
	\frac{2g'}{g} + \frac{\alpha + \beta + 1}{f - 1} f' - \gamma \left(\frac{1}{f - 1} - \frac{1}{f}\right) f'- \frac{f''}{f'} = 0
\]
\[
	\frac{2g'}{g} + \frac{\alpha + \beta + 1 - \gamma}{f - 1} f' + \frac{\gamma}{f} f'- \frac{f''}{f'} = 0
\]
\[
	\ln g^2 + \ln (f - 1)^{\alpha + \beta + 1 - \gamma} + \ln f^\gamma - \ln f' = const = \ln C
\]
\[
	g^2 (f - 1)^{\alpha + \beta + 1 - \gamma} f^\gamma = C f'
\]
\[
	\int g^2 dt = C \int (f - 1)^{\gamma - \alpha - \beta - 1} f^{-\gamma} df \text{ даёт связь между $f(t)$ и $g(t)$}
\]
Учёт условия даёт:
\[
	z'' + 
	\left(
		\frac{g''}{g}  - 2 \frac{g'^2}{g^2} + \alpha \beta
	\right) z = 0	
\]
Если ввести функцию:
\[
	h(t) = \frac{1}{g}
\]
\[
	\frac{z''}{z} - 
	\frac{h''}{h} + \alpha \beta
	 = 0	
\]

\subsection{Вырожденная гипергеометрическая функция (функция Куммера)}

Уравнение Куммера:
\[
	x y'' + (\gamma - x) y' - \alpha y = 0
\]
Его можно получить заменой в уравнении Эйлера $x$ на $x/\beta$ и устремив $\beta \to \infty$. Функция Куммера:
\[
	F(\alpha; \gamma; x) = 1+
	\frac{\alpha}{\gamma} \frac{x}{1!} + 
	\frac{\alpha(\alpha + 1)}{\gamma (\gamma + 1)} \frac{x^2}{2!} +
	\frac{\alpha(\alpha + 1)(\alpha + 2)}{\gamma (\gamma + 1)(\gamma + 2)} \frac{x^3}{3!} +
	\ldots
\]
Ряд сходится при любых значениях $x$. Общее решение уравнения Куммера:
\[
	y = C_1 F(\alpha; \gamma; x) + C_2 x^{1 - \gamma} F(\alpha - \gamma + 1; 2 - \gamma; x)
\]
Интегральное представление:
\[
	F(\alpha; \gamma; x) = 
	\frac{\Gamma(\gamma)}{\Gamma(\alpha) \Gamma(\gamma - \alpha)}
	\int\limits_0^1 t^{\alpha-1}(1 - t)^{\gamma - \alpha - 1} e^{tx} dt
\]
Отсюда можно получить:
\[
	F(\alpha; \gamma; x) = 
	e^x F(\gamma - \alpha; \gamma; -x) 
\]
Очевидное соотношение:
\[
	\frac{d F(\alpha; \gamma; x)}{dx} = \frac{\alpha}{\gamma} F(\alpha + 1; \gamma + 1; x)
\]

\subsection{Преобразование уравнения Куммера}

\[
	x y''_{xx} + (\gamma - x) y'_x - \alpha y = 0
\]
Выполняем замену:
\[
	x = f(t) \quad y = g(t) z(t)
\]
\[
	\begin{aligned}
	& y''_{xx} = \frac{1}{f'^2} (g z'' + 2 g' z' + g'' z) - \frac{f''}{f'^3} (gz' + g'z)
	\\
	& y'_x = \frac{1}{f'} (gz' + g'z)
	\end{aligned}
\]
\[
	\frac{f g}{f'^2} z'' +
	\left(
		\frac{2 g' f}{f'^2} - \frac{f'' f g}{f'^3}  +\frac{\gamma - f}{f'} g 
	\right) z' +
	\left(
	\frac{g'' f}{f'^2} - \frac{f'' f g'}{f'^3}  +\frac{\gamma - f}{f'} g' - \alpha
	\right) z = 0
\]
\[
	z'' +
	\left(
	\frac{2 g'}{g} - \frac{f''}{f'}  +\frac{\gamma - f}{f} f' 
	\right) z' +
	\left(
	\frac{g''}{g} - \frac{f''}{f'} \frac{g'}{g} +\frac{\gamma - f}{f'} \frac{g'}{g}  - \alpha
	\right) z = 0
\]
Добавляем одно условие:
\[
	\frac{2 g'}{g} - \frac{f''}{f'}  +\frac{\gamma - f}{f} f' = 0
\]
\[
	\ln g^2 - \ln f'  + \ln f^\gamma - f = \ln C
\]
\[
	g^2 = C f^{-\gamma} e^{f} f'
\]
\[
	\int g^2 dt = C \int f^{-\gamma} e^{f} df
\]
Вводим функцию:
\[
	h(t) = \frac{1}{g(t)}
\]
Получаем уравнение:
\[
	\frac{z''}{z} - 
	\frac{h''}{h} - \alpha
	= 0	
\]