\section{Решение линейных уравнений}

Поставим следующую задачу:
\[
	L\left[y\right] = \frac{\partial y}{\partial t} 
\]
\[
	y(x, t)\Big|_{t = 0} = y(x, 0) 
\]
$L$ -- произвольный линейный оператор.
Пусть собственные значения $Y(x, \lambda)$ удовлетворяет уравнению:
\[
	L\left[Y\right] = \lambda Y
\]
и если спектр $\lambda$ сплошной, существует функция $Y^{-1}(x, \lambda')$, такая что:
\[
	\int\limits_{-\infty}^{\infty} Y^{-1}(x, \lambda') Y(x, \lambda) dx = \delta(\lambda - \lambda')
\]
Будем искать $y$ в виде:
\[
	y = \int\limits_{-\infty}^{\infty} A(\lambda) e^{\lambda t} Y(x, \lambda)\, d\lambda
\]
Простой подстановкой можно проверить, что это выражение в самом деле является решением уравнения. $A(\lambda)$ определим из начальных условий:
\[
	y(x, 0) = \int\limits_{-\infty}^{\infty} A(\lambda) Y(x, \lambda)\, d\lambda
\]
\[
	A(\lambda) = \int\limits_{-\infty}^{\infty} y(x, 0) Y^{-1}(x, \lambda)\, dx
\]
Подставляем в решение и получаем:
\[
	y(x, t) = \int\limits_{-\infty}^{\infty} y(x', 0) \int\limits_{-\infty}^{\infty} e^{\lambda t}Y^{-1}(x', \lambda) Y(x, \lambda)\, d\lambda\, dx'
\]

\section{Решение линейных уравнений в случае дискретного спектра}

В случае дискретного спектра у нас есть система функций:
\[
	Y_1(x), Y_2(x), \ldots, Y_n(x), \ldots
\]
и соответствующая ей последовательность чисел:
\[
	\lambda_1, \lambda_2, \ldots
\]
Если удалось построить дополнительную систему функций
\[
	Y_1^{-1}(x), Y_2^{-1}(x), \ldots, Y_n^{-1}(x), \ldots
\]
такую что
\[
	\int\limits_{-\infty}^{\infty} Y_k^{-1}(x) Y_n(x) dx = \delta_{kn}
\]
То решение уравнения можно найти в виде:
\[
	y = \sum_{k=1}^{\infty} A_k e^{\lambda_k t} Y_k(x)
\]
\[
	A_k = \int\limits_{-\infty}^{\infty} Y_k^{-1}(x) y(x, 0) dx
\]
Получаем окончательно решение:
\[
	y(x, t) = \int\limits_{-\infty}^{\infty} y(x', 0) \sum_{k=1}^{\infty} e^{\lambda_k t} Y_k^{-1}(x') Y_k(x) dx'
\]