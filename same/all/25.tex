\section{Уравнение Больцмана в $\tau$-приближении}

Уравнение Больцмана (Лиувилля) в $\tau$-приближении:
\[
	\frac{d f}{d t} = \frac{\partial f}{\partial t} + \vec{v} \cdot \frac{\partial f}{\partial \vec{r}} + \vec{F} \cdot \frac{\partial f}{\partial \vec{p}} = \frac{f_0 - f}{\tau}
\]
Вдоль характеристики:
\[
	\begin{aligned}
	& \frac{d \vec{r}}{d t} = \vec{v} \\
	& \frac{d \vec{p}}{d t} = \vec{F} \\
	& \frac{d f}{d t} = \frac{f_0 - f}{\tau}
	\end{aligned}
\]
Пусть 
\[
	f = GV
\]
\[
	\frac{d G}{d t} + \frac{G}{\tau} = 0 \quad \Rightarrow \quad G = A e^{-t/\tau}
\]
\[
	\frac{d V}{d t} = \frac{1}{A \tau} e^{t/\tau} f_0 \quad \Rightarrow \quad V = \frac{1}{A \tau} \int\limits_{a}^t e^{t'/\tau}f_0(t', \vec{r}'(t'), \vec{p}'(t')) dt'  + V_0 
\]
\[
	f(t, \vec{r}, \vec{p})  = GV = \left(\frac{1}{\tau} \int\limits_{a}^t e^{t'/\tau}f_0(t', \vec{r}'(t'), \vec{p}'(t')) dt'  + V_0 A \right) e^{-t/\tau}
\]
Решение должно быть конечно при любых $t$, в том числе и при $t \to -\infty$. Отсюда следует:
\[
	V_0 A = \frac{1}{\tau} \int\limits_{-\infty}^{a} e^{t'/\tau}f_0(t', \vec{r}'(t'), \vec{p}'(t')) dt'
\]
Окончательно:
\[
	f(t, \vec{r}, \vec{p}) = \frac{1}{\tau} \int\limits_{-\infty}^t \exp \left(\frac{t' - t}{\tau}\right) f_0(t', \vec{r}'(t'), \vec{p}'(t')) dt'
\]
где $\vec{p}(t')$, $\vec{r}(t')$ удовлетворяют системе:
\[
	\begin{aligned}
	& \frac{d \vec{r}'}{d t'} = \vec{v} \\
	& \frac{d \vec{p}'}{d t'} = \vec{F}
	\end{aligned} 
\]
с начальными условиями:
\[
	\vec{r}'(t) = \vec{r} \quad \vec{p}'(t) = \vec{p}
\]
Рассмотрим систему с гамильтонианом:
\[
	H = H(\vec{p}, \vec{r}) = E(\vec{p}) + U(\vec{r})
\]
то есть явно не зависящую от времени. Уравнения движения в гамильтоновой форме:
\[
	\begin{aligned}
	& \frac{d \vec{r}}{d t} = \frac{\partial H}{\partial \vec{p}} \\
	& \frac{d \vec{p}}{d t} = - \frac{\partial H}{\partial \vec{r}}
	\end{aligned}
\]
В этом случае энергия сохраняется вдоль траектории-характеристики:
\[
	\frac{d H}{d t} = 
	\frac{\partial H}{\partial t} + \frac{\partial H}{\partial \vec{p}} \cdot \frac{\partial H}{\partial \vec{r}} - \frac{\partial H}{\partial \vec{r}} \cdot \frac{\partial H}{\partial \vec{p}} = 0
\]
Выберем 
\[
	f_0 = \frac{1}{N} \exp \left(- \frac{E}{kT}\right)
\]
И найдём плотность тока:
\[
	\vec{j}(t) = \int \vec{v}(\vec{p}) f(t) d^3r\, d^3p = \int \frac{\vec{v}(\vec{p})}{N \tau} \int\limits_{-\infty}^t \exp \left(\frac{t' - t}{\tau}\right) \exp \left(- \frac{E(\vec{p'}(t'))}{kT}\right) dt'\, d^3r\, d^3p
\]