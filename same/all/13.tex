\section{Движение частицы в материальной среде в нерелятивистском случае}

Рассмотрим материальную среду, частицы которой не взаимодействуют между собой. Уравнения движения для такой среды будут иметь вид уравнений Эйлера с нулевой правой частью:
\[
	\rho \Dff{\vec{v}}{t} = \rho \left(\dff{\vec{v}}{t} + (\vec{v}\cdot\nabla) \vec{v}\right) = 0
\]
\[
	\dff{\rho}{t} + \div (\rho \vec{v}) = 0
\]
То, что в среде находится частица учтём с помощью выражения:
\[
	\rho = \rho_f + m \delta(\vec{r} - \vec{\xi}(t))
\]
Из уравнения непрерывности следует:
\[
	\dff{\rho_f}{t} + \div (\rho_f \vec{v}) 
	- m \Dff{\vec{\xi}}{t} \cdot \nabla\delta(\vec{r} - \vec{\xi}(t)) 
	+ m \vec{v} \cdot \nabla\delta(\vec{r} - \vec{\xi}(t)) 
	+ m \delta(\vec{r} - \vec{\xi}(t)) \div{\vec{v}} = 0
\]
Можно предположить (к сожалению мне это казалось абсолютно точным, но вдруг я где-то ошибся), что отсюда следует:
\[
	\dff{\rho_f}{t} + \div (\rho_f \vec{v}) = 0
\]
\[
	\div \vec{v} = 0 \text{ но только в точке с частицей}
\]
\[
	\vec{v} = \Dff{\vec{\xi}}{t} \text{ но только в точке с частицей}
\]
Разделяем выражения и интегрируем по $\vec{r}$, учитывая, что на $\vec{r} = \vec{\xi}(t)$ $\vec{v} = d\vec{\xi}/dt$:
\[
	m\Dfs{\vec{\xi}}{t} = - \int \rho_f \left(\dff{\vec{v}}{t} + (\vec{v}\cdot\nabla) \vec{v}\right) dV
\]
На деле как легко видеть скорости частиц не меняются при движении, если поле скоростей непрерывно. Интерес представляет один единственный случай, когда поле скоростей терпит разрыв. В этом случае частица догоняет и сталкивается с частицами среды и, как следствие, движется уже ускоренно. Здесь есть несколько важных и нетривиальных моментов. Очень хочется, чтобы разрыв был в одной единственной точке -- в точке с частицей. Но если разрыв конечен, то для частицы как материальной точки производные от скорости обратятся в нуль и ничего не будет наблюдаться. Если предположить, что в среде существует фронт или линия на которой поле скоростей терпит разрыв, то расположение этой линии или фронта может быть вообще говоря произвольным. Это путает все карты.
Как облечь всё изложенное выше в математическую форму я пока не знаю. По этой причине на задачу пришлось забить.