\section{Колебания с произвольной возбуждающей силой}

Уравнение колебаний:
\[
	\ddot{x} + \omega_0^2 x = f(t)
\]
\[
	x(0) = x_0, \quad \dot{x}(0) = \dot{x}_0
\]
Решение будем искать в классе непрерывных функций. Рассмотрим вспомогательную задачу:
\[
	\ddot{y} + \omega_0^2 y = \eta(t - t')
\]
\[
	y(0) = 0, \quad \dot{y}(0) = 0 \quad t' > 0
\]
Её решение с непрерывной первой производной, как легко убедиться простой подстановкой:
\[
	y = 
	\begin{cases}
	0 & t<t' \\
	\frac{1 - \cos(\omega_0 (t - t'))}{\omega_0^2} & t \geqslant t'
	\end{cases}
\]
Тогда решением задачи:
\[
	\ddot{z} + \omega_0^2 z = \delta(t - t')
\]
\[
	z(0) = 0, \quad \dot{z}(0) = 0 \quad t' > 0
\]
будет
\[
	z(t, t') = 
	\begin{cases}
	0 & t<t' \\
	\frac{\sin(\omega_0 (t - t'))}{\omega_0} & t \geqslant t'
	\end{cases}
\]
А решение исходной задачи:
\[
	x(t) = x_0 \cos (\omega_0 t) + \frac{\dot{x}_0}{\omega_0} \sin (\omega_0 t) + \int\limits_{0}^{\infty} z(t, t') f(t') dt' = 
	x_0 \cos (\omega_0 t) + \frac{\dot{x}_0}{\omega_0} \sin (\omega_0 t) + \frac{1}{\omega_0} \int\limits_{0}^{t} \sin(\omega_0 (t - t')) f(t') dt'
\]
Есть одно замечание. Можно показать, что в случае задачи относительно $y$  $\dot{y}$ будет непрерывно, если непрерывно само решение? а $\dot{y}$ ограничено сверху на любом конечном промежутке (достаточно домножить на $\dot{y}$ и проинтегрировать в произвольном диапазоне), а для задачи относительно $z$, если решение непрерывно, то первая производная терпит разрыв. В то же время без этих требований построить решения затруднительно, особенно $y$.