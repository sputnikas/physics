\section{Немного об ОТО}

\subsection{Базисы}

Произвольный базис -- ковариантный:
\[
	\vec{e}_i
\]
Для данного ковариантного базиса можно построить контрвариантный базис:
\[
	\vec{e}^j \cdot \vec{e}_i = \delta^j_i
\] 

\subsection{Метрический тензор}

\[
	g_{ij} = \vec{e}_i \cdot \vec{e}_j
	\quad
	g^{ij} = \vec{e}^i \cdot \vec{e}^j
\]
С точки зрения данного определения метрический тензор представляет собой компоненты векторов ко-(контр-) вариантного базиса в разложении по контр-(ко-)  вариантному базису:
\[
	g^{ip} g_{pj} = g^{ip} \vec{e}_p \cdot \vec{e}_j = \vec{e}^i \cdot \vec{e}_j = \delta^i_j
\]
С точки зрения такого определения метрический тензор выглядит симметричным, но:
\[
	g_{ij} = g^*_{ji}
\]
Представить себе другой случай пространства с кручением я не могу. Далее метрика везде будет действительной.

\subsection{Символы Кристоффеля}

Определение для 2 рода:
\[
	\frac{\partial \vec{e}_i}{\partial q^j} = \frac{\partial^2 \vec{r}}{\partial q^j \partial q^i} = \Gamma^k_{ij} \vec{e}_k
\]
1 род:
\[
	\Gamma_{p, ij} = g_{pk} \Gamma^{k}_{ij}
\]
\[
	\vec{e}_p \cdot \frac{\partial \vec{e}_i}{\partial q^j} + 
	\frac{\partial \vec{e}_p}{\partial q^j} \cdot \vec{e}_p =
	\Gamma_{ij}^k g_{pk} + \Gamma^k_{pj} g_{ki}
\]
\[
	\begin{aligned}
	& \frac{\partial g_{pi}}{\partial q^j} = \Gamma_{p,ij} + \Gamma_{i,pj} \\
	& \frac{\partial g_{ij}}{\partial q^p} = \Gamma_{i,jp} + \Gamma_{j,ip} \\
	& \frac{\partial g_{jp}}{\partial q^i} = \Gamma_{j,pi} + \Gamma_{p,ji}
	\end{aligned}
\]