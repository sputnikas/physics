\section{Система Молчанова-Селькова и всё что с ней связано}

В организмах как известно протекают различные химические реакции. Представление о мозге как о большой автоколебательной системе привело меня к поиску простейшей модели связанной автоколебательной системы. Случайно мне встретилась простейшая из автоколебательных систем -- система Молчанова-Селькова (речь шла о химических реакциях, но здесь приводятся только уравнения):
\[
	\begin{cases}
	\frac{\partial \xi}{\partial t} = \beta - \xi \eta^2 \\
	\frac{\partial \eta}{\partial t} = \xi \eta^2 - \xi
	\end{cases}
\]
Обычно $\beta$ -- внешний поток субстрата -- считается постоянным и ($\beta > 0$). Пока остановимся на этом случае.

\subsection{Устойчивость по Ляпунову}

\[
	\begin{cases}
	0 = \beta - \xi_0 \eta_0^2 \\
	0 = \xi_0 \eta_0^2 - \xi_0
	\end{cases}
	\quad 
	\Rightarrow
	\quad
	\begin{cases}
	\eta_0 = \pm 1 \\
	\xi_0 = \beta
	\end{cases}
\]
Определим теперь характер точек:
\[
	\xi = \xi_0 + \delta \xi \quad \eta = \eta_0 + \delta \eta
\]
\[
	\begin{cases}
	\frac{\partial \delta\xi}{\partial t} = - 2 \xi_0 \eta_0 \delta \eta - \eta_0^2 \delta \xi = \mp 2 \beta \delta \eta - \delta \xi\\
	\frac{\partial \delta\eta}{\partial t} = 2 \xi_0 \eta_0 \delta \eta + (\eta_0^2 - 1) \delta \xi = \pm 2 \beta \delta \eta
	\end{cases}
\]
\[
	\delta \eta = A_\eta e^{\pm 2 \beta t} \quad \delta \xi = A_\xi e^{-t} - A_\eta e^{\pm 2 \beta t}
\]
Откуда видно, что точка $(1, \beta)$ является седловой, точка $(-1, \beta)$ представляет собой устойчивый узел.