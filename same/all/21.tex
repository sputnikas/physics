\section{Общая форма уравнений электродинамики}

Пока пространство-время обычное -- плоское. Интервал:
\[
	ds^2 = dx^i dx_i
\]

Будем исходить из того, что электромагнитное поле задаётся четырёхпотенциалом $A_i$. Силовая характеристика тензорная:
\[
	F_{ij} = \frac{\partial A_j}{\partial x^i} -  \frac{\partial A_i}{\partial x^j}
\]
Отсюда следует одно уравнение электродинамики:
\[
	\frac{\partial F_{ij}}{\partial x^k} + \frac{\partial F_{jk}}{\partial x^i} + \frac{\partial F_{ki}}{\partial x^j} = 0
\]
Или оно же, но через дуальный тензор:
\[
	\frac{\partial \tilde{F}^{ij}}{\partial x^j} = 0 
\]
А также то, что тензор $F_{ij}$ антисимметричный:
\[
	F_{ij} = - F_{ji}
\] 
Существуют четыре инварианта, которыми определяются уравнения электродинамики:
\[
	S_1 = F^{ij}F_{ij} \text{ характеризует поле}
\]
\[
	S_2 = \tilde{F}^{ij}F_{ij} \text{ характеризует поле}
\]
\[
	S_3 = j^i A_i \text{ характеризует взаимодействие поля с источниками}
\]
\[
	S_4 = A^i A_i \text{ ещё одна характеристика поля, в квантовой теории позволяет учесть массу переносчика взаимодействия}
\]
Лагранжева плотность:
\[
	L = L(S_1, S_2, S_3, S_4)
\]
Варьируем поле $A_i$ (далее везде отброшены слагаемые типа дивергенций, по которым можно проинтегрировать вариацию лагранжевой плотности):
\[
	\delta L = 
	\frac{\partial L}{\partial S_1} \delta S_1 + 
	\frac{\partial L}{\partial S_2} \delta S_2 + 
	\frac{\partial L}{\partial S_3} \delta S_3 + 
	\frac{\partial L}{\partial S_4} \delta S_4
\]
\begin{enumerate}
	\item 
	\[
		\begin{gathered}
		\frac{\partial L}{\partial S_1} \delta S_1 = 2 F^{ij} \frac{\partial L}{\partial S_1} \delta F_{ij} = 
		2 F^{ij} \frac{\partial L}{\partial S_1} \left(\frac{\partial \delta A_j}{\partial x^i} - \frac{\partial \delta A_i}{\partial x^j} \right)
		= \\ =
		[\text{раскрываем и меняем местами во втором слагаемом индексы $i, j$} ] = \\ =
		4 F^{ij} \frac{\partial L}{\partial S_1} \frac{\partial \delta A_j}{\partial x^i} = 
		4 \frac{\partial }{\partial x^i} \left(F^{ij} \frac{\partial L}{\partial S_1} \delta A_j \right) - 4 \frac{\partial}{\partial x^i} \left(F^{ij} \frac{\partial L}{\partial S_1} \right) \delta A_j =
		4 \frac{\partial}{\partial x^j} \left(F^{ij} \frac{\partial L}{\partial S_1} \right) \delta A_i
		\end{gathered}
	\]
	\item 
	\[
		\begin{gathered}
		\frac{\partial L}{\partial S_2} \delta S_2 = \frac{\partial L}{\partial S_2} \frac{1}{2}\left(\epsilon^{ijkl} F_{ij} \delta F_{kl} +  \epsilon^{ijkl} F_{kl} \delta F_{ij} \right)
		= \frac{\partial L}{\partial S_2} \frac{1}{2}\left(\epsilon^{klij} F_{kl} \delta F_{ij} +  \epsilon^{ijkl} F_{kl} \delta F_{ij} \right)
		= \frac{\partial L}{\partial S_2} \frac{1}{2}(\epsilon^{klij} + \epsilon^{ijkl}) F_{kl} \delta F_{ij} = 
		\\ =
		[\text{так как перестановка чётная}] 
		= 2 \frac{1}{2} \epsilon^{ijkl} F_{kl} \frac{\partial L}{\partial S_2} \delta F_{ij}
		= 2 \tilde{F}^{ij} \frac{\partial L}{\partial S_2} \left(\frac{\partial \delta A_j}{\partial x^i} - \frac{\partial \delta A_i}{\partial x^i} \right)
		= [\text{также как и выше}] = \\
		= 4 \frac{\partial}{\partial x^j} \left(\tilde{F}^{ij} \frac{\partial L}{\partial S_2} \right) \delta A_i
		\end{gathered}
	\]
	\item
	\[
		\frac{\partial L}{\partial S_3} \delta S_3 = j^i \frac{\partial L}{\partial S_3} \delta A_i
	\]
	\item 
	\[
		\frac{\partial L}{\partial S_4} \delta S_4 = 2 A^i \frac{\partial L}{\partial S_4}  \delta A_i
	\]
\end{enumerate}
Получаем:
\[
	\delta L = 
	\left(
	4 \frac{\partial}{\partial x^j} \left(F^{ij} \frac{\partial L}{\partial S_1} \right) + 
	4 \frac{\partial}{\partial x^j} \left(\tilde{F}^{ij} \frac{\partial L}{\partial S_2} \right) +
	\frac{\partial L}{\partial S_3} j^i + 
	\frac{\partial L}{\partial S_4} 2 A^i
	\right) \delta A_i
\]
Вспоминая выражение для действия и равенство нулю вариации действия получаем:
\[
	4 \frac{\partial}{\partial x^j} \left(F^{ij} \frac{\partial L}{\partial S_1} \right) + 
	4 \frac{\partial}{\partial x^j} \left(\tilde{F}^{ij} \frac{\partial L}{\partial S_2} \right) +
	\frac{\partial L}{\partial S_3} j^i + 
	\frac{\partial L}{\partial S_4} 2 A^i = 0
\]
Для электродинамики Максвелла:
\[
	L = - \frac{1}{4\mu_0} S_1 - S_3
\]
Следующее выражение имеет какое-то отношение к Шрёдингеру ($m$ -- масса фотона):
\[
	L = - \frac{1}{4\mu_0} S_1 - S_3 - \frac{m^2 c^2}{2 \hbar^2 \mu_0} S_4
\]
Лагранжева плотность Борна:
\[
	L = \frac{1}{\mu_0 \lambda^2} \left( 1 - \sqrt{1 + \frac{\lambda^2}{2}  S_1} \right) - S_3
\]
Лагранжева плотность Борна-Инфельда:
\[
	L = \frac{1}{\mu_0 \lambda^2} \left( 1 - \sqrt{1 + \frac{\lambda^2}{2}  S_1 - \frac{\lambda^4}{16}  S_2^2} \right) - S_3
\]
Для нелинейной теории Шрёдингера:
\[
	L = \frac{1}{2 \mu_0 \lambda^2} \ln \left(1 - \frac{\lambda^2}{2} S_1\right) - S_3
\]