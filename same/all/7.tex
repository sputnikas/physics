\section{Оператор импульса в сферических координатах и коммутационные соотношения}

Нам известно, что оператор импульса это градиент с коэффициентом. Градиент в сферических координатах:
\[
	\nabla = \left\{\frac{\partial}{\partial r}, \frac{1}{r}\frac{\partial}{\partial \theta}, \frac{1}{r \sin \theta}\frac{\partial}{\partial \alpha} \right\}
\]
Оператор Лапласа, который фигурирует в уравнении Шрёдингера:
\[
	\Delta = \nabla \cdot \nabla = \frac{\partial^2}{\partial r^2} + \frac{1}{r^2}\frac{\partial^2}{\partial \theta^2} +\frac{1}{r^2 \sin^2 \theta}\frac{\partial^2}{\partial \alpha^2}
\]
Но умные люди давно подсчитали, что:
\[
	\Delta = 
	\frac{1}{r^2}\frac{\partial}{\partial r} \left(r^2 \frac{\partial}{\partial r}\right)
	+ \frac{1}{r^2 \sin \theta} \frac{\partial}{\partial \theta} \left(\sin \theta \frac{\partial}{\partial \theta}\right)
	+ \frac{1}{r^2 \sin^2 \theta} \frac{\partial^2}{\partial \alpha^2}
\]
и результаты, вообще говоря, не совпадают. Где мы ошиблись? Ответ на этот вопрос чрезвычайно прост: мы не должны были забывать об ортах. Запишем градиент иначе:
\[
	\nabla = \vec{e}_r \frac{\partial}{\partial r} + \vec{e}_\theta \frac{1}{r}\frac{\partial}{\partial \theta} + \vec{e}_\alpha \frac{1}{r \sin \theta}\frac{\partial}{\partial \alpha}
\]
И воспользуемся формулами:
\[
\begin{aligned}
& \frac{\partial \vec{e}_r}{\partial r} = 0 \\
& \frac{\partial \vec{e}_r}{\partial \theta} = \vec{e}_\theta \\
& \frac{\partial \vec{e}_r}{\partial \alpha} = \sin \theta\, \vec{e}_\alpha
\end{aligned}
\quad
\begin{aligned}
& \frac{\partial \vec{e}_\theta}{\partial r} = 0 \\
& \frac{\partial \vec{e}_\theta}{\partial \theta} = -\vec{e}_r \\
& \frac{\partial \vec{e}_\theta}{\partial \alpha} = \cos \theta\, \vec{e}_\alpha
\end{aligned}
\quad
\begin{aligned}
& \frac{\partial \vec{e}_\alpha}{\partial r} = 0 \\
& \frac{\partial \vec{e}_\alpha}{\partial \theta} = 0 \\
& \frac{\partial \vec{e}_\alpha}{\partial \alpha} = -\sin\theta \, \vec{e}_r -\cos \theta\, \vec{e}_\theta
\end{aligned}
\]
Тогда
\[
	\begin{gathered}
		\Delta = 
		\vec{e}_r\cdot \frac{\partial}{\partial r} 
		\left(\vec{e}_r \frac{\partial}{\partial r} + \vec{e}_\theta \frac{1}{r}\frac{\partial}{\partial \theta} + \vec{e}_\alpha \frac{1}{r \sin \theta}\frac{\partial}{\partial \alpha}\right)
		+ \\ +
		\vec{e}_\theta\cdot \frac{1}{r}\frac{\partial}{\partial \theta}
		\left(\vec{e}_r \frac{\partial}{\partial r} + \vec{e}_\theta \frac{1}{r}\frac{\partial}{\partial \theta} + \vec{e}_\alpha \frac{1}{r \sin \theta}\frac{\partial}{\partial \alpha}\right)
		+
		\vec{e}_\alpha\cdot \frac{1}{r \sin \theta}\frac{\partial}{\partial \alpha}
		\left(\vec{e}_r \frac{\partial}{\partial r} + \vec{e}_\theta \frac{1}{r}\frac{\partial}{\partial \theta} + \vec{e}_\alpha \frac{1}{r \sin \theta}\frac{\partial}{\partial \alpha}\right) 
		= \\ =
		\frac{\partial^2}{\partial r^2} +
		\frac{1}{r} \frac{\partial}{\partial r} +
		\frac{1}{r^2} \frac{\partial^2}{\partial \theta^2} +
		\frac{1}{r} \frac{\partial}{\partial r} +
		\frac{1}{r^2} \frac{\cos \theta}{\sin \theta} \frac{\partial}{\partial \theta} +
		\frac{1}{r^2 \sin^2 \theta} \frac{\partial^2}{\partial \alpha^2}
		= \\ =
		\frac{1}{r^2}\frac{\partial}{\partial r} \left(r^2 \frac{\partial}{\partial r}\right)
		+ \frac{1}{r^2 \sin \theta} \frac{\partial}{\partial \theta} \left(\sin \theta \frac{\partial}{\partial \theta}\right)
		+ \frac{1}{r^2 \sin^2 \theta} \frac{\partial^2}{\partial \alpha^2}
	\end{gathered}
\]
Итак, оператор импульса в сферических координатах:
\[
	\hat{\vec{p}} = -i\hbar \left(\vec{e}_r \frac{\partial}{\partial r} + \vec{e}_\theta \frac{1}{r}\frac{\partial}{\partial \theta} + \vec{e}_\alpha \frac{1}{r \sin \theta}\frac{\partial}{\partial \alpha} \right)
\]
Соотношения коммутации в случае декартовых координат:
\[
	[\hat{p}_i, \hat{p}_j] = \hat{p}_i \hat{p}_j - \hat{p}_j \hat{p}_i = 0
\]
С точки зрения векторов это означает, что в терминах полного произведения векторов матрица $\hat{\vec{p}}\hat{\vec{p}}$ симметрична.
\[
	\begin{gathered}
	\nabla\nabla = 
	\vec{e}_r \frac{\partial}{\partial r} 
	\left(\vec{e}_r \frac{\partial}{\partial r} + \vec{e}_\theta \frac{1}{r}\frac{\partial}{\partial \theta} + \vec{e}_\alpha \frac{1}{r \sin \theta}\frac{\partial}{\partial \alpha}\right)
	+ \\ +
	\vec{e}_\theta \frac{1}{r}\frac{\partial}{\partial \theta}
	\left(\vec{e}_r \frac{\partial}{\partial r} + \vec{e}_\theta \frac{1}{r}\frac{\partial}{\partial \theta} + \vec{e}_\alpha \frac{1}{r \sin \theta}\frac{\partial}{\partial \alpha}\right)
	+
	\vec{e}_\alpha \frac{1}{r \sin \theta}\frac{\partial}{\partial \alpha}
	\left(\vec{e}_r \frac{\partial}{\partial r} + \vec{e}_\theta \frac{1}{r}\frac{\partial}{\partial \theta} + \vec{e}_\alpha \frac{1}{r \sin \theta}\frac{\partial}{\partial \alpha}\right) 
	= \\ =
	\vec{e}_r \vec{e}_r \frac{\partial^2}{\partial r^2} +
	\vec{e}_r \vec{e}_\theta \frac{\partial}{\partial r} \frac{1}{r}\frac{\partial}{\partial \theta} +
	\vec{e}_r \vec{e}_\alpha \frac{\partial}{\partial r} \frac{1}{r \sin \theta}\frac{\partial}{\partial \alpha} +
	\vec{e}_\theta \vec{e}_r \left(\frac{1}{r} \frac{\partial^2}{\partial \theta\, \partial r} - \frac{1}{r^2} \frac{\partial}{\partial \theta} \right) + \\ +
	\vec{e}_\theta \vec{e}_\theta \left(\frac{1}{r} \frac{\partial}{\partial r} +
	\frac{1}{r^2} \frac{\partial^2}{\partial \theta^2} \right) +
	\vec{e}_\theta \vec{e}_\alpha \left(\frac{1}{r}\frac{\partial}{\partial \theta} \frac{1}{r \sin \theta}\frac{\partial}{\partial \alpha}\right) +
	\vec{e}_\alpha \vec{e}_r \left(\frac{1}{r \sin \theta} \frac{\partial}{\partial \alpha} \frac{\partial}{\partial r} - \frac{1}{r^2 \sin \theta} \frac{\partial}{\partial \alpha} \right) + \\ +
	\vec{e}_\alpha \vec{e}_\theta \left(\frac{1}{r \sin \theta}\frac{\partial}{\partial \alpha}\frac{1}{r}\frac{\partial}{\partial \theta} - \frac{\cos \theta}{r \sin \theta} \frac{1}{r \sin \theta}\frac{\partial}{\partial \alpha}\right) +
	\vec{e}_\alpha \vec{e}_\alpha \left(\frac{1}{r} \frac{\partial}{\partial r} + \frac{\cos \theta}{r^2 \sin \theta}\frac{\partial}{\partial \theta} + \frac{1}{r^2 \sin^2 \theta} \frac{\partial^2}{\partial \alpha^2} \right)
	\end{gathered}
\]
Выпишем матрицу $\nabla\nabla$:
\[
	\begin{pmatrix}
		\cfrac{\partial^2}{\partial r^2} && 
		\cfrac{\partial}{\partial r} \cfrac{1}{r}\cfrac{\partial}{\partial \theta} &&
		\cfrac{\partial}{\partial r} \cfrac{1}{r \sin \theta}\cfrac{\partial}{\partial \alpha} \\\\
		\cfrac{1}{r} \cfrac{\partial^2}{\partial \theta\, \partial r} - \cfrac{1}{r^2} \cfrac{\partial}{\partial \theta} &&
		\cfrac{1}{r} \cfrac{\partial}{\partial r} + \cfrac{1}{r^2} \cfrac{\partial^2}{\partial \theta^2} &&
		\cfrac{1}{r}\cfrac{\partial}{\partial \theta} \cfrac{1}{r \sin \theta}\cfrac{\partial}{\partial \alpha} \\\\
		\cfrac{1}{r \sin \theta} \cfrac{\partial}{\partial \alpha} \cfrac{\partial}{\partial r} - \cfrac{1}{r^2 \sin \theta} \cfrac{\partial}{\partial \alpha} &&
		\cfrac{1}{r \sin \theta}\cfrac{\partial}{\partial \alpha}\cfrac{1}{r}\cfrac{\partial}{\partial \theta} - \cfrac{\cos \theta}{r \sin \theta} \cfrac{1}{r \sin \theta}\cfrac{\partial}{\partial \alpha} &&
		\cfrac{1}{r} \cfrac{\partial}{\partial r} + \cfrac{\cos \theta}{r^2 \sin \theta}\cfrac{\partial}{\partial \theta} + \cfrac{1}{r^2 \sin^2 \theta} \cfrac{\partial^2}{\partial \alpha^2}
	\end{pmatrix}
\]
Она симметрична. Интересно, что данный результат нельзя получить, если рассматривать покомпонентные коммутационные соотношения.