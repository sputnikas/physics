\section{Решение уравнения Лапласа}

Рассмотрим задачу:
\[
	\Delta u = s(\vec{r})
\]

Граничные условия пока не приводятся, так как теорема единственности неизвестна.

\subsection{Теорема единственности}

Пусть $u$ и $v$ удовлетворяют уравнениям:
\[
	\Delta u = s(\vec{r}) \quad \Delta v = s(\vec{r})
\]
Рассмотрим интеграл:
\[
	\int\limits_V (u \Delta v - v \Delta u) dV = \int\limits_V (u - v) s(\vec{r}) dV
\]
\[
	u \Delta v = u \nabla \cdot \nabla v = \nabla \cdot (u \nabla v) - (\nabla u) \cdot (\nabla v)
\]
\[
	u \Delta v - v \Delta u = \div (u \nabla v - v \nabla u)
\]
\[
	\int\limits_V \div (u \nabla v - v \nabla u) dV = 
	\oint\limits_S \left(u \frac{\partial v}{\partial n} - v \frac{\partial u}{\partial n}\right) dS =
	\int\limits_V (u - v) s(\vec{r}) dV
\]
Следовательно для единственности решения достаточно задать на некоторой поверхности или совокупности поверхностей:
\[
	u = f(\vec{r}) \quad \frac{\partial u}{\partial n} = g(\vec{r}) \text{ на } S
\]
В самом деле в этом случае из интеграла следует:
\[
	\int\limits_V (u - v) s(\vec{r}) dV
\]
И из произвольности функции $s$ получаем:
\[
	u = v
\]

\subsection{Решение $N$-мерного уравнения Лапласа с $\delta$-образной правой частью}

\[
	\Delta G = \delta (\vec{r} - \vec{r}')
\]

В интегральной форме:
\[
	\oint\limits_S \grad G \cdot d\vec{S} = \int\limits_V \delta (\vec{r} - \vec{r}') dV
\]

Задача обладает сферической симметрией и относительно $\vec{R} = \vec{r} - \vec{r}'$, когда $S$ это сфера радиуса $R$:
\[
	S_N \frac{\partial G}{\partial R} = 1
\]
\[
	\frac{\partial G}{\partial R} = \frac{\Gamma(N/2)}{2 \pi^{N/2}} \frac{1}{R^{N - 1}}
\]
\[
	G = 
	\begin{cases}
	- \frac{\Gamma(N/2)}{2(N - 2) \pi^{N/2}} \frac{1}{R^{N - 2}} + C & N \ne 2 \\
	\frac{1}{2\pi} \ln R + C & N = 2
	\end{cases}
\]
При $N > 2$ будем считать $C = 0$.

\subsection{Общее решение уравнения Лапласа}

Уравнение:
\[
	\Delta u = s(\vec{r})
\]
При граничных условиях:
\[
	u = f(\vec{r}) \quad \frac{\partial u}{\partial n} = g(\vec{r}) \text{ на } S
\]
Рассмотрим интеграл:
\[
	\int\limits_V (u \Delta G - G \Delta u) dV = 
	\oint\limits_S \left(u \frac{\partial G}{\partial n} - G \frac{\partial u}{\partial n}\right) dS
\]
С учётом:
\[
	\Delta G(\vec{r}, \vec{r}') = \delta (\vec{r} - \vec{r}')
\]
Получаем:
\[
	u(\vec{r}') = \int\limits_V G(\vec{r}, \vec{r}') s(\vec{r}) dV + \oint\limits_S \left(f(\vec{r}) \frac{\partial G(\vec{r}, \vec{r}')}{\partial n} - G(\vec{r}, \vec{r}') g(\vec{r}) \right) dS
\]
\[
	\begin{aligned}
	&
	\int\limits_V G(\vec{r}, \vec{r}') s(\vec{r}) dV
	\text{ -- объёмный потенциал}
	\\
	& - \oint\limits_S G(\vec{r}, \vec{r}') g(\vec{r}) dS 
	\text{ -- потенциал простого слоя}
	\\
	& \oint\limits_S f(\vec{r}) \frac{\partial G(\vec{r}, \vec{r}')}{\partial n} dS
	\text{ -- потенциал двойного слоя}
	\end{aligned}
\]

