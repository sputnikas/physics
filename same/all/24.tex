\section{О проводниках}

Классическая теория проводимости основана на уравнении движения электрона, записанном в следующем виде:
\[
	\frac{d\vec{p}}{dt} = - \nu \vec{p} + q\vec{E} + q\vec{v} \times \vec{B}
\]
Трудный вопрос будет ли работать это уравнение в релятивистском случае, поэтому оставим его пока в покое и рассмотрим нерелятивистский случай (тем более, что трудно представить себе среду с релятивистскими электронами и кристаллической формой). Поэтому:
\[
	\vec{p} = m \vec{v}
\]
\[
	\vec{j} = q n \vec{v}
\]
Для классического проводника:
\[
	\frac{d\vec{p}}{dt} = 0
\]
\[
	- \nu m \vec{v} + q \vec{E} + q\vec{v} \times \vec{B} = 0
\]
\[
	\nu m \vec{v} + q \vec{B} \times \vec{v} = q \vec{E} 
\]
Для простоты выкладок обозначим:
\[
	\lambda = \frac{\nu m}{q}
\]
\[
	\begin{pmatrix}
	\lambda & B_z & -B_y \\
	-B_z & \lambda & B_x \\
	B_y & -B_x & \lambda
	\end{pmatrix}
	\begin{pmatrix}
	v_x \\
	v_y \\
	v_z
	\end{pmatrix}
	= 
	\begin{pmatrix}
	E_x \\
	E_y \\
	E_z
	\end{pmatrix}
\]
Добавим ещё немного обозначений: $\hat{I}$ -- единичная матрица,
\[
	\hat{B} = 
	\begin{pmatrix}
	0 & B_z & -B_y \\
	-B_z & 0 & B_x \\
	B_y & -B_x & 0
	\end{pmatrix}
	\quad
	\hat{B}^2 = 
	\begin{pmatrix}
	- B_z^2 - B_y^2	& B_x B_y 			& B_x B_z \\
	B_x B_y 		& - B_z^2 - B_x^2 	& B_y B_z \\
	B_x B_z 		& B_y B_z  			& - B_x^2 - B_y^2
	\end{pmatrix}
	\quad
	B^2 = B_x^2 + B_y^2 + B_z^2
\]
Уравнение:
\[
	(\lambda \hat{I} + \hat{B}) \vec{v} = \vec{E}
\]
\[
	(\lambda \hat{I} + \hat{B})^{-1} = \frac{1}{\lambda(\lambda^2 + B^2)} 
	\begin{pmatrix}
	B_x^2 + \lambda^2   	 & B_x B_y - \lambda B_z 	& B_x B_z + \lambda B_y 	\\
	B_x B_y + \lambda B_z	 & B_y^2 + \lambda^2    	& B_y B_z - \lambda B_x 	\\
	B_x B_z - \lambda B_y	 & B_y B_z + \lambda B_x    & B_z^2 + \lambda^2		 	\\
	\end{pmatrix} =
	\frac{1}{\lambda(\lambda^2 + B^2)}  ((\lambda^2 + B^2) \hat{I} + \hat{B}^2 - \lambda \hat{B})
\]
\[
	\vec{v} = \frac{\vec{E}}{\lambda} + \frac{1}{\lambda(\lambda^2 + B^2)} (\hat{B}^2 - \lambda \hat{B}) \vec{E}
	=
	\frac{\vec{E}}{\lambda} + \frac{1}{\lambda(\lambda^2 + B^2)} (\vec{B}\times(\vec{B} \times \vec{E}) + \lambda \vec{E} \times \vec{B})
	= 
	\frac{\lambda\vec{E}}{\lambda^2 + B^2} + \frac{1}{\lambda(\lambda^2 + B^2)} (\vec{B} (\vec{E} \cdot \vec{B} ) + \lambda \vec{E} \times \vec{B})
\]
Отсюда следует один интересный вывод, если изложенный механизм действительно имеет место, то проводимость вдоль направления $\vec{E}$, когда $\vec{E} \perp \vec{B}$:
\[
	\sigma = \frac{qn \lambda}{\lambda^2 + B^2} = \frac{q^2 n}{m\nu \left(1 + \frac{m^2\nu^2 B^2}{q^2}\right)}
\]
и также существуют токи в направлении перпендикулярном $\vec{E}$ и $\vec{B}$.
Если $\vec{E} \parallel \vec{B}$:
\[
	\vec{v} = 
	\frac{\lambda\vec{E}}{\lambda^2 + B^2} + \frac{B^2 \vec{E}}{\lambda(\lambda^2 + B^2)} = \frac{\vec{E}}{\lambda}
\]
\[
	\sigma = \frac{qn}{\lambda} = \frac{q^2 n}{m\nu}
\]
Найдём параметр $\lambda$ для меди:
\[
	\lambda = \frac{qn}{\sigma} = qn\rho_e = -\frac{ k e N_A \rho\rho_e}{M} = -\frac{ 1\cdot 1{,}6\cdot 10^{-19} \cdot 6{,}02\cdot 10^{23} \cdot  1{,}75 \cdot 10^{-8} \cdot  8{,}9\cdot 10^3 }{63{,}5\cdot 10^{-3}} \approx - 2{,}4 \cdot 10^2 \text{ Тл}
\]
Лучшим чем медь проводником является серебро. Для него:
\[
	\lambda = -\frac{ 1\cdot 1{,}6\cdot 10^{-19} \cdot 6{,}02\cdot 10^{23} \cdot  1{,}5 \cdot 10^{-8} \cdot  8{,}9\cdot 10^3 }{63{,}5\cdot 10^{-3}} \approx - 2{,}0 \cdot 10^2 \text{ Тл}
\]
Для всех остальных веществ этот параметр существенно больше и как следствие:
\[
	\vec{v} \approx
	\frac{\vec{E}}{\lambda} + \frac{1}{\lambda^3} (\vec{B} (\vec{E} \cdot \vec{B} ) + \lambda \vec{E} \times \vec{B}) \approx \frac{\vec{E}}{\lambda} 
\]