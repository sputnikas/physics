\documentclass[a4paper,10pt]{article}
\usepackage[T2A]{fontenc}
\usepackage[utf8]{inputenc}%включаем свою кодировку: koi8-r или utf8 в UNIX, cp1251 в 
\usepackage[english,russian]{babel}%используем русский и английский языки с переносами
\usepackage{amssymb,amsfonts,amsmath,mathtext, esint,cite,enumerate,float,bm,hyperref}
\usepackage{geometry}
\geometry{left=1cm}
\geometry{right=1cm}
\geometry{top=1cm}
\geometry{bottom=2cm}
\usepackage{tikz}
\usepackage{longtable}
\allowdisplaybreaks

\newcommand{\D}{\mathrm{d}}
% Производные
\newcommand{\dsl}[2]{{\partial #1}/{\partial #2}}
\newcommand{\df}[1]{\cfrac{\partial}{\partial #1}}
\newcommand{\dff}[2]{\frac{\partial #1}{\partial #2}}
\newcommand{\dfs}[2]{\frac{\partial^2 #1}{\partial #2^2}}
\newcommand{\dfss}[3]{\frac{\partial^2 #1}{\partial #2\, \partial #3}}
\newcommand{\Df}[1]{\frac{d}{d #1}}
\newcommand{\Dff}[2]{\frac{d #1}{d #2}}
\newcommand{\Dfs}[2]{\frac{d^2 #1}{d #2^2}}
\newcommand{\cDf}[1]{\cfrac{d}{d #1}}
\newcommand{\cDff}[2]{\cfrac{d #1}{d #2}}
\newcommand{\cDfs}[2]{\cfrac{d^2 #1}{d #2^2}}
\newcommand{\dfn}[3]{\frac{\partial^#1 #2}{\partial #3^#1}}
% Векторы
%\renewcommand{\vec}[1]{\boldsymbol{#1}}
%\newcommand{\ort}[1]{\,\boldsymbol{\mathrm{e}}_#1}
% Векторный анализ
\renewcommand{\div}{\,\mathrm{div}\,}
\newcommand{\rot}{\,\mathrm{rot}\,}
\newcommand{\grad}{\,\mathrm{grad}\,}
\newcommand{\laplas}[4]{\dfs{#1}{#2}+\dfs{#1}{#3}+\dfs{#1}{#4}}
\newcommand{\laplasxyz}[1]{\dfs{#1}{x}+\dfs{#1}{y}+\dfs{#1}{z}}
\newcommand{\rotc}[4]{\dff{#1}{#2} - \dff{#3}{#4}}
\newcommand{\rotcx}[3]{\dff{#1\vphantom{E}_#3}{#2} - \dff{#1\vphantom{E}_#2}{#3}}
% Функции
\renewcommand{\cosh}{\,\mathrm{ch}\,}
\renewcommand{\sinh}{\,\mathrm{sh}\,}
\renewcommand{\tanh}{\,\mathrm{th}\,}
\newcommand{\sign}{\mathrm{\,sgn}\,}
\renewcommand{\Im}{\,\mathrm{Im}\,}
\renewcommand{\Re}{\,\mathrm{Re}\,}
\renewcommand{\det}[4]{#1 #4 - #2 #3}
\renewcommand{\matrix}[4]{\begin{pmatrix}#1 & #2 \\ #3 & #4\end{pmatrix}}
\newcommand{\matrixw}[5]{\begin{#5matrix}#1 & #2 \\ #3 & #4\end{#5matrix}} % #5 = b p s v V
\newcommand{\col}[2]{\begin{pmatrix}#1 & #2\end{pmatrix}}
\newcommand{\colw}[3]{\begin{#3matrix}#1 & #2 \end{#3matrix}} % #3 = b p s v V
\newcommand{\row}[2]{\begin{pmatrix}#1 \\ #2\end{pmatrix}}
\newcommand{\roww}[3]{\begin{#3matrix}#1 \\ #2 \end{#3matrix}} % #3 = b p s v V
\newcommand{\matrixt}[2]{\begin{#1matrix} #2\end{#1matrix}}
\newcommand{\matrixrotr}[7]{\begin{#1matrix} \ort{#2} & \ort{#3} & \ort{#4} \\ \df{#2} & \df{#3} & \df{#4} \\ #5 & #6 & #7 \end{#1matrix}}
\newcommand{\matrixrotd}[4]{\begin{#1matrix} \ort{x} & \ort{y} & \ort{z} \\ \df{x} & \df{y} & \df{z} \\ #2 & #3 & #4 \end{#1matrix}}
\newcommand{\matrixrotdv}[2]{\begin{#1matrix} \ort{x} & \ort{y} & \ort{z} \\ \df{x} & \df{y} & \df{z} \\ #2_x & #2_y & #2_z \end{#1matrix}}
\newcommand{\eps}{\varepsilon}
\renewcommand{\phi}{\varphi}

\newcommand{\shtr}{\mathop{\!\vphantom{E}'}}
\newcommand{\ind}[1]{\mathop{\!\vphantom{E}_{#1}}}

\setlength{\parindent}{0cm}
\begin{document}
\tableofcontents
\clearpage

\documentclass[a4paper,10pt]{article} %размер бумаги устанавливаем А4, шрифт 12пунктов
\usepackage[T2A]{fontenc}
\usepackage[utf8]{inputenc}%включаем свою кодировку: koi8-r или utf8 в UNIX, cp1251 в 
\usepackage[english,russian]{babel}
\usepackage{amssymb,amsfonts,amsmath,mathtext,cite,enumerate,float}
\usepackage{hyperref}
\renewcommand{\rmdefault}{ftm}

\usepackage{geometry}
\geometry{left=1cm}
\geometry{right=1cm}
\geometry{top=1cm}
\geometry{bottom=1cm}

\begin{document}
	
	Вспомним что:
	$$
		\sin x = A x \prod\limits_{n = 1}^{\infty} \left(1 - \frac{x^2}{\pi^2 n^2}\right) =
		A x \left[
		1 -
		\frac{x^2}{\pi^2} \left(1 + \frac{1}{2^2} + \frac{1}{3^2} + \ldots \right) +
		\frac{x^4}{\pi^4} \alpha_4 + 
		\frac{x^6}{\pi^6} \alpha_6 + \ldots
		\right]
	$$
	Теперь:
	$$
		\sin x = x - \frac{x^3}{6} + \frac{x^5}{120} - \ldots
	$$
	и финт ушами:
	$$
		\left(1 + \frac{1}{2^2} + \frac{1}{3^2} + \ldots \right)^2 = 
		1 + \frac{1}{2^4} + \frac{1}{3^4} + \ldots + 2 \alpha_4
	$$
	Ба-бах:
	$$
		1 + \frac{1}{2^4} + \frac{1}{3^4} + \ldots = \frac{\pi^4}{36} - 2 \frac{\pi^4}{120} = \frac{\pi^4}{90}
	$$

\end{document}

\section{Сферические координаты}

Связь декартовых $(x, y, z)$ и сферических координат $(r, \theta, \alpha)$ даётся выражениями:
\[
	\begin{aligned}
		& x = r \sin \theta \cos \alpha \\
		& y = r \sin \theta \sin \alpha \\
		& z = r \cos \theta 
	\end{aligned}
\]
Орты сферической системы координат найдём из соотношений:
\[
	\begin{aligned}
	& \left|\frac{\partial \vec{r}}{\partial r}\right| \vec{e}_r = \frac{\partial \vec{r}}{\partial r} =  \sin \theta \cos \alpha\, \vec{e}_x + \sin \theta \sin \alpha\, \vec{e}_y + \cos \theta\, \vec{e}_z \\
	& \left|\frac{\partial \vec{r}}{\partial \theta}\right| \vec{e}_\theta = \frac{\partial \vec{r}}{\partial \theta} =  r \cos \theta \cos \alpha\, \vec{e}_x + r \cos \theta \sin \alpha\, \vec{e}_y - r \sin \theta\, \vec{e}_z \\
	& \left|\frac{\partial \vec{r}}{\partial \alpha}\right| \vec{e}_\alpha = \frac{\partial \vec{r}}{\partial \alpha} = - r \sin \theta \sin \alpha\, \vec{e}_x + r \sin \theta \cos \alpha\, \vec{e}_y
	\end{aligned}
\]
Коэффициенты Ламе:
\[
	\begin{aligned}
		& H_r =  \left|\frac{\partial \vec{r}}{\partial r}\right| = 1 \\
		& H_\theta = \left|\frac{\partial \vec{r}}{\partial \theta}\right| = r \\
		& H_\alpha = \left|\frac{\partial \vec{r}}{\partial \alpha}\right| = r \sin \theta
	\end{aligned}
\]
Орты
\[
	\begin{aligned}
	& \vec{e}_r = \sin \theta \cos \alpha\, \vec{e}_x + \sin \theta \sin \alpha\, \vec{e}_y + \cos \theta\, \vec{e}_z \\
	& \vec{e}_\theta = \cos \theta \cos \alpha\, \vec{e}_x + \cos \theta \sin \alpha\, \vec{e}_y - \sin \theta\, \vec{e}_z \\
	& \vec{e}_\alpha = -\sin \alpha\, \vec{e}_x + \cos \alpha\, \vec{e}_y
	\end{aligned}
\]
Орты декартовой системы, выраженные через орты сферической системы:
\[
	\begin{aligned}
	& \vec{e}_x = \sin \theta \cos \alpha\, \vec{e}_r + \cos \theta \cos \alpha\, \vec{e}_\theta -\sin \alpha\,  \vec{e}_\alpha \\
	& \vec{e}_y = \sin \theta \sin \alpha\, \vec{e}_r + \cos \theta \sin \alpha\, \vec{e}_\theta + \cos \alpha\, \vec{e}_\alpha \\
	& \vec{e}_z = \cos \theta\,\vec{e}_r - \sin \theta\, \vec{e}_\theta
	\end{aligned}
\]
Немного о производных от ортов (для вывода нужно помнить, что орты декартовой системы образуют базис, не зависящий от его положения, в то время как орты сферической системы образуют базис, который меняется от точки к точке):
\[
	\begin{aligned}
	& \frac{\partial \vec{e}_r}{\partial r} = 0 \\
	& \frac{\partial \vec{e}_r}{\partial \theta} = \vec{e}_\theta \\
	& \frac{\partial \vec{e}_r}{\partial \alpha} = \sin \theta\, \vec{e}_\alpha
	\end{aligned}
	\quad
	\begin{aligned}
	& \frac{\partial \vec{e}_\theta}{\partial r} = 0 \\
	& \frac{\partial \vec{e}_\theta}{\partial \theta} = -\vec{e}_r \\
	& \frac{\partial \vec{e}_\theta}{\partial \alpha} = \cos \theta\, \vec{e}_\alpha
	\end{aligned}
	\quad
	\begin{aligned}
	& \frac{\partial \vec{e}_\alpha}{\partial r} = 0 \\
	& \frac{\partial \vec{e}_\alpha}{\partial \theta} = 0 \\
	& \frac{\partial \vec{e}_\alpha}{\partial \alpha} = -\sin\theta \, \vec{e}_r -\cos \theta\, \vec{e}_\theta
	\end{aligned}
\]

\section{Волновые функции в $p$ и $x$ представлениях}

Для описания квантовых процессов вводят волновые функции, которые являются суперпозициями волн д'Бройля. Для одной частицы:
\[
	\psi(\vec{r}, t) = A \int C_(\vec{p}, t) \exp \left(\frac{i}{\hbar} \vec{p} \cdot \vec{r}\right) d^3p
\]
\[
	C(\vec{p}, t) = B \int \psi(\vec{r}, t) \exp \left(-\frac{i}{\hbar} \vec{p} \cdot \vec{r}\right) d^3r
\]
$\psi(\vec{r}, t)$ -- волновая функция в $x$ представлении. $C(\vec{p}, t)$ -- волновая функция в $p$ представлении. $A$ и $B$ -- действительные числа, которые можно найти из условия нормировки. Нормировка:
\[
	\int \psi^*(\vec{r}, t) \psi(\vec{r}, t) d^3r = \int C^*(\vec{p}, t) C(\vec{p}, t) d^3p = 1
\]
\[
	\begin{gathered}
		A^2\iint C^*(\vec{p'}, t) C(\vec{p}, t) \exp \left(\frac{i}{\hbar} (\vec{p'} - \vec{p}) \cdot \vec{r}\right) d^3p\, d^3p'\, d^3r 
		= \\
		=
		A^2\int C^*(\vec{p'}, t) C(\vec{p}, t) (2\pi \hbar)^3 \delta(\vec{p'} - \vec{p}) d^3p\, d^3p'
		= \\
		=
		A^2 (2\pi \hbar)^3 \int C^*(\vec{p}, t) C(\vec{p}, t) d^3p = A^2 (2\pi \hbar)^3 = 1
	\end{gathered}
\]
\[
	\begin{gathered}
		B^2\iint \psi^*(\vec{r'}, t) \psi(\vec{r}, t) \exp \left(\frac{i}{\hbar} \vec{p} \cdot (\vec{r'} - \vec{r})\right) d^3p \, d^3r \, d^3r'
		= \\
		=
		B^2 (2\pi\hbar)^3 \iint \psi^*(\vec{r'}, t) \psi(\vec{r}, t) \delta(\vec{r'} - \vec{r}) d^3r \, d^3r' = B^2 (2\pi\hbar)^3 \iint \psi^*(\vec{r}, t) \psi(\vec{r}, t) d^3r 
		= \\
		=
		B^2 (2\pi\hbar)^3 = 1
	\end{gathered}
\]
\[
	A = B =\frac{1}{(2\pi\hbar)^{3/2}}
\]
Обратимость непосредственно следует из обратимости преобразования Фурье.

\section{Оператор импульса в $x$ представлении}
Оператор импульса в $p$ представлении, просто вектор $\vec{p}$. 
По определению оператор импульса $\hat{\vec{p}}$ в $x$ представлении:
\[
	\langle \vec{p} \,\rangle = \int \psi^*(\vec{r}, t) \hat{\vec{p}} \psi(\vec{r}, t) d^3r
\]
Средний импульс:
\[
	\begin{gathered}
	\langle \vec{p} \,\rangle = 
	\int C^*(\vec{p}, t) \vec{p} C(\vec{p}, t) d^3p = \\ =
	\frac{1}{(2\pi\hbar)^{3}} \iiint \vec{p} \psi^*(\vec{r'}, t) \psi(\vec{r}, t) \exp \left(\frac{i}{\hbar} \vec{p} \cdot (\vec{r'} - \vec{r})\right) d^3r'\, d^3r\, d^3p = \\ =
	\frac{1}{(2\pi\hbar)^{3}} \iint \psi^*(\vec{r'}, t) \psi(\vec{r}, t) \frac{\hbar}{i}\frac{\partial}{\partial \vec{r'}}\int\exp \left(\frac{i}{\hbar} \vec{p} \cdot (\vec{r'} - \vec{r})\right) d^3p\, d^3r'\, d^3r = \\ =
	\iint \psi^*(\vec{r'}, t) \psi(\vec{r}, t) \frac{\hbar}{i} \frac{\partial}{\partial \vec{r'}} \delta(\vec{r'} - \vec{r}) d^3r'\, d^3r = \\ =
	[\text{интегрируем по частям и учитываем, что $\psi = 0$ на $\infty$}] =
	\end{gathered}
\]
\[
	\begin{gathered}
	= - \frac{\hbar}{i} \iint \delta(\vec{r'} - \vec{r}) \frac{\partial}{\partial \vec{r'}} \psi^*(\vec{r'}, t) \psi(\vec{r}, t) d^3r'\, d^3r = \\ =
	- \frac{\hbar}{i} \int \psi(\vec{r}, t) \frac{\partial}{\partial \vec{r}} \psi^*(\vec{r}, t) d^3r = \\ =
	[\text{ещё одно интегрирование по частям}] = \\ =
	\frac{\hbar}{i} \int \psi^*(\vec{r}, t) \frac{\partial}{\partial \vec{r}} \psi(\vec{r}, t) d^3r
	\end{gathered}
\]
\[
	\hat{\vec{p}} = \frac{\hbar}{i} \frac{\partial}{\partial \vec{r}} = - i \hbar \frac{\partial}{\partial \vec{r}}
\]

\section{Оператор координаты в $p$ представлении}
Оператор координаты в $x$ представлении $\vec{r}$. В $p$ представлении по определению:
\[
	\langle \vec{r}\,\rangle = \int C^*(\vec{p}, t) \hat{\vec{r}} C(\vec{p}, t) d^3p	
\]
Итак:
\[
	\begin{gathered}
	\langle \vec{r}\,\rangle = \int \psi^*(\vec{r}, t) \vec{r} \psi(\vec{r}, t) d^3 r = \\ =
	\frac{1}{(2\pi\hbar)^{3}} \iiint C^*(\vec{p'}, t) C(\vec{p}, t) \vec{r} 
	\exp 
	\left(
		-\frac{i}{\hbar}
			(\vec{p'} - \vec{p})\cdot \vec{r}
	\right) d^3r\,d^3p\,d^3p' =
	\\ =
	\frac{1}{(2\pi\hbar)^{3}} \iint C^*(\vec{p'}, t) C(\vec{p}, t)
	\frac{\hbar}{i}
	\frac{\partial}{\partial \vec{p}}
	\int
	\exp 
	\left(
		\frac{i}{\hbar}
		(\vec{p} - \vec{p'})\cdot \vec{r}
	\right) d^3r\,d^3p\,d^3p' = 
	\\ =
	\frac{\hbar}{i}\iint C^*(\vec{p'}, t) C(\vec{p}, t)
	\frac{\partial}{\partial \vec{p}}
	\delta(\vec{p} - \vec{p'}) d^3p\, d^3p'
	= \\ =
	[\text{интегрируем по частям и учитываем, что $C = 0$ на $\infty$}] = \\ =
	-\frac{\hbar}{i}
	\iint 
	C^*(\vec{p}, t)
	\frac{\partial}{\partial \vec{p}}
	C(\vec{p}, t)
	d^3p
	\end{gathered}
\]
В результате:
\[
	\hat{\vec{r}} = 
	-\frac{\hbar}{i} \frac{\partial}{\partial \vec{p}} =
	i\hbar \frac{\partial}{\partial \vec{p}} 
\]
Уравнение Шрёдингера сохраняет свою форму:
\[
	\sum_{i=1}^N\frac{p_i^2}{2m_i} C + \hat{U}\left(i \hbar \frac{\partial}{\partial p_1}, \ldots, i \hbar \frac{\partial}{\partial p_N}\right) C = i\hbar \frac{\partial C}{\partial t}
\]
\section{Определение и некоторые свойства функций Эйри}

Функциями Эйри называют:
\[
	\mathrm{Ai\,}(x) = \frac{1}{\pi} \int\limits_0^\infty \cos \left( \frac{t^3}{3} + xt\right) dt
\]
\[
	\mathrm{Bi\,}(x) = \frac{1}{\pi} \int\limits_0^\infty \left[e^{-\frac{1}{3}t^3 - xt} + \sin \left( \frac{t^3}{3} + xt\right)\right] dt
\]
Асимптотика (можно найти, проинтегрировав методом перевала в комплексной плоскости):
\[
	\mathrm{Ai\,}(x) \underset{x \to \infty}{\approx} \frac{1}{2\sqrt{\pi} x^{1/4}} \exp \left( -\frac{2}{3} x^{3/2}\right)
\]
\[
	\mathrm{Bi\,}(x) \underset{x \to \infty}{\approx} \frac{1}{\sqrt{\pi} x^{1/4}} \exp \left( \frac{2}{3} x^{3/2}\right)
\]
\[
	\mathrm{Ai\,}(x) \underset{x \to -\infty}{\approx} \frac{1}{\sqrt{\pi} x^{1/4}} \sin \left( \frac{2}{3} |x|^{3/2} + \frac{\pi}{4}\right)
\]
\[
	\mathrm{Bi\,}(x) \underset{x \to -\infty}{\approx} \frac{1}{\sqrt{\pi} x^{1/4}} \cos \left( \frac{2}{3} |x|^{3/2}+ \frac{\pi}{4}\right)
\]
Интеграл (легко находится с использованием дельта функции)
\[
	\int\limits_{-\infty}^{\infty} \mathrm{Ai\,}(x)\, dx = 1
\]
Функции Эйри удовлетворяют уравнению Эйри:
\[
	u'' - xu = 0
\]
Рассмотрим интеграл:
\[
	\int\limits_{a}^{\infty}\mathrm{\,Ai\,}'(x) \mathrm{\,Ai\,}''(x) dx= 
	\int\limits_{a}^{\infty} x \mathrm{\,Ai\,}'(x) \mathrm{\,Ai\,}(x) dx
\]
\[
	\Rightarrow \frac{\left[\mathrm{\,Ai\,}'(x) \right]^2}{2} \Bigg|_{a}^{\infty} =
	x \frac{\left[\mathrm{\,Ai\,}(x) \right]^2}{2} \Bigg|_{a}^{\infty} - 
	\int\limits_{a}^{\infty} \frac{\left[\mathrm{\,Ai\,}(x) \right]^2}{2} dx
\]
Откуда следует весьма важный интеграл:
\[
	\int\limits_{a}^{\infty} \left[\mathrm{\,Ai\,}(x) \right]^2 dx =
	\left[\mathrm{\,Ai\,}'(a) \right]^2  - a \left[\mathrm{\,Ai\,}(a) \right]^2
\]

\section{Решение линейных уравнений}

Поставим следующую задачу:
\[
	L\left[y\right] = \frac{\partial y}{\partial t} 
\]
\[
	y(x, t)\Big|_{t = 0} = y(x, 0) 
\]
$L$ -- произвольный линейный оператор.
Пусть собственные значения $Y(x, \lambda)$ удовлетворяет уравнению:
\[
	L\left[Y\right] = \lambda Y
\]
и если спектр $\lambda$ сплошной, существует функция $Y^{-1}(x, \lambda')$, такая что:
\[
	\int\limits_{-\infty}^{\infty} Y^{-1}(x, \lambda') Y(x, \lambda) dx = \delta(\lambda - \lambda')
\]
Будем искать $y$ в виде:
\[
	y = \int\limits_{-\infty}^{\infty} A(\lambda) e^{\lambda t} Y(x, \lambda)\, d\lambda
\]
Простой подстановкой можно проверить, что это выражение в самом деле является решением уравнения. $A(\lambda)$ определим из начальных условий:
\[
	y(x, 0) = \int\limits_{-\infty}^{\infty} A(\lambda) Y(x, \lambda)\, d\lambda
\]
\[
	A(\lambda) = \int\limits_{-\infty}^{\infty} y(x, 0) Y^{-1}(x, \lambda)\, dx
\]
Подставляем в решение и получаем:
\[
	y(x, t) = \int\limits_{-\infty}^{\infty} y(x', 0) \int\limits_{-\infty}^{\infty} e^{\lambda t}Y^{-1}(x', \lambda) Y(x, \lambda)\, d\lambda\, dx'
\]

\section{Решение линейных уравнений в случае дискретного спектра}

В случае дискретного спектра у нас есть система функций:
\[
	Y_1(x), Y_2(x), \ldots, Y_n(x), \ldots
\]
и соответствующая ей последовательность чисел:
\[
	\lambda_1, \lambda_2, \ldots
\]
Если удалось построить дополнительную систему функций
\[
	Y_1^{-1}(x), Y_2^{-1}(x), \ldots, Y_n^{-1}(x), \ldots
\]
такую что
\[
	\int\limits_{-\infty}^{\infty} Y_k^{-1}(x) Y_n(x) dx = \delta_{kn}
\]
То решение уравнения можно найти в виде:
\[
	y = \sum_{k=1}^{\infty} A_k e^{\lambda_k t} Y_k(x)
\]
\[
	A_k = \int\limits_{-\infty}^{\infty} Y_k^{-1}(x) y(x, 0) dx
\]
Получаем окончательно решение:
\[
	y(x, t) = \int\limits_{-\infty}^{\infty} y(x', 0) \sum_{k=1}^{\infty} e^{\lambda_k t} Y_k^{-1}(x') Y_k(x) dx'
\]
\section{Частица в однородном силовом поле}

Пусть сила, действующая на частицу, равна $F$ по модулю и направлена вдоль оси $z$ в обратном направлении. Гамильтониан в классическом случае:
\[
	\hat{H} = \frac{\hat{p}_z^2}{2m} + F \hat{z} 
\]
Уравнение Шрёдингера:
\[
	\frac{p_z^2}{2m} C(p_z, t) + i \hbar F \frac{\partial C(p_z, t)}{\partial p_z} = i\hbar \frac{\partial C(p_z, t)}{\partial t}
\]
Вместо $\lambda$ будет выступать $-iE/\hbar$. Для $Y$ получаем уравнение
\[
	\frac{p_z^2}{2m} Y + i \hbar F \frac{\partial Y}{\partial p_z} = E Y
\]
Будем искать $Y$ в форме $\exp f(p_z)$:
\[
	i \hbar F \frac{\partial f(p_z)}{\partial p_z} = E - \frac{p_z^2}{2m}
\]
\[
	f(p_z) = - i \frac{E}{\hbar F} p_z + i \frac{1}{6 m \hbar F} p_z^3 + const
\]
Константа уйдёт в нормировочный множитель. 
\[
	Y(p_z, E) = A \exp \left(-\frac{i}{\hbar} \left( \frac{E}{F} p_z - \frac{1}{6 m F} p_z^3 \right) \right)
\]
\[
	Y^{-1}(p_z, E) = B \exp \left(\frac{i}{\hbar} \left( \frac{E}{F} p_z - \frac{1}{6 m F} p_z^3 \right) \right)
\]
\[
	AB \int\limits_{-\infty}^{\infty} \exp \left(\frac{i p_z}{\hbar F} (E' - E) \right) dp_z = 
	2\pi AB\hbar F\, \delta(E' - E) = \delta(E' - E)
\]
Откуда можно в качестве $A$ и $B$ можно взять константы:
\[
	A = B = \frac{1}{\sqrt{2\pi \hbar F}}
\]
\[
	\begin{gathered}
		C(p_z, t) = \int\limits_{-\infty}^{\infty} C(p_z, 0) \int\limits_{-\infty}^{\infty} AB
		\exp \left(-\frac{i}{\hbar} \left( E \left(\frac{p_z - p_z'}{F} + t\right) - \frac{1}{6 m F} (p_z^3 - p_z'^3) \right)
		\right)\, dE\, dp_z' = \\
		=
		\int\limits_{-\infty}^{\infty} C(p_z', 0) \delta(p_z - p_z' + F t) 
		\exp \left(\frac{i}{6 m \hbar F} (p_z^3 - p_z'^3) \right)
		\, dp_z' = \\
		=
		C(p_z + F t, 0) \exp \left(\frac{i(-3 p_z^2 F t - 3 p_z F^2 t^2 - F^3 t^3)}{6 m \hbar F} \right) = \\
		=
		C(p_z + F t, 0) \exp \left(-\frac{i}{\hbar} \left(\frac{p_z^2}{2m} t + \frac{p_z F t^2}{2 m} + \frac{F^2 t^3}{6 m}\right) \right)
	\end{gathered}
\]
Чтобы получить дискретный спектр нужно ввести дополнительные условия. Чаще всего таким условием выступает поверхность, за пределы которой частица не может попасть. В качестве такой поверхности может выступать поверхность Земли для поля тяжести, обкладка конденсатора.

\section{Гауссовский волновой пакет в однородном поле}

Пусть
\[
	\psi(z, 0) = \frac{1}{\sqrt[4]{2\pi \sigma^2}} \exp \left( -\frac{(z - z_0)^2}{4\sigma^2}\right)
\]
\[
	\begin{gathered}
		C(p_z, 0) = \frac{1}{\sqrt[4]{2\pi \sigma^2}}\frac{1}{(2\pi\hbar)^{1/2}} \int\limits_{-\infty}^{\infty} \exp \left(-\frac{(z - z_0)^2}{4\sigma^2} - \frac{i}{\hbar} p_z z  \right) dz = \\
		=
		\frac{1}{\sqrt[4]{2\pi \sigma^2}}\frac{1}{(2\pi\hbar)^{1/2}} \exp\left(-\frac{i}{\hbar}p_z z_0\right)\int\limits_{-\infty}^{\infty} \exp \left(-\frac{(z - z_0)^2}{4\sigma^2} - \frac{i}{\hbar} p_z (z - z_0)  \right) dz = \\
		=
		\frac{1}{\sqrt[4]{2\pi \sigma^2}}\frac{1}{(2\pi\hbar)^{1/2}} \exp\left(-\frac{i}{\hbar}p_z z_0\right)
		\int\limits_{-\infty}^{\infty} \exp \left(-\frac{(z - z_0)^2}{4\sigma^2} - 2 \frac{i}{\hbar} \sigma p_z \frac{(z - z_0)}{2 \sigma}  \right) dz = \\
		=
		\frac{1}{\sqrt[4]{2\pi \sigma^2}}\frac{1}{(2\pi\hbar)^{1/2}} \exp\left(-\frac{i}{\hbar}p_z z_0\right)
		\int\limits_{-\infty}^{\infty} \exp \left(-\left(\frac{(z - z_0)}{2 \sigma} + \frac{i}{\hbar} \sigma p_z \right)^2 - \frac{\sigma^2 p_z^2}{\hbar^2}  \right) dz = \\
		=
		\frac{2\sqrt[4]{2\pi \sigma^2}}{(2\pi\hbar)^{1/2}} \exp\left(-\frac{i}{\hbar}p_z z_0 - \frac{\sigma^2 p_z^2}{\hbar^2} \right) = \\
		=
		\sqrt[4]{\frac{2\sigma^2}{\pi \hbar^2}} \exp\left(-\frac{i}{\hbar}p_z z_0 - \frac{\sigma^2 p_z^2}{\hbar^2} \right)
	\end{gathered}
\]
\[
	\begin{gathered}
		\langle z \rangle =
		\sqrt{\frac{2\sigma^2}{\pi \hbar^2}} 
		\int\limits_{-\infty}^{\infty} \left(z_0 - \frac{2i\hbar \sigma^2}{\hbar^2} (p_z + Ft) + \frac{p_z}{m} t +  \frac{F t^2}{2m}\right) 
		\exp\left( - \frac{2\sigma^2 (p_z + Ft)^2}{\hbar^2} \right) dp_z = \\
		=
		\sqrt{\frac{2\sigma^2}{\pi \hbar^2}} 
		\int\limits_{-\infty}^{\infty} \left(z_0 - \frac{2i\hbar \sigma^2}{\hbar^2} (p_z + Ft) + \frac{p_z + F t}{m} t -  \frac{F t^2}{2m}\right) 
		\exp\left( - \frac{2\sigma^2 (p_z + Ft)^2}{\hbar^2} \right) dp_z = \\
		=
		z_0 - \frac{F t^2}{2m}	
	\end{gathered}
\]
\[
	\begin{gathered}
		\langle z^2 \rangle =
		\sqrt{\frac{2\sigma^2}{\pi \hbar^2}} 
		\int\limits_{-\infty}^{\infty} \left(z_0 - \frac{F t^2}{2m} + \left(\frac{t}{m}- 2\frac{i \sigma^2}{\hbar}\right) (p_z + Ft)\right)^2
		\exp\left( - \frac{2\sigma^2 (p_z + Ft)^2}{\hbar^2} \right) dp_z + \\
		+
		\sqrt{\frac{2\sigma^2}{\pi \hbar^2}} 
		\int\limits_{-\infty}^{\infty} \left(2\sigma^2 + \frac{i\hbar}{m} t \right)
		\exp\left( - \frac{2\sigma^2 (p_z + Ft)^2}{\hbar^2} \right) dp_z = \\
		=
		\left(z_0 - \frac{F t^2}{2m}\right)^2 +
		\left(\frac{t}{m}- 
		2\frac{i \sigma^2}{\hbar}\right)^2 \frac{\hbar^2}{4 \sigma^2} +
		\left(2\sigma^2 + \frac{i\hbar}{m} t \right) = \\
		=
		\left(z_0 - \frac{F t^2}{2m}\right)^2 +
		\frac{t^2\hbar^2}{4 m^2 \sigma^2} + \sigma^2
	\end{gathered}
\]
Дисперсия:
\[
	\langle z^2 \rangle - \langle z \rangle^2 =
	\frac{t^2\hbar^2}{4 m^2 \sigma^2} + \sigma^2
\]

\section{Частица в однородном поле над непроницаемым барьером}

Рассмотрим частицу в треугольной потенциальной яме:

\parbox{\textwidth}{
	\centering
\begin{tikzpicture}[>= stealth]
\draw[->] (0, -0.2) -- (0, 3) node[left]{$U$};
\draw[->] (-0.2, 0) -- (3, 0) node[below]{$z$};
\draw[line width = 2pt] (0, 0) node[above left]{$0$} -- (0, 2.5);
\draw[line width = 2pt] (0, 0) -- (2.5, 2.5) node[below right]{$F z$}; 
\end{tikzpicture}
}
В этом случае спектр дискретный. Нам необходимо от импульсной формы перейти к координатной. В этом случае получится обычное стационарное уравнение Шрёдингера c решением:
\[
	\begin{gathered}
	\psi(z, E) = \int\limits_{-\infty}^{\infty} 
	A \exp \left(-\frac{i}{\hbar} \left( \frac{E - Fz}{F} p_z - \frac{1}{6 m F} p_z^3 \right) \right) dp_z = \\ =
	\int\limits_{-\infty}^{\infty} 
	A \cos \left( -\frac{E - Fz}{F \hbar} p_z + \frac{1}{2 m F \hbar} \frac{p_z^3}{3} \right) dp_z
	= \\ =
	A (2 m F \hbar)^{1/3} \mathrm{\,Ai\,} \left((2 m F \hbar)^{1/3}\frac{Fz - E}{F \hbar}\right)	
	\end{gathered}
\]
Зная корни функции Эйри $x_1, x_2, \ldots$, легко найти спектр:
\[
	E_k = - (2 m)^{-1/3} \hbar^{2/3} F^{2/3} x_k
\]
А воспользовавшись интегралом:
\[
	\int\limits_{x_k}^{\infty} \mathrm{\,Ai\,}^2 (x)\,dx = \left[\mathrm{\,Ai\,}' (x_k)\right]^2,
\]
нормировочную константу:
\[
	A = \hbar^{-2/3} (2 m F)^{-1/6} |\mathrm{\,Ai\,}' (x_k)|^{-1}
\]
Окончательно волновые функции:
\[
	\psi_k(z) = \frac{(2mF)^{1/6}}{\hbar^{1/3}|\mathrm{\,Ai\,}' (x_k)|} \mathrm{\,Ai\,} \left((2 m F \hbar)^{1/3} \frac{z}{\hbar} + x_k \right)
\]
Для электронов:
\[
	E_k \approx -1{,}14\times10^{6} F^{2/3} x_k \text{ эВ}
\]
Найдём количество значений $x_k$ приходящихся на интервал $(k_{b}T, k_{b}(T + \Delta T))$ при $\Delta T= 10 $К и $T = 300 $К (обозначим эту величину $N$) для поля силы тяжести, что соответствует частице над поверхностью Земли. Для этого воспользуемся приближённым выражением:
\[
	\mathrm{Ai\,}(x) \underset{x \to -\infty}{\approx} \frac{1}{\sqrt{\pi} x^{1/4}} \sin \left( \frac{2}{3} |x|^{3/2} + \frac{\pi}{4}\right)
\]
\[
	|x_k|^{3/2} \approx \frac{3}{2}\pi k - \frac{3\pi}{8}
\]
\[
	N \approx \frac{2^{3/2}}{3\pi g \hbar m^{1/2}} k_{b}^{3/2} T^{3/2} ((1 + \Delta T/T)^{3/2} - 1) \approx 
	\frac{2^{1/2}}{\pi g \hbar m^{1/2}} k_{b}^{3/2} T^{1/2} \Delta T
\]
Для электронов:
\[
	N \approx 4{,}07\times10^{15}
\]
Далее можно приближённо вычислить среднюю температуру таких частиц и так далее.
\section{Оператор импульса в сферических координатах и коммутационные соотношения}

Нам известно, что оператор импульса это градиент с коэффициентом. Градиент в сферических координатах:
\[
	\nabla = \left\{\frac{\partial}{\partial r}, \frac{1}{r}\frac{\partial}{\partial \theta}, \frac{1}{r \sin \theta}\frac{\partial}{\partial \alpha} \right\}
\]
Оператор Лапласа, который фигурирует в уравнении Шрёдингера:
\[
	\Delta = \nabla \cdot \nabla = \frac{\partial^2}{\partial r^2} + \frac{1}{r^2}\frac{\partial^2}{\partial \theta^2} +\frac{1}{r^2 \sin^2 \theta}\frac{\partial^2}{\partial \alpha^2}
\]
Но умные люди давно подсчитали, что:
\[
	\Delta = 
	\frac{1}{r^2}\frac{\partial}{\partial r} \left(r^2 \frac{\partial}{\partial r}\right)
	+ \frac{1}{r^2 \sin \theta} \frac{\partial}{\partial \theta} \left(\sin \theta \frac{\partial}{\partial \theta}\right)
	+ \frac{1}{r^2 \sin^2 \theta} \frac{\partial^2}{\partial \alpha^2}
\]
и результаты, вообще говоря, не совпадают. Где мы ошиблись? Ответ на этот вопрос чрезвычайно прост: мы не должны были забывать об ортах. Запишем градиент иначе:
\[
	\nabla = \vec{e}_r \frac{\partial}{\partial r} + \vec{e}_\theta \frac{1}{r}\frac{\partial}{\partial \theta} + \vec{e}_\alpha \frac{1}{r \sin \theta}\frac{\partial}{\partial \alpha}
\]
И воспользуемся формулами:
\[
\begin{aligned}
& \frac{\partial \vec{e}_r}{\partial r} = 0 \\
& \frac{\partial \vec{e}_r}{\partial \theta} = \vec{e}_\theta \\
& \frac{\partial \vec{e}_r}{\partial \alpha} = \sin \theta\, \vec{e}_\alpha
\end{aligned}
\quad
\begin{aligned}
& \frac{\partial \vec{e}_\theta}{\partial r} = 0 \\
& \frac{\partial \vec{e}_\theta}{\partial \theta} = -\vec{e}_r \\
& \frac{\partial \vec{e}_\theta}{\partial \alpha} = \cos \theta\, \vec{e}_\alpha
\end{aligned}
\quad
\begin{aligned}
& \frac{\partial \vec{e}_\alpha}{\partial r} = 0 \\
& \frac{\partial \vec{e}_\alpha}{\partial \theta} = 0 \\
& \frac{\partial \vec{e}_\alpha}{\partial \alpha} = -\sin\theta \, \vec{e}_r -\cos \theta\, \vec{e}_\theta
\end{aligned}
\]
Тогда
\[
	\begin{gathered}
		\Delta = 
		\vec{e}_r\cdot \frac{\partial}{\partial r} 
		\left(\vec{e}_r \frac{\partial}{\partial r} + \vec{e}_\theta \frac{1}{r}\frac{\partial}{\partial \theta} + \vec{e}_\alpha \frac{1}{r \sin \theta}\frac{\partial}{\partial \alpha}\right)
		+ \\ +
		\vec{e}_\theta\cdot \frac{1}{r}\frac{\partial}{\partial \theta}
		\left(\vec{e}_r \frac{\partial}{\partial r} + \vec{e}_\theta \frac{1}{r}\frac{\partial}{\partial \theta} + \vec{e}_\alpha \frac{1}{r \sin \theta}\frac{\partial}{\partial \alpha}\right)
		+
		\vec{e}_\alpha\cdot \frac{1}{r \sin \theta}\frac{\partial}{\partial \alpha}
		\left(\vec{e}_r \frac{\partial}{\partial r} + \vec{e}_\theta \frac{1}{r}\frac{\partial}{\partial \theta} + \vec{e}_\alpha \frac{1}{r \sin \theta}\frac{\partial}{\partial \alpha}\right) 
		= \\ =
		\frac{\partial^2}{\partial r^2} +
		\frac{1}{r} \frac{\partial}{\partial r} +
		\frac{1}{r^2} \frac{\partial^2}{\partial \theta^2} +
		\frac{1}{r} \frac{\partial}{\partial r} +
		\frac{1}{r^2} \frac{\cos \theta}{\sin \theta} \frac{\partial}{\partial \theta} +
		\frac{1}{r^2 \sin^2 \theta} \frac{\partial^2}{\partial \alpha^2}
		= \\ =
		\frac{1}{r^2}\frac{\partial}{\partial r} \left(r^2 \frac{\partial}{\partial r}\right)
		+ \frac{1}{r^2 \sin \theta} \frac{\partial}{\partial \theta} \left(\sin \theta \frac{\partial}{\partial \theta}\right)
		+ \frac{1}{r^2 \sin^2 \theta} \frac{\partial^2}{\partial \alpha^2}
	\end{gathered}
\]
Итак, оператор импульса в сферических координатах:
\[
	\hat{\vec{p}} = -i\hbar \left(\vec{e}_r \frac{\partial}{\partial r} + \vec{e}_\theta \frac{1}{r}\frac{\partial}{\partial \theta} + \vec{e}_\alpha \frac{1}{r \sin \theta}\frac{\partial}{\partial \alpha} \right)
\]
Соотношения коммутации в случае декартовых координат:
\[
	[\hat{p}_i, \hat{p}_j] = \hat{p}_i \hat{p}_j - \hat{p}_j \hat{p}_i = 0
\]
С точки зрения векторов это означает, что в терминах полного произведения векторов матрица $\hat{\vec{p}}\hat{\vec{p}}$ симметрична.
\[
	\begin{gathered}
	\nabla\nabla = 
	\vec{e}_r \frac{\partial}{\partial r} 
	\left(\vec{e}_r \frac{\partial}{\partial r} + \vec{e}_\theta \frac{1}{r}\frac{\partial}{\partial \theta} + \vec{e}_\alpha \frac{1}{r \sin \theta}\frac{\partial}{\partial \alpha}\right)
	+ \\ +
	\vec{e}_\theta \frac{1}{r}\frac{\partial}{\partial \theta}
	\left(\vec{e}_r \frac{\partial}{\partial r} + \vec{e}_\theta \frac{1}{r}\frac{\partial}{\partial \theta} + \vec{e}_\alpha \frac{1}{r \sin \theta}\frac{\partial}{\partial \alpha}\right)
	+
	\vec{e}_\alpha \frac{1}{r \sin \theta}\frac{\partial}{\partial \alpha}
	\left(\vec{e}_r \frac{\partial}{\partial r} + \vec{e}_\theta \frac{1}{r}\frac{\partial}{\partial \theta} + \vec{e}_\alpha \frac{1}{r \sin \theta}\frac{\partial}{\partial \alpha}\right) 
	= \\ =
	\vec{e}_r \vec{e}_r \frac{\partial^2}{\partial r^2} +
	\vec{e}_r \vec{e}_\theta \frac{\partial}{\partial r} \frac{1}{r}\frac{\partial}{\partial \theta} +
	\vec{e}_r \vec{e}_\alpha \frac{\partial}{\partial r} \frac{1}{r \sin \theta}\frac{\partial}{\partial \alpha} +
	\vec{e}_\theta \vec{e}_r \left(\frac{1}{r} \frac{\partial^2}{\partial \theta\, \partial r} - \frac{1}{r^2} \frac{\partial}{\partial \theta} \right) + \\ +
	\vec{e}_\theta \vec{e}_\theta \left(\frac{1}{r} \frac{\partial}{\partial r} +
	\frac{1}{r^2} \frac{\partial^2}{\partial \theta^2} \right) +
	\vec{e}_\theta \vec{e}_\alpha \left(\frac{1}{r}\frac{\partial}{\partial \theta} \frac{1}{r \sin \theta}\frac{\partial}{\partial \alpha}\right) +
	\vec{e}_\alpha \vec{e}_r \left(\frac{1}{r \sin \theta} \frac{\partial}{\partial \alpha} \frac{\partial}{\partial r} - \frac{1}{r^2 \sin \theta} \frac{\partial}{\partial \alpha} \right) + \\ +
	\vec{e}_\alpha \vec{e}_\theta \left(\frac{1}{r \sin \theta}\frac{\partial}{\partial \alpha}\frac{1}{r}\frac{\partial}{\partial \theta} - \frac{\cos \theta}{r \sin \theta} \frac{1}{r \sin \theta}\frac{\partial}{\partial \alpha}\right) +
	\vec{e}_\alpha \vec{e}_\alpha \left(\frac{1}{r} \frac{\partial}{\partial r} + \frac{\cos \theta}{r^2 \sin \theta}\frac{\partial}{\partial \theta} + \frac{1}{r^2 \sin^2 \theta} \frac{\partial^2}{\partial \alpha^2} \right)
	\end{gathered}
\]
Выпишем матрицу $\nabla\nabla$:
\[
	\begin{pmatrix}
		\cfrac{\partial^2}{\partial r^2} && 
		\cfrac{\partial}{\partial r} \cfrac{1}{r}\cfrac{\partial}{\partial \theta} &&
		\cfrac{\partial}{\partial r} \cfrac{1}{r \sin \theta}\cfrac{\partial}{\partial \alpha} \\\\
		\cfrac{1}{r} \cfrac{\partial^2}{\partial \theta\, \partial r} - \cfrac{1}{r^2} \cfrac{\partial}{\partial \theta} &&
		\cfrac{1}{r} \cfrac{\partial}{\partial r} + \cfrac{1}{r^2} \cfrac{\partial^2}{\partial \theta^2} &&
		\cfrac{1}{r}\cfrac{\partial}{\partial \theta} \cfrac{1}{r \sin \theta}\cfrac{\partial}{\partial \alpha} \\\\
		\cfrac{1}{r \sin \theta} \cfrac{\partial}{\partial \alpha} \cfrac{\partial}{\partial r} - \cfrac{1}{r^2 \sin \theta} \cfrac{\partial}{\partial \alpha} &&
		\cfrac{1}{r \sin \theta}\cfrac{\partial}{\partial \alpha}\cfrac{1}{r}\cfrac{\partial}{\partial \theta} - \cfrac{\cos \theta}{r \sin \theta} \cfrac{1}{r \sin \theta}\cfrac{\partial}{\partial \alpha} &&
		\cfrac{1}{r} \cfrac{\partial}{\partial r} + \cfrac{\cos \theta}{r^2 \sin \theta}\cfrac{\partial}{\partial \theta} + \cfrac{1}{r^2 \sin^2 \theta} \cfrac{\partial^2}{\partial \alpha^2}
	\end{pmatrix}
\]
Она симметрична. Интересно, что данный результат нельзя получить, если рассматривать покомпонентные коммутационные соотношения.
\section{Гипергеометрические функции}

\subsection{Гипергеометрическая функция (функция Гаусса)}

Сумма бесконечной геометрической прогрессии:
\[
	S = b_0(1 + x + x^2 + \ldots)
\]
Её обобщением является следующая трёхпараметрическая функция:
\[
	F(\alpha, \beta; \gamma; x) = 
	1+
	\frac{\alpha \beta}{\gamma} \frac{x}{1!} + 
	\frac{\alpha(\alpha + 1) \beta (\beta + 1)}{\gamma (\gamma + 1)} \frac{x^2}{2!} +
	\frac{\alpha(\alpha + 1)(\alpha + 2) \beta (\beta + 1)(\beta + 2)}{\gamma (\gamma + 1)(\gamma + 2)} \frac{x^3}{3!} +
	\ldots
\]
Ряд сходится только при $|x| < 1$.
Некоторые очевидные соотношения:
\[
	F'(\alpha, \beta; \gamma; x) = 
	\frac{\partial F(\alpha, \beta; \gamma; x)}{\partial x} = 
	\frac{\alpha \beta}{\gamma} F(\alpha+1, \beta + 1; \gamma + 1; x)
\]

Она является частным решением дифференциального уравнения Эйлера:
\[
	x(x - 1) y'' + [(\alpha + \beta + 1)x - \gamma] y' + \alpha \beta y = 0
\]
Общее решение уравнения Эйлера:
\[
	y = C_1 F(\alpha, \beta; \gamma; x) + C_2 x^{1 - \gamma} F(\alpha - \gamma + 1, \beta - \gamma + 1; 2 - \gamma; x)
\]
Из определения гипергеометрической функции следует:
\[
	\gamma \ne 0, -1, -2, \ldots
\]
Интегральное представление:
\[
	F(\alpha, \beta; \gamma; x) =
	\frac{\Gamma(\gamma)}{\Gamma(\beta) \Gamma(\gamma - \beta)}
	\int\limits_0^1 t^{\beta-1}(1 - t)^{\gamma - \beta - 1} (1 - tx)^{-\alpha} dt
\]
Как следствие:
\[
	F(\alpha, \beta; \gamma; 1) = \frac{\Gamma(\gamma)\Gamma(\gamma - \alpha - \beta)}{\Gamma(\gamma - \alpha)\Gamma(\gamma - \beta)}
\]
Другая форма дифференциального уравнения Эйлера:
\[
	\frac{d}{dx}
	\left[x^\gamma(x - 1)^{\alpha + \beta + 1 - \gamma} y'\right] + \alpha \beta x^{\gamma - 1}(x - 1)^{\alpha + \beta - \gamma} y = 0
\]

\subsection{Преобразование уравнения Эйлера}

\[
	x(x - 1) y''_{xx} + [(\alpha + \beta + 1)x - \gamma] y'_{x} + \alpha \beta y = 0
\]
Выполняем замену:
\[
	x = f(t) \quad y = g(t) z(t)
\]
\[
	\begin{aligned}
	& y''_{xx} = \frac{1}{f'} \frac{d}{dt} \left(\frac{y'}{f'}\right) = 
	\frac{1}{f'^2} y'' - \frac{f''}{f'^3} y'
	= \frac{1}{f'^2} (g z'' + 2 g' z' + g'' z) - \frac{f''}{f'^3} (gz' + g'z)
	\\
	& y'_x = \frac{1}{f'} (gz' + g'z)
	\end{aligned}
\]
\[
	\begin{aligned}
	\frac{f(f - 1)}{f'^2} g z'' + &
	\left(
		\frac{2g'f(f - 1)}{f'^2} + \frac{(\alpha + \beta + 1)f - \gamma}{f'} g - \frac{f''f(f - 1)}{f'^3} g
	\right) z' + \\ & +
	\left(
		\frac{f(f - 1)}{f'^2} g'' + \frac{(\alpha + \beta + 1)f - \gamma}{f'} g' - \frac{f''f(f - 1)}{f'^3} g' + \alpha \beta
	\right) z = 0
	\end{aligned}
\]
\[
	z'' + 
	\left(
		\frac{2g'}{g} + \frac{(\alpha + \beta + 1)f - \gamma}{f(f - 1)} f' - \frac{f''}{f'}
	\right) z' +
	\left(
		\frac{g''}{g}  + \frac{(\alpha + \beta + 1)f - \gamma}{f(f - 1)} \frac{g'}{g} - \frac{f''}{f'} \frac{g'}{g} + \alpha \beta
	\right) z = 0
\]
Накладываем дополнительное условие:
\[
	\frac{2g'}{g} + \frac{(\alpha + \beta + 1)f - \gamma}{f(f - 1)} f' - \frac{f''}{f'} = 0
\]
\[
	\frac{2g'}{g} + \frac{\alpha + \beta + 1}{f - 1} f' - \gamma \left(\frac{1}{f - 1} - \frac{1}{f}\right) f'- \frac{f''}{f'} = 0
\]
\[
	\frac{2g'}{g} + \frac{\alpha + \beta + 1 - \gamma}{f - 1} f' + \frac{\gamma}{f} f'- \frac{f''}{f'} = 0
\]
\[
	\ln g^2 + \ln (f - 1)^{\alpha + \beta + 1 - \gamma} + \ln f^\gamma - \ln f' = const = \ln C
\]
\[
	g^2 (f - 1)^{\alpha + \beta + 1 - \gamma} f^\gamma = C f'
\]
\[
	\int g^2 dt = C \int (f - 1)^{\gamma - \alpha - \beta - 1} f^{-\gamma} df \text{ даёт связь между $f(t)$ и $g(t)$}
\]
Учёт условия даёт:
\[
	z'' + 
	\left(
		\frac{g''}{g}  - 2 \frac{g'^2}{g^2} + \alpha \beta
	\right) z = 0	
\]
Если ввести функцию:
\[
	h(t) = \frac{1}{g}
\]
\[
	\frac{z''}{z} - 
	\frac{h''}{h} + \alpha \beta
	 = 0	
\]

\subsection{Вырожденная гипергеометрическая функция (функция Куммера)}

Уравнение Куммера:
\[
	x y'' + (\gamma - x) y' - \alpha y = 0
\]
Его можно получить заменой в уравнении Эйлера $x$ на $x/\beta$ и устремив $\beta \to \infty$. Функция Куммера:
\[
	F(\alpha; \gamma; x) = 1+
	\frac{\alpha}{\gamma} \frac{x}{1!} + 
	\frac{\alpha(\alpha + 1)}{\gamma (\gamma + 1)} \frac{x^2}{2!} +
	\frac{\alpha(\alpha + 1)(\alpha + 2)}{\gamma (\gamma + 1)(\gamma + 2)} \frac{x^3}{3!} +
	\ldots
\]
Ряд сходится при любых значениях $x$. Общее решение уравнения Куммера:
\[
	y = C_1 F(\alpha; \gamma; x) + C_2 x^{1 - \gamma} F(\alpha - \gamma + 1; 2 - \gamma; x)
\]
Интегральное представление:
\[
	F(\alpha; \gamma; x) = 
	\frac{\Gamma(\gamma)}{\Gamma(\alpha) \Gamma(\gamma - \alpha)}
	\int\limits_0^1 t^{\alpha-1}(1 - t)^{\gamma - \alpha - 1} e^{tx} dt
\]
Отсюда можно получить:
\[
	F(\alpha; \gamma; x) = 
	e^x F(\gamma - \alpha; \gamma; -x) 
\]
Очевидное соотношение:
\[
	\frac{d F(\alpha; \gamma; x)}{dx} = \frac{\alpha}{\gamma} F(\alpha + 1; \gamma + 1; x)
\]

\subsection{Преобразование уравнения Куммера}

\[
	x y''_{xx} + (\gamma - x) y'_x - \alpha y = 0
\]
Выполняем замену:
\[
	x = f(t) \quad y = g(t) z(t)
\]
\[
	\begin{aligned}
	& y''_{xx} = \frac{1}{f'^2} (g z'' + 2 g' z' + g'' z) - \frac{f''}{f'^3} (gz' + g'z)
	\\
	& y'_x = \frac{1}{f'} (gz' + g'z)
	\end{aligned}
\]
\[
	\frac{f g}{f'^2} z'' +
	\left(
		\frac{2 g' f}{f'^2} - \frac{f'' f g}{f'^3}  +\frac{\gamma - f}{f'} g 
	\right) z' +
	\left(
	\frac{g'' f}{f'^2} - \frac{f'' f g'}{f'^3}  +\frac{\gamma - f}{f'} g' - \alpha
	\right) z = 0
\]
\[
	z'' +
	\left(
	\frac{2 g'}{g} - \frac{f''}{f'}  +\frac{\gamma - f}{f} f' 
	\right) z' +
	\left(
	\frac{g''}{g} - \frac{f''}{f'} \frac{g'}{g} +\frac{\gamma - f}{f'} \frac{g'}{g}  - \alpha
	\right) z = 0
\]
Добавляем одно условие:
\[
	\frac{2 g'}{g} - \frac{f''}{f'}  +\frac{\gamma - f}{f} f' = 0
\]
\[
	\ln g^2 - \ln f'  + \ln f^\gamma - f = \ln C
\]
\[
	g^2 = C f^{-\gamma} e^{f} f'
\]
\[
	\int g^2 dt = C \int f^{-\gamma} e^{f} df
\]
Вводим функцию:
\[
	h(t) = \frac{1}{g(t)}
\]
Получаем уравнение:
\[
	\frac{z''}{z} - 
	\frac{h''}{h} - \alpha
	= 0	
\]
\section{Частица в скалярном поле Юкавы}

Скалярное поле Юкавы:
\[
U = U_0 \frac{e^{-\gamma r}}{r}
\]
Стационарное уравнение Шрёдингера:
\[
- \frac{\hbar^2}{2M} \frac{1}{r^2} \frac{\partial}{\partial r} r^2 \frac{\partial \psi}{\partial r} + \frac{\hat{L}^2\psi}{2Mr^2} +
U_0 \frac{e^{-\gamma r}}{r} \psi = E \psi
\]
Представляем $\psi$ в виде:
\[
\psi = R(r) Y_{lm}(\theta, \alpha)
\]
Получаем:
\[
- \frac{\hbar^2}{2M} \frac{1}{r^2} \frac{d}{d r} r^2 \frac{d R}{d r} + \frac{\hbar^2 l (l+1)}{2Mr^2} R +
U_0 \frac{e^{-\gamma r}}{r} R = E R
\]
\section{Решение линейных уравнений второго порядка при известном частном решении}

Пусть дано уравнение:
\[
	y'' + f(x) y' + g(x) y = 0
\]
и $\varphi(x)$ -- его частное решение. Будем искать его второе решение в виде: $\psi \varphi$. При этом начальные условия перейдут в
\[
	y(x_0) = \psi (x_0)\varphi(x_0) = y_0
\] 
\[
	y'(x_0) = \psi' (x_0) \varphi (x_0) + \psi (x_0) \varphi' (x_0) = y'_0
\]
\[
	\psi (x_0) = \frac{y_0}{\varphi(x_0)}
\]
\[
	\psi' (x_0) \varphi^2 (x_0) = \varphi (x_0) y_0' - y_0 \varphi' (x_0)
\]
Тогда
\[
	\psi'' \varphi + 2 \psi' \varphi' + f \psi' \varphi = 0
\]
\[
	\frac{\psi''}{\psi'} = - 2 \frac{\varphi'}{\varphi} - f
\]
\[
	\psi' = \psi'(x_0) \frac{\varphi^2(x_0)}{\varphi^2(x)} \exp \left( - \int\limits_{x_0}^{x} f\,dx\right)
\]
\[
	\psi = \psi' (x_0) \varphi^2 (x_0) \int\limits_{x_0}^{x} \frac{1}{\varphi^2(x)} \exp \left( - \int\limits_{x_0}^{x} f\,dx\right) dx + \psi (x_0)
\]
\[
	y = \frac{\varphi(x)}{\varphi(x_0)} \left[\varphi (x_0) ( \varphi (x_0) y_0' - y_0 \varphi' (x_0)) \int\limits_{x_0}^{x} \frac{1}{\varphi^2(x)} \exp \left( - \int\limits_{x_0}^{x} f\,dx\right) dx + y_0 \right]
\]
Данный результат представляет собой немного изменённую формулу Лиувилля-Остроградского.
Определитель Вронского:
\[
	\begin{vmatrix}
		\varphi & \varphi \psi \\
		\varphi' & \varphi' \psi + \psi' \varphi
	\end{vmatrix} = \psi' \varphi^2 = 
	\psi'(x_0) \varphi^2(x_0) \exp \left( - \int\limits_{x_0}^{x} f\,dx\right) \ne 0, \text{ если } \psi'(x_0) \varphi^2(x_0) \ne 0
\]
То есть решения, полученные таким методом, линейно независимы, если $\psi'(x_0)\ne 0$ и $\varphi(x_0) \ne 0$
\section{Расширение изначально однородно заряженного шара}

\textit{Первый метод}. 

Рассмотрим эту задачу с точки зрения симметрии. Во-первых, понятно, что в системе отсутствует магнитное поле. Во-вторых, плотность заряда $\rho$ и поле скоростей $\vec{v}$ зависят только от расстояния до центра шара. Также скорость имеет только одну компоненту $v_r$, также как и электрическое поле $E_r$.
 
Выберем внутри шара радиуса $R$ заряд которой равен $Q$ сферу радиуса $r_0$. Такая сфера ограничивает заряд:
\[
	q = \frac{r_0^3}{R^3} Q
\]

Найдём как движется точка на поверхности такой сферы. Уравнение движения:
\[
	\frac{d^2 r}{dt^2} = \alpha E_r = \alpha  k \frac{q}{r^2}
\]
\[
	\frac{v_r^2}{2} - \frac{v_{0r}^2}{2} = \alpha k q \left(- \frac{1}{r} + \frac{1}{r_0}\right)
\]
\[
	v_r = \sqrt{v_{0r}^2 + \frac{2\alpha kq}{r_0} - \frac{2\alpha k q}{r}} = \sqrt{a - \frac{b}{r}}
\]
\[
	\int\limits_{r_0}^{r} 
	\frac{\sqrt{r} dr}{\sqrt{ar - b}}
	=
	t - t_0
\]
\[
	\frac{b}{a^{3/2}} 
	\left.
	\left(\sqrt{\frac{ar}{b}} \sqrt{\frac{ar}{b}-1} + 
	\ln \left|\sqrt{\frac{ar}{b} - 1} + \sqrt{\frac{ar}{b}}\right|\right) 
	\right|_{r_0}^{r}
	= t - t_0
\]
Плотность зарядов можно найти, определяя для каждого слоёв, между которыми находится заряд $dq$ толщина между слоями $dr_0$ в момент времени $t$ соответствующее расстояние между слоями:
$dr$:
\[
	\rho = \frac{1}{4\pi r^2}\frac{dr_0}{dr}\frac{dq}{dr_0}
\]
В случае если $v_{r}(0) = 0$, получаем:
\[
	b = 2\alpha k q
\]
\[
	a = \frac{2\alpha kq}{r_0}
\]
\[
	\frac{r_0^{3/2}}{(2\alpha k q)^{1/2}} 
	\left(\sqrt{\frac{r}{r_0}} \sqrt{\frac{r}{r_0}-1} + 
	\ln \left|\sqrt{\frac{r}{r_0} - 1} + \sqrt{\frac{r}{r_0}}\right|\right)
	= t
\]
\[
	\left(\sqrt{\frac{r}{r_0}} \sqrt{\frac{r}{r_0}-1} + 
	\ln \left|\sqrt{\frac{r}{r_0} - 1} + \sqrt{\frac{r}{r_0}}\right|\right)
	= \sqrt{
		\frac{2\alpha k Q}{R^{3}} } t
\]
На рисунке приведена зависимость $r(t)$, для различных значений $r_0$.
\begin{figure}[h!]
	\centering
	\includegraphics{images/png/sphere1.png}
\end{figure}
\section{Движение заряженной частицы в скрещенных полях}
Рассмотрим движение частицы массой $m$, зарядом $q$ в скрещенном поле $\vec{E} = -\vec{u}\times\vec{B}$. Уравнения движения:
\[
	\Dff{\vec{p}}{t} = q (\vec{E} + \vec{v} \times \vec{B}) =
	q (\vec{v} - \vec{u})\times\vec{B}
\]
Вводим собственное время $\tau$:
\[
	\Dff{\tau}{t} = \sqrt{1 - v^2/c^2} = \frac{mc^2}{W}
\]
Тогда:
\[
	\Dff{\vec{p}}{\tau} =
	\frac{q}{m} (\vec{p} - \vec{u}W/c^2)\times\vec{B}
\]
Вводим циклотронную частоту:
\[
	\vec{\omega}_0 = \frac{q}{m}\vec{B}
\]
\[
	\Dff{\vec{p}}{\tau} =
	- \vec{\omega}_0\times(\vec{p} - \vec{u}W/c^2)
\]
Рассмотрим также энергию:
\[
	W = \sqrt{p^2c^2 + m^2 c^4}
\]
\[
	\Dff{W}{\tau} = \frac{c^2}{W} \vec{p}\cdot\Dff{\vec{p}}{\tau} = 
	\vec{p}\cdot(\vec{\omega}_0\times\vec{u})
\]
Заметим теперь, что:
\[
	\vec{\omega}_0 \cdot \vec{p} = \vec\omega_0 \cdot \vec{p}_0
\]
\[
	\Dff{\vec{u}\cdot\vec{p}}{\tau} = - \vec{u}\cdot(\vec{\omega}_0\times\vec{p}) = \vec{p}\cdot(\vec{\omega}_0\times\vec{u}) = \Dff{W}{\tau} 
\]
В результате:
\[	
	W - \vec{u}\cdot\vec{p} = const = W_0 - \vec{u}\cdot\vec{p}_0 = W_0'
\]
\[
	\Dff{\vec{p}}{\tau} =
	- \vec{\omega}_0\times(\vec{p} - \vec{u}(\vec{u}\cdot\vec{p})/c^2 - W_0'/c^2)
\]
Введём правую тройку ортов:
\[
	\vec{e}_\omega = \frac{\vec\omega_0}{\omega_0} \quad
	\vec{e}_t = \frac{\vec{e}_\omega \times \vec{u}}{|\vec{e}_\omega \times \vec{u}|} = \frac{\vec{e}_\omega \times \vec{u}}{u\sin \theta} \quad
	\vec{e}_s = \vec{e}_\omega\times\vec{e}_t = \frac{\vec{e}_\omega u \cos \theta - \vec{u}}{u\sin \theta} = \frac{\vec{e}_\omega \cos \theta - \vec{u}/u}{\sin \theta}
\]
\[
	\vec{u} = u \cos \theta \vec{e}_\omega - u \sin \theta \vec{e}_s
\]
\[
	\Dff{p_s}{\tau} \vec{e}_s + \Dff{p_t}{\tau} \vec{e}_t =
	\omega_0 p_s \vec{e}_t - 
	\omega_0 p_t \vec{e}_s - 
	\omega_0 u^2 \sin^2 \theta p_s/c^2 \vec{e}_t + 
	\omega_0 u^2 \cos \theta \sin \theta p_\omega /c^2 \vec{e}_t +
	\omega_0 u \sin \theta W_0'/c^2 \vec{e}_t
\]
\[	
	\begin{aligned}
	& p_\omega = p_{0\omega} \\
	& \Dff{p_s}{\tau} = - \omega_0 p_t \\
	& \Dff{p_t}{\tau} = \omega_0 (1 - u^2 \sin^2 \theta/c^2) p_s + \omega_0 u \sin \theta W_0'/c^2 + \omega_0 u^2 \cos \theta \sin \theta p_{0\omega} /c^2 =
	\omega_0 (1 - u^2 \sin^2 \theta/c^2) p_s + \omega_0 u \sin \theta (W_0 + u \sin \theta p_{0s})/c^2
	\end{aligned}
\]
\[	
	\Dfs{p_t}{\tau} = \omega_0^2 (u^2 \sin^2 \theta/c^2 - 1) p_\tau
\]
Введём обозначения:
\[
	k = \omega_0 \sqrt{u^2 \sin^2 \theta/c^2 - 1}
\]
\[
	\begin{aligned}
	& m\Dff{x_t}{\tau} = p_t = A \sinh (k\tau) + p_{0t} \cosh(k\tau) \\
	& m\Dff{x_s}{\tau} = p_s = - \frac{\omega_0}{k} (A \cosh (k\tau) + p_{0t} \sinh (k\tau)) + \frac{\omega_0^2 u \sin \theta (W_0 + u \sin \theta p_{0s})}{k^2 c^2} \\
	& \Rightarrow A = \frac{\omega_0 u \sin \theta (W_0 + u \sin \theta p_{0s})}{k c^2} - \frac{k}{\omega_0} p_{0s}  = 
	\frac{\omega_0 (u \sin \theta W_0/c^2 + p_{0s})}{k} \\
	& \Rightarrow p_s = - \frac{\omega_0}{k} (A (\cosh (k\tau) - 1) + p_{0t} \sinh (k\tau)) + p_{0s} \\
	& m\Dff{x_\omega}{\tau} = p_\omega = p_{0\omega} \\
	& m c^2\Dff{t}{\tau} = W = W_0 + u \sin \theta (p_s - p_{0s})
	\end{aligned}
\]
\[
	\begin{aligned}
	& x_t = \frac{1}{mk} (A (\cosh (k\tau) - 1) + p_{0t} \sinh(k\tau)) + x_{0t} \\
	& x_s = - \frac{\omega_0}{k^2m} (A (\sinh (k\tau) - k\tau) + p_{0t} (\cosh (k\tau) - 1)) + \frac{p_{0s} \tau}{m} + x_{0s} \\
	& x_\omega = \frac{p_{0\omega} \tau}{m} + x_{0\omega} \\
	& t = \frac{1}{mc^2} (W_0 \tau + u \sin \theta (m (x_s - x_{0s}) - p_{0s} \tau))
	\end{aligned}
\]
Если $u \sin \theta = c$:
\[
	\begin{aligned}
	& \qquad A = A'/k = \frac{\omega_0 (W_0/c + p_{0s})}{k}\\
	& x_t = \frac{1}{m} (A'\frac{\tau^2}{2} + p_{0t} \tau) + x_{0t} \\
	& x_s = - \frac{\omega_0}{m} (A' \frac{\tau^3}{6} + p_{0t} \frac{\tau^2}{2} + \frac{p_{0s} \tau}{m} + x_{0s} \\
	& x_\omega = \frac{p_{0\omega} \tau}{m} + x_{0\omega} \\
	& t = \frac{1}{mc^2} (W_0 \tau + u \sin \theta (m (x_s - x_{0s}) - p_{0s} \tau))
	\end{aligned}
\]
Если $u \sin \theta < c$:
\[
	\begin{aligned}
	& \qquad k = i \omega \qquad A = - i A'\\
	& x_t = \frac{1}{m\omega} (- A' (\cos (\omega\tau) - 1) + p_{0t} \sin(\omega\tau)) + x_{0t} \\
	& x_s = \frac{\omega_0}{\omega^2m} (A (\sin (\omega \tau) - \omega \tau) + p_{0t} (\cos (\omega \tau) - 1)) + \frac{p_{0s} \tau}{m} + x_{0s} \\
	& x_\omega = \frac{p_{0\omega} \tau}{m} + x_{0\omega} \\
	& t = \frac{1}{mc^2} (W_0 \tau + u \sin \theta (m (x_s - x_{0s}) - p_{0s} \tau))
	\end{aligned}
\]

\section{Движение частицы в материальной среде в нерелятивистском случае}

Рассмотрим материальную среду, частицы которой не взаимодействуют между собой. Уравнения движения для такой среды будут иметь вид уравнений Эйлера с нулевой правой частью:
\[
	\rho \Dff{\vec{v}}{t} = \rho \left(\dff{\vec{v}}{t} + (\vec{v}\cdot\nabla) \vec{v}\right) = 0
\]
\[
	\dff{\rho}{t} + \div (\rho \vec{v}) = 0
\]
То, что в среде находится частица учтём с помощью выражения:
\[
	\rho = \rho_f + m \delta(\vec{r} - \vec{\xi}(t))
\]
Из уравнения непрерывности следует:
\[
	\dff{\rho_f}{t} + \div (\rho_f \vec{v}) 
	- m \Dff{\vec{\xi}}{t} \cdot \nabla\delta(\vec{r} - \vec{\xi}(t)) 
	+ m \vec{v} \cdot \nabla\delta(\vec{r} - \vec{\xi}(t)) 
	+ m \delta(\vec{r} - \vec{\xi}(t)) \div{\vec{v}} = 0
\]
Можно предположить (к сожалению мне это казалось абсолютно точным, но вдруг я где-то ошибся), что отсюда следует:
\[
	\dff{\rho_f}{t} + \div (\rho_f \vec{v}) = 0
\]
\[
	\div \vec{v} = 0 \text{ но только в точке с частицей}
\]
\[
	\vec{v} = \Dff{\vec{\xi}}{t} \text{ но только в точке с частицей}
\]
Разделяем выражения и интегрируем по $\vec{r}$, учитывая, что на $\vec{r} = \vec{\xi}(t)$ $\vec{v} = d\vec{\xi}/dt$:
\[
	m\Dfs{\vec{\xi}}{t} = - \int \rho_f \left(\dff{\vec{v}}{t} + (\vec{v}\cdot\nabla) \vec{v}\right) dV
\]
На деле как легко видеть скорости частиц не меняются при движении, если поле скоростей непрерывно. Интерес представляет один единственный случай, когда поле скоростей терпит разрыв. В этом случае частица догоняет и сталкивается с частицами среды и, как следствие, движется уже ускоренно. Здесь есть несколько важных и нетривиальных моментов. Очень хочется, чтобы разрыв был в одной единственной точке -- в точке с частицей. Но если разрыв конечен, то для частицы как материальной точки производные от скорости обратятся в нуль и ничего не будет наблюдаться. Если предположить, что в среде существует фронт или линия на которой поле скоростей терпит разрыв, то расположение этой линии или фронта может быть вообще говоря произвольным. Это путает все карты.
Как облечь всё изложенное выше в математическую форму я пока не знаю. По этой причине на задачу пришлось забить.
\section{Волновые уравнения}
\subsection{Однородное одномерное волновое уравнение в неограниченной среде с заданными начальными условиями}

Как оно выглядит? Да, вот так:
\[
	\dfs{u}{x} - \frac{1}{v^2} \dfs{u}{t} = 0
\]
В обычном случае $v = const$. Начальные условия:
\[
	\begin{aligned}
		& u(0, x) = f(x) \\
		& \dff{u}{t} \Big|_{(0, x)} = g(x)
	\end{aligned}
\]
Его очень легко решить, если воспользоваться заменой координат:
\[
	\begin{aligned}
	& \eta = x + v t \\
	& \xi = x - v t
	\end{aligned}	
\]
\[
	\begin{aligned}
		& \dff{u}{x} = \dff{u}{\eta} \dff{\eta}{x} + \dff{u}{\xi} \dff{\xi}{x} =
			\dff{u}{\eta} + \dff{u}{\xi} \\
		& \dfs{u}{x} = \dfs{u}{\eta} \dff{\eta}{x} + \dfss{u}{\xi}{\eta} \dff{\xi}{x} + \dfss{u}{\eta}{\xi} \dff{\eta}{x} +  \dfs{u}{\xi} \dff{\xi}{x} =
			\dfs{u}{\eta} +  \dfs{u}{\xi} + 2 \dfss{u}{\xi}{\eta} \\
		& \dff{u}{t} = \dff{u}{\eta} \dff{\eta}{t} + \dff{u}{\xi} \dff{\xi}{t} =
			v \left(\dff{u}{\eta} - \dff{u}{\xi}\right) \\
		& \dfs{u}{t} = v\dfs{u}{\eta} \dff{\eta}{t} + v\dfss{u}{\xi}{\eta} \dff{\xi}{t} - v\dfss{u}{\eta}{\xi} \dff{\eta}{t} -  v\dfs{u}{\xi} \dff{\xi}{t} =
		v^2\left(\dfs{u}{\eta} +  \dfs{u}{\xi} - 2 \dfss{u}{\xi}{\eta} \right)\\ 
	\end{aligned}
\]
\[
	\dfs{u}{x} - \frac{1}{v^2} \dfs{u}{t} = 4 \dfss{u}{\xi}{\eta} = 0
\]
\[
	\dff{u}{\eta} = U_1(\eta)
\]
\[
	u = U(\eta) + V(\xi)
\]
\[
	\begin{aligned}
	& u(0, x) = U(x) + V(x) = f(x) \\
	& \dff{u}{t} \Big|_{(0, x)} = v \left(U'(x) - V'(x)\right) = g(x)
	\end{aligned}
	\Rightarrow
	\begin{aligned}
	& U(x) + V(x) = f(x) \\
	& U(x) - V(x) = \frac{1}{v} \int g(x) dx = \frac{G(x)}{v}
	\end{aligned}
\]
\[
	u = \frac{f(x + v t) + f(x - v t)}{2} + \frac{G(x + v t) - G(x - v t)}{2 v} = \frac{f(x + v t) + f(x - v t)}{2} + \frac{1}{2 v} \int\limits_{x - v t}^{x + v t} g(x) dx
\]

\subsection{Метод преобразования Фурье для неоднородного одномерного волнового уравнения в бесконечной среде с нулевыми начальными условиями}

\[
\dfs{u}{x} - \frac{1}{v^2} \dfs{u}{t} = s(x, t)
\]
$v = const$. Начальные условия:
\[
	\begin{aligned}
	& u(0, x) = 0 \\
	& \dff{u}{t} \Big|_{(0, x)} = 0
	\end{aligned}
\]
Только координата $x$ меняется от $-\infty$ до $\infty$, поэтому преобразование Фурье будет иметь вид:
\[
	\begin{aligned}
	& u(x, t) = \int\limits_{-\infty}^{\infty} U(k, t) e^{-ikx} dk \qquad\Leftrightarrow \qquad  U(k, t) = \frac{1}{2\pi}\int\limits_{-\infty}^{\infty} u(x, t) e^{ikx} dx\\
	& s(x, t) = \int\limits_{-\infty}^{\infty} S(k, t) e^{-ikx} dk
	\qquad\Leftrightarrow \qquad
	S(k, t) = \frac{1}{2\pi}\int\limits_{-\infty}^{\infty} s(x, t) e^{ikx} dx
	\end{aligned}
\]
Выполняем преобразование Фурье исходного уравнения:
\[
	\frac{1}{2\pi} \int\limits_{-\infty}^{\infty}\dfs{u}{x} e^{ikx} dx = 
	\frac{1}{2\pi} \dff{u}{x} e^{ikx} \Big|_{-\infty}^{\infty} - 
	\frac{ik}{2\pi} \int\limits_{-\infty}^{\infty}\dff{u}{x} e^{ikx} dx =
	\frac{1}{2\pi} \left(\dff{u}{x} - ik u \right) e^{ikx} \Big|_{-\infty}^{\infty}
	- k^2 U
\]
При условии, что:
\[
	 \lim\limits_{x \to \pm \infty }\left(\dff{u}{x} - ik u \right) = 0
\]
получаем преобразованное уравнение:
\[
	- k^2 U - \frac{1}{v^2} \Dfs{U}{t} = S(k, t)
\]
И начальные условия:
\[
	\begin{aligned}
	& U(k, 0) = 0 \\
	& \Dff{U}{t} \Big|_{(k, 0)} = 0
	\end{aligned}
\]
Решение этого уравнения:
\[
	U(k, t) = - \frac{v}{k} \int\limits_{0}^{t} \sin(vk (t - t')) S(k, t') dt'
\]
\[
	\begin{gathered}
	u(x, t) = -\int\limits_{0}^{t} \! \int\limits_{-\infty}^{\infty} \frac{v}{k} \sin(vk (t - t')) S(k, t') e^{-ikx}  dk\, dt' =
	-\int\limits_{0}^{t} \! \int\limits_{-\infty}^{\infty} \! \int\limits_{-\infty}^{\infty} \frac{v}{2k\pi} \sin(vk (t - t')) s(x', t') e^{-ik(x - x')}  dk\, dt'\, dx' = \\ =
	-\frac{v}{4} \int\limits_{0}^{t}  \! \int\limits_{-\infty}^{\infty} [ \sign (x - x' + v(t - t')) - \sign (x - x' - v (t - t')) ] s(x', t') dt' dx' = \\ =
	-\frac{v}{4} \int\limits_{0}^{t} \! \int\limits_{-\infty}^{\infty} [ \sign (\xi + v \tau) - \sign (\xi - v\tau) ] s(x - \xi, t - \tau) d\tau d\xi 
	\end{gathered}
\]
\[
	\begin{aligned}
	\sign (x + a) - \sign (x - a) = 
	\begin{cases}
	\left\{
	\begin{aligned}
	& 0, & x > a, \\
	& 1, & x = a, \\
	& 2, & -a < x < a, \\
	& 1, & x = - a, \\
	& 0, & x < - a.
	\end{aligned}
	\right\},
	& a > 0, \\
	0,
	& a = 0, \\
	\left\{
	\begin{aligned}
	& 0, & x > - a, \\
	& -1, & x = - a, \\
	& -2, & a < x < -a, \\
	& -1, & x = a, \\
	& 0, & x < a.
	\end{aligned}
	\right\},
	& a < 0. \\
	\end{cases}
	=
	\begin{cases}
	0, & |x|>|a|, \\ 
	\sign(a), & |x|=|a|, \\ 
	2 \sign(a), & |x| < |a|.
	\end{cases}
	=
	2 \eta(|a| - |x|) \sign(a)
	\end{aligned}
\]
\[
	 \sign (\xi + v \tau) - \sign (\xi - v\tau) = 2 \eta(v|\tau| - |\xi|) \sign(v\tau)
\]
\[
	\begin{aligned}
	u(x, t) =
	-\frac{v}{2} \int\limits_{0}^{t} \! \int\limits_{-\infty}^{\infty} \eta(v|\tau| - |\xi|) \sign(v\tau) s(x - \xi, t - \tau) d\tau\, d\xi = \\ =
	[\text{если, $s(x - \xi, t - \tau)$ не имеет $\delta$-образных разывов}] = \\ =
	-\frac{v}{2} \int\limits_{0}^{t} \! \int\limits_{-v\tau}^{v\tau} s(x - \xi, t - \tau) d\tau\, d\xi
	\end{aligned}
\]
%Проверим, что найденное решение удовлетворяет уравнению:
%\[
%	\begin{gathered}
%	\dfs{u}{x} - \frac{1}{v^2} \dfs{u}{t} =
%	-\frac{v}{2} \int\limits_{0}^{t} \! \int\limits_{-\infty}^{\infty} \eta(\tau - |\xi|/v)
%	\left(\dfs{}{x} - \frac{1}{v^2} \dfs{}{t}\right) s(x - \xi, t - \tau) d\tau\, d\xi = \\ =
%	-\frac{v}{2} \int\limits_{0}^{t} \! \int\limits_{-\infty}^{\infty} \eta(\tau - |\xi|/v)
%	\left(\dfs{}{\xi} - \frac{1}{v^2} \dfs{}{\tau}\right) s(x - \xi, t - \tau) d\tau\, d\xi
%	\end{gathered}
%\]

\subsection{Примеры}

\textbf{1. Пусть:}
\[
s(x, t) = \delta(x - x_0) \delta (t - t_0) 
\]
что соответствует импульсу, мгновенно переданному отдельной точке струны.
\[
\begin{gathered}
u(x, t) =
-\frac{v}{4} \int\limits_{0}^{t} \! \int\limits_{-\infty}^{\infty} [ \sign (\xi + v \tau) - \sign (\xi - v\tau) ] s(x - \xi, t - \tau) d\tau d\xi =
\\ =
-\frac{v}{4} \int\limits_{0}^{t} \! \int\limits_{-\infty}^{\infty} [ \sign (\xi + v \tau) - \sign (\xi - v\tau) ] \delta(x - x_0 - \xi) \delta (t - t_0 - \tau)  d\tau d\xi =
\\ =
-\frac{v}{4} \int\limits_{0}^{t}  [ \sign (x - x_0 + v \tau) - \sign (x - x_0 - v\tau) ] \delta (t - t_0 - \tau)  d\tau = 
\\ =
-\frac{v}{4} [ \sign (x - x_0 + v (t - t_0)) - \sign (x - x_0 - v(t - t_0)) ] \eta(t - t_0)
\end{gathered}
\]
Проверяем:
\[
\dfs{u}{x} = - \frac{v}{2} [\delta' (x - x_0 + v (t - t_0)) - \delta' (x - x_0 - v(t - t_0))]\eta(t - t_0)
\]
\[
\begin{gathered}
\dfs{u}{t} = -\frac{v}{4} \left\{
2 v^2 [\delta' (x - x_0 + v (t - t_0)) - \delta' (x - x_0 - v(t - t_0)) ] \eta(t - t_0) +
\right.\\ + \left.
4 v [\delta (x - x_0 + v (t - t_0)) + \delta (x - x_0 - v(t - t_0))]\delta(t - t_0) +
\right.\\ + \left.
[ \sign (x - x_0 + v (t - t_0)) - \sign (x - x_0 - v(t - t_0)) ] \delta'(t - t_0)
\right\}
\end{gathered}
\]
Воспользуемся тождествами:
\[
f(x)\delta(x - x_0) + [F(x) - F(x_0)] \delta'(x - x_0) = 0, \qquad F(x) = \int f(x) dx
\]
\[
f(x) \delta(x -x_0) = f(x_0) \delta(x - x_0)
\]
Кое-что подсократилось:
\[
\begin{gathered}
\dfs{u}{t} = -\frac{v}{4} \left\{
2 v^2 [\delta' (x - x_0 + v (t - t_0)) - \delta' (x - x_0 - v(t - t_0)) ] \eta(t - t_0) +
\right.\\ + \left.
4 v \delta (x - x_0) \delta(t - t_0)
\right\}
\end{gathered}
\]
\[
\Rightarrow
\dfs{u}{x} - \frac{1}{v^2} \dfs{u}{t} = \delta (x - x_0) \delta(t - t_0)
\]
На рисунке показан профиль струны в различные моменты времени:
\begin{center}
	\includegraphics[width=0.5\textwidth]{images/png/for_delta_force.png}
\end{center}

\textbf{Одно тождество}

Рассмотрим функцию вида:
\[
	u(x, t) = \eta(t - t_0) \eta(\xi) f(\xi) = g(\xi, t) f(\xi), \text{ если } \xi = |t - t_0| - |x - x_0|/v 
\]
\[
	\begin{gathered}
	u''_{tt} = 
	\dfs{g}{t} f + 2 \dff{g}{t} \dff{f}{\xi} \dff{\xi}{t} + g \dfs{f}{t} 
	\end{gathered}
\]
\[
	\begin{gathered}
	u''_{xx} = 
	\dfs{g}{x} f + 2 \dff{g}{x} \dff{f}{\xi} \dff{\xi}{x} + g \dfs{f}{x} 
	\end{gathered}
\]
\[
	\begin{gathered}
	u''_{xx} - \frac{1}{v^2} u''_{tt} = 
	f\left(\dfs{g}{x} - \frac{1}{v^2} \dfs{g}{t} \right) + 
	2 \dff{f}{\xi} \dff{g}{\xi} \left[\left(\dff{\xi}{x}\right)^2 - \frac{1}{v^2} \left(\dff{\xi}{t}\right)^2 \right] -
	\frac{2}{v^2} \delta(t - t_0) \eta (\xi) \dff{f}{\xi} \dff{\xi}{t} + 
	\\ +
	g\left[\df{x} \left(\dff{f}{\xi} \dff{\xi}{x} \right) - \frac{1}{v^2} \df{t} \left(\dff{f}{\xi} \dff{\xi}{t} \right)  \right]
	= 
	\\ =
	f\left(\dfs{g}{x} - \frac{1}{v^2} \dfs{g}{t} \right) + 
	2 \dff{f}{\xi} \dff{g}{\xi} \left[\left(\dff{\xi}{x}\right)^2 - \frac{1}{v^2} \left(\dff{\xi}{t}\right)^2 \right] -
	\frac{2}{v^2} \delta(t - t_0) \eta (\xi) \dff{f}{\xi} \dff{\xi}{t} + 
	\\ +
	g\dfs{f}{\xi} \left[\left(\dff{\xi}{x}\right)^2 - \frac{1}{v^2} \left(\dff{\xi}{t}\right)^2 \right] +
	g \dff{f}{\xi} \left(\dfs{\xi}{x} - \frac{1}{v^2} \dfs{\xi}{t} \right)
	\end{gathered}
\]
\[
	\dfs{g}{x} - \frac{1}{v^2} \dfs{g}{t} = - \frac{2}{v} \delta(t - t_0) \delta(x - x_0) \quad \text{(см. пред. пример)}
\]
\[
	\begin{aligned}
	& \dff{\xi}{t} = \sign(t - t_0) & \dff{\xi}{x} = - \sign(x - x_0) \frac{1}{v} \\
	& \dfs{\xi}{t} = 2\delta(t - t_0) & \dfs{\xi}{x} = - 2\delta(x - x_0) \frac{1}{v} \\
	\end{aligned}
\]
\[
	 \begin{gathered}
	 \dff{g}{\xi} \left[\left(\dff{\xi}{x}\right)^2 - \frac{1}{v^2} \left(\dff{\xi}{t}\right)^2 \right] =
	 \eta(t - t_0) \delta(\xi) \frac{1 - 2 \eta(-|x - x_0|) - 1 + 2 \eta(-|t - t_0|)}{v^2} =
	 \\ =
	 \frac{2\eta(t - t_0)}{v^2} (\delta(-|x - x_0|/v) \eta(-|t - t_0|) - \delta(\xi) \eta(-v|t - t_0|) ) = 0
	 \end{gathered}
\]
\[
	\delta(t - t_0) \dff{\xi}{t} = \delta(t - t_0) \sign(t - t_0)  = 0
\]
\[
	\begin{gathered}
	g\left[\left(\dff{\xi}{x}\right)^2 - \frac{1}{v^2} \left(\dff{\xi}{t}\right)^2 \right] =
	\frac{2\eta(t - t_0)}{v^2} (\eta(-|x - x_0|/v) \eta(-|t - t_0|) - \eta(|t - t_0|) \eta(-|x - x_0|) ) =
	\\ =
	- \frac{2\eta(t - t_0)}{v^2} \eta(-|x - x_0|/v) \sign^2(t - t_0)
	\end{gathered}
\]
\[
	\begin{gathered}
	g \left(\dfs{\xi}{x} - \frac{1}{v^2} \dfs{\xi}{t} \right) = 
	- \eta(t - t_0) \eta(\xi) \frac{2}{v^2} \left(\delta((x - x_0)/v) + \delta(t - t_0) \right) =
	\\ =
	- \frac{2}{v^2} \left(\eta^2(t - t_0) \delta((x - x_0)/v) + \frac{1}{2} \eta(-|x - x_0|/v)\delta(t - t_0) \right) =
	\\ =
	- \frac{2}{v^2} \left(\eta(t - t_0) \delta((x - x_0)/v) - \frac{1}{2} \eta(-|t - t_0|) \delta((x - x_0)/v)  + \frac{1}{2} \eta(-|x - x_0|/v)\delta(t - t_0) \right) =
	\\ =
	[\text{чуть выше такое уже встречалось}] =
	- \frac{2}{v} \eta(t - t_0) \delta(x - x_0)
	\end{gathered}
\]
В результате:
\[
	u''_{xx} - \frac{1}{v^2} u''_{tt}  = - \frac{2}{v} f(0) \delta(t - t_0) \delta(x - x_0) -
	f''(|t - t_0|) \frac{\eta(t - t_0)}{v^2} \sign^2(t - t_0) \eta(-|x - x_0|) -
	\frac{2}{v} f'(|t - t_0|) \eta(t - t_0) \delta(x - x_0)
\]

Здесь следует отметить одну деталь: результат имеет вид:
\[
	A\delta(x) + B\eta(-|x|)
\]
Из интегрального определения $\delta-$функции:
\[
	A\delta(x) + B\eta(-|x|) = A\delta(x)
\]
И
\[
	u''_{xx} - \frac{1}{v^2} u''_{tt}  = - \frac{2}{v} f(0) \delta(t - t_0) \delta(x - x_0) -
	\frac{2}{v} f'(|t - t_0|) \eta(t - t_0) \delta(x - x_0)
\]

\textbf{2. Пусть:}
\[
s(x, t) = \delta(x - x_0) \eta(t - t_0)
\]
что соответствует постоянной во времени силе, действующей в фиксированной точке струны, начиная с некоторого момента.
\[
\begin{gathered}
u(x, t) =
-\frac{v}{4} \int\limits_{0}^{t} \! \int\limits_{-\infty}^{\infty} [ \sign (\xi + v \tau) - \sign (\xi - v\tau) ]  \delta(x - x_0 - \xi) \eta(t - t_0 - \tau) d\tau d\xi =
\\ =
-\frac{v}{4} \int\limits_{0}^{t} [ \sign (x - x_0 + v \tau) - \sign (x - x_0 - v\tau) ] \eta (t - t_0 - \tau)  d\tau =
\\ =
-\frac{v}{4} \eta(t - t_0) \int\limits_{0}^{t - t_0} [ \sign (x - x_0 + v \tau) - \sign (x - x_0 - v\tau) ] d\tau = 
-\frac{v}{4} \eta(t - t_0) \int\limits_{0}^{t - t_0} 2 \eta(v|\tau| - |x - x_0|) d\tau =
\\ =
-\frac{v}{4} \eta(t - t_0) \eta(t - t_0 - |x - x_0|/v) \int\limits_{|x - x_0|/v}^{t - t_0} 2 \eta(v\tau - |x - x_0|) d\tau = 
-\frac{v}{2} \eta(t - t_0) \eta(t - t_0 - |x - x_0|/v) \int\limits_{|x - x_0|/v}^{t - t_0} d\tau =
\\ =
-\frac{v}{2} \eta(t - t_0) \eta(t - t_0 - |x - x_0|/v) (t - t_0 - |x - x_0|/v) 
\end{gathered}
\]
Проверка по формуле выше показывает, что это правильный ответ. Форма струны в различные моменты времени показана на рисунке:
\begin{center}
	\includegraphics[width=0.5\textwidth]{images/png/for_deltac_force.png}
\end{center}

\textbf{3. Пусть:}
\[
	s(x, t) = \begin{cases}
	0, & x \notin [-l/2, l/2] \\
	S_0\eta(t - t_0)/l, & x \in [-l/2, l/2]
	\end{cases} =
	\frac{S_0}{l} [ \eta(x + l/2) - \eta(x - l/2)] \eta(t - t_0)
\]
что соответствует постоянной силе, действующей на участок струны.
\[
\begin{gathered}
	u(x, t) =
	-\frac{v}{2} \int\limits_{0}^{t} \! \int\limits_{-v\tau}^{v\tau} s(x - \xi, t - \tau) d\tau\, d\xi = 
	\\ =
	-\frac{v}{2} \int\limits_{0}^{t} \! \int\limits_{-v\tau}^{v\tau} \frac{S_0}{l} [ \eta(x + l/2 - \xi) - \eta(x - l/2 - \xi)]\eta(t - t_0 - \tau) d\tau\, d\xi = 
	\\ =
	-\frac{v}{2} \int\limits_{0}^{t - t_0} \! \int\limits_{-v\tau}^{v\tau} \frac{S_0}{l} [ \eta(x + l/2 - \xi) - \eta(x - l/2 - \xi)] d\tau\, d\xi = 
	-\frac{v}{2}\frac{S_0}{l} \int\limits_{S} d\tau\, d\xi
\end{gathered}
\]
Область интегрирования $S$ приведена на рисунке, и представляет собой пересечение полосы шириной $l$ и треугольника, ограниченного кривыми $y = v(t - t_0)$, $y = x$, $y = -x$.
\begin{center}
	\includegraphics{images/png/for_example3.png}
	%\usetikzlibrary{patterns}
\begin{tikzpicture}

\draw[thick,-stealth]  (0,-1) -- (0,6);
\draw[thick,-stealth] (-4,0) -- (4,0);
\draw[very thick, pattern=vertical lines] (-3,4) -- (3, 4) -- (0,0) -- (-3, 4);
\draw[very thick, dashed] (2, -1) -- (2, 5);
\draw[very thick, dashed] (-1, -1) -- (-1, 5);
\path[pattern=horizontal lines] (2, -1) rectangle (-1, 5);

\node[below, fill = white] at (0.5,0) {$x$};
\node[above right, fill = white] at (0,4) {$v(t - t_0)$};
\node[below left] at (-1,0) {$x - l/2$};
\node[below right] at (2,0) {$x + l/2$};

\end{tikzpicture}
\end{center}
Решение можно представить с помощью функций, определяющих площадь пересечения треугольника с левой полуплоскостью, заданной прямой параллельной оси ординат:
\[
\begin{aligned}
& S_+(x, x_A, x_B, x_C, y_B) =  
\frac{1}{2} y_B \frac{(x - x_A)^2}{x_B - x_A} \eta(x - x_A) \eta(x_B - x)  + \\ & +
\frac{1}{2} y_B \left[(x_B - x_A) + \frac{2 x_C - x - x_B}{x_C - x_B} (x - x_B)\right] \eta(x - x_B) \eta(x_C - x) +
\frac{1}{2} y_B (x_C - x_A) \eta(x - x_C)
\end{aligned}
\]
\[
	\begin{aligned}
	u(x, t) & = -\frac{v}{2}\frac{S_0}{l} [S_+(x + l/2, -v(t - t_0), 0, v(t-t_0), v(t - t_0)) - S_+(x - l/2, -v(t - t_0), 0, v(t-t_0), v(t - t_0))]
	\end{aligned}
\]
К сожалению, выражение слишком громоздко, и его трудно проверить. На рисунке приведена форма струны в различные моменты времени:
\begin{center}
	\includegraphics[width=0.5\textwidth]{images/png/for_const_force.png}
\end{center}
Примечательно, что сначала график пологий, но со временем приобретает более параболическую форму.

\textbf{4. Пусть:}

\[
s(x, t) = \delta(x - x_0) e^{i\omega t} \eta(t - t_0)
\]
что соответствует силе действующей в точке по синусоидальному закону.
\[
\begin{gathered}
u(x, t) =
-\frac{v}{4} \int\limits_{0}^{t} \! \int\limits_{-\infty}^{\infty} [ \sign (\xi + v \tau) - \sign (\xi - v\tau) ] s(x - \xi, t - \tau) d\tau d\xi =
\\ =
-\frac{v}{4} \int\limits_{0}^{t} \! \int\limits_{-\infty}^{\infty} [ \sign (\xi + v \tau) - \sign (\xi - v\tau) ] \delta(x - x_0 - \xi) e^{i\omega(t - \tau)} \eta(t - t_0 - \tau) d\tau d\xi =
\\ =
-\frac{v}{2} \eta(t - t_0) \int\limits_{0}^{t - t_0} \eta(v\tau - |x - x_0|) e^{i\omega(t - \tau)} d\tau =
-\frac{v}{2} \eta(t - t_0) \eta(t - t_0 - |x -x_0|/v) e^{i\omega t} \int\limits_{|x - x_0|/v}^{t - t_0} e^{-i\omega\tau} d\tau =
\\ =
\frac{v}{2i\omega} \eta(t - t_0) \eta(t - t_0 - |x -x_0|/v) e^{i\omega t} e^{-i\omega(t - t_0 - |x - x_0|/v)} =
\frac{v}{2i\omega} \eta(t - t_0) \eta(t - t_0 - |x -x_0|/v) \left(e^{i\omega t_0} - e^{i\omega(t - |x - x_0|/v)} \right)
\end{gathered}
\]
Как легко видеть, ответ правильный. Действительная часть:
\[
	u(x, t) = \frac{v}{2\omega} \eta(t - t_0) \eta(t - t_0 - |x -x_0|/v) \left(\sin (\omega t_0) - \sin(\omega(t - |x - x_0|/v)) \right)
\]
Форма в различные моменты времени представлена на рисунке:
\begin{center}
	\includegraphics[width=0.5\textwidth]{images/png/for_sin_force.png}
\end{center}

\subsection{Вариационный принцип}

Рассмотрим функционал:
\[
	J(u) = \int\limits_{R^2} \left[\frac{1}{v^2} \left(\dff{u}{t}\right)^2 - \left(\dff{u}{x}\right)^2 \right] dx\, dt  
\]
\[
	\begin{gathered}
	\delta \left[\frac{1}{v^2} \left(\dff{u}{t}\right)^2 - \left(\dff{u}{x}\right)^2 \right] = 
	2 \left[\frac{1}{v^2} \dff{u}{t} \dff{\delta u}{t}  - \dff{u}{x} \dff{\delta u}{x} \right] =
	2 \left[\frac{1}{v^2} \df{t} \left(\dff{u}{t} \delta u \right)  - \df{x} \left(\dff{u}{x} \delta u \right) \right] -
	2 \left[\frac{1}{v^2} \dfs{u}{t} - \dfs{u}{x}\right] \delta u
	\end{gathered}
\]
\[
	\Rightarrow \quad \frac{1}{v^2} \dfs{u}{t} - \dfs{u}{x} = 0
\]
Для функционала:
\[
	J(u) = \int\limits_{R^2} \left[\frac{1}{v^2} \left(\dff{u}{t}\right)^2 - \left(\dff{u}{x}\right)^2 - 2 s(x, t) u \right] dx\, dt
\]
\[
	\frac{1}{v^2} \dfs{u}{t} - \dfs{u}{x} = s(x, t)
\]
Выполним преобразование Фурье:
\[
	u = u^* = \int\limits_{R^2} U(k, \omega) e^{i(\omega t - k x)} dk\, d\omega
	\qquad
	s = s^* = \int\limits_{R^2} S(k, \omega) e^{i(\omega t - k x)} dk\, d\omega
\]
\[
	\begin{gathered}
	J(U) = 
	\\ =
	\int\limits_{R^2} 
	\int\limits_{R^2}
	\int\limits_{R^2}
	\left[\frac{1}{v^2} \omega \omega' U^*(k, \omega) U(k', \omega') - k k' U^*(k, \omega) U(k', \omega') - 2 S^*(k, \omega) U(k', \omega') \right]
	e^{i((\omega' - \omega) t - (k' - k) x)}  dk\, d\omega\, dk'\, d\omega'\, dx\, dt
	=
	\\ =
	2\pi 
	\int\limits_{R^2} 
	\int\limits_{R^2}
	\left[\frac{1}{v^2} \omega \omega' U^*(k, \omega) U(k', \omega') - k k' U^*(k, \omega) U(k', \omega') - 2 S^*(k, \omega) U(k', \omega') \right]
	\delta(\omega' - \omega) \delta(k' - k) dk\, d\omega\, dk'\, d\omega'
	=
	\\ =
	2\pi 
	\int\limits_{R^2} 
	\left[\frac{1}{v^2} \omega^2 U^*(k, \omega) U(k, \omega) - k^2 U^*(k, \omega) U(k, \omega) - 2 S^*(k, \omega) U(k, \omega) \right]
	dk\, d\omega
	\end{gathered}
\]

\subsection{Решение n-мерного волнового уравнения с дельтообразной правой частью}

Волновое уравнение с дельтообразной правой частью в бесконечном пространстве-времени чудесная вещь:
\[
	\Delta u - \frac{\partial^2 u}{\partial t^2} = \delta(t - t')\delta(\vec{r} - \vec{r}') 
\]
Выполняем преобразование Фурье:
\[
	u = \int U(\vec{k}, \omega) e^{i(\vec{k}\cdot\vec{r} - \omega t)} dV_k dt
\]
Введём обозначения:
\[
	\vec{R} = \vec{r} - \vec{r}' \quad T = t - t'
\]
\[
	\delta(\vec{R}) \delta(T) = \frac{1}{(2\pi)^{n + 1}} \int e^{i(\vec{k}\cdot\vec{R} - \omega T)} dV_k d\omega
\]
Решение уравнения:
\[
	u = \frac{1}{(2\pi)^{n + 1}} \int \frac{e^{i(\vec{k}\cdot\vec{R} - \omega T)}}{\omega^2 - k^2} dV_k d\omega =  
	\frac{1}{(2\pi)^{n + 1}} \frac{2 \pi^{(n - 1)/2}}{\Gamma((n - 1)/2)} 
	\int\limits_{-\infty}^{\infty}\!
	\int\limits_{0}^{\infty}\!
	\int\limits_{0}^{\pi}
	 \frac{e^{i(k R \cos \theta - \omega T)}}{\omega^2 - k^2} k^{n - 1} \sin^{n - 2} \theta\, d\theta\, dk\, d\omega = (*)
\]
Учитываем соотношения:
\[
	\frac{1}{\omega^2 - k^2} = \frac{1}{2k} \left(\frac{1}{\omega - k} - \frac{1}{\omega + k} \right)
\]
\[
	\int\limits_{-\infty}^{\infty} \frac{e^{ia x}}{x - x_0} f(x) dx =
	i e^{iax_0} f(x_0) \pi \sign a
\]
\[
	\int\limits_0^\pi e^{i\alpha \cos \theta} \sin^n \theta \, d\theta
	=
	\Gamma \left(\dfrac{n + 1}{2}\right) 2^{n/2} \sqrt{\pi} \frac{J_{n/2} (\alpha)}{\alpha^{n/2}}  
\]
Получаем:
\[
	\int\limits_{-\infty}^{\infty} \frac{e^{ - i\omega T}}{\omega^2 - k^2} d\omega 
	=
	\frac{1}{2k} \int\limits_{-\infty}^{\infty} \frac{e^{ - i\omega T}}{\omega - k} d\omega
	-
	\frac{1}{2k} \int\limits_{-\infty}^{\infty} \frac{e^{ - i\omega T}}{\omega + k} d\omega
	= \\ =
	\frac{i\pi \sign (-T)}{2k} \left[e^{-i k T} - e^{ikT}\right]
	=
	- \frac{\pi \sign (T)}{k} \sin (kT)
\]
\[
	\begin{gathered}
	(*) =
	- \frac{1}{(2\pi)^{n}} \frac{\pi^{(n - 1)/2}}{\Gamma((n - 1)/2)} 
	\int\limits_{0}^{\infty}
	\sin (kT) \sign(T) k^{n - 2} 
	\int\limits_{0}^{\pi}
	e^{i(k R \cos \theta)} 
	\sin^{n - 2} \theta\, d\theta\, dk
	= \\ =
	- \frac{1}{(2\pi)^{n}} \frac{\pi^{(n - 1)/2}}{\Gamma((n - 1)/2)} 
	\Gamma((n - 1)/2) 2^{(n - 2)/2} \pi^{1/2}
	\int\limits_{0}^{\infty}
	\sin (kT) \sign(T) k^{n - 2} 
	\frac{J_{(n - 2)/2} (kR)}{(kR)^{(n - 2)/2}}   dk
	= \\ =
	- \frac{1}{2^{(n + 2)/2} \pi^{n/2}} \frac{\sign(T)}{R^{n - 1}}
	\int\limits_{0}^{\infty}
	\sin (xT/R) x^{(n - 2)/2} 
	J_{(n - 2)/2} (x)   dx
	\end{gathered} 
\]


\subsection{Решение 3-мерного волнового уравнения с дельтообразной правой частью}

\[
	\Delta G - \frac{\partial^2 G}{\partial t^2} = \delta(t - t')\delta(\vec{r} - \vec{r}') 
\]
Решение уравнения:
\[
	G = \frac{1}{(2\pi)^{4}} \int \frac{e^{i(\vec{k}\cdot\vec{R} - \omega T)}}{\omega^2 - k^2} dV_k d\omega =  
	\frac{\pi^{-3}}{2^3} 
	\int\limits_{-\infty}^{\infty}\!
	\int\limits_{0}^{\infty}\!
	\int\limits_{0}^{\pi}
	\frac{e^{i(k R \cos \theta - \omega T)}}{\omega^2 - k^2} k^{2} \sin \theta\, d\theta\, dk\, d\omega = (*)
	\]
	\[
	\frac{1}{\omega^2 - k^2} = \frac{1}{2k} \left(\frac{1}{\omega - k} - \frac{1}{\omega + k} \right)
	\]
	\[
	\begin{gathered}
	(*) =
	\frac{i \pi^{-2}}{2^{4}} 
	\int\limits_{0}^{\infty}\!
	\int\limits_{0}^{\pi}
	\left(e^{ - i k T}  - e^{i k T} \right) \sign(-T)
	e^{i(k R \cos \theta)} 
	k \sin \theta\, d\theta\, dk
	= \\ =
	\frac{\pi^{-2}}{2^{4} R} 
	\int\limits_{0}^{\infty}
	\left(e^{ - i k T}  - e^{i k T} \right) \sign(-T)
	\left(e^{-i k R} - e^{i k R} \right)
	\,dk =
	-\frac{1}{4\pi^2 R} 
	\int\limits_{0}^{\infty}
	\sin(kT) \sign(-T)
	\sin(kR)
	\,dk
	= \\ =
	\frac{1}{4\pi^{2} R} 
	\int\limits_{0}^{\infty}
	\sin(k|T|)
	\sin(kR)
	\,dk
	= 
	\frac{1}{8\pi R} 
	\left(\delta(R - |T|)
	-
	\delta(R + |T|) \right)
	\end{gathered} 
\]
Здесь следует отметить тот факт, что данное решение должно удовлетворять нулевым начальным условиям. Что проверяется непосредственно подстановкой значения $T = 0 (t = t')$ в предпоследний интеграл. Условия:
\[
	G \Big|_{t = t'} = 0, \frac{\partial G}{\partial t} \Big|_{t = t'} = 0
\]
На этом фокусы не заканчиваются. Посмотрим на поведение функции в точке с "частицей" $R \to 0$:
\[
	G = - \frac{1}{4\pi} \delta'(|T|)
\]
Что также легко получить, поменяв предел и интеграл местами в предпоследнем интеграле. Представим решение в другом виде:
\[
	\begin{gathered}
	G = 
	\begin{cases}
	\frac{1}{8\pi R} \delta(R - |T|), & R \ne 0, \quad T \ne 0 \\
	- \frac{1}{4\pi} \delta'(|T|), & R = 0, \quad T \ne 0 \\
	0, & T = 0
	\end{cases}
	= 
	\begin{cases}
	\frac{1}{8\pi R} (\delta(R - T) + \delta(R + T)), & R \ne 0, \quad T \ne 0 \\
	- \frac{1}{4\pi} \delta'(|T|), & R = 0, \quad T \ne 0 \\
	0, & T = 0
	\end{cases}
	= \\ =
	\begin{cases}
	\frac{1}{4\pi} \delta(R^2 - T^2), & R \ne 0, \quad T \ne 0 \\
	- \frac{1}{4\pi} \delta'(|T|), & R = 0, \quad T \ne 0 \\
	0, & T = 0
	\end{cases}
	\end{gathered}
\]

\subsection{Решение неоднородного 3-мерного волнового уравнения}

Рассмотрим следующее уравнение:
\[
	\Delta u - \frac{\partial^2 u}{\partial t^2} = s(\vec{r}, t)
\]
По принципу суперпозиции:
\[
	s(\vec{r}, t) = \int\limits_{-\infty}^{\infty} \int\limits_V s(\vec{r}', t') \delta(t - t')\delta(\vec{r} - \vec{r}') dt' dV'	
\]
В результате решение с нулевыми начальными условиями в бесконечном пространстве:
\[
	\begin{gathered}
	u(\vec{r}, t) = 
	\int\limits_{-\infty}^{\infty} \int\limits_V s(\vec{r}', t') G(\vec{r}, \vec{r}', t, t') dV' dt' 
	=
	\int\limits_V \frac{1}{8 \pi R} \int\limits_{-\infty}^{\infty} (\delta(R - (t - t')) + \delta(R + (t - t'))) s(\vec{r}', t')  dt' dV'
	= \\ =
	\int\limits_V \frac{1}{8 \pi R} (s(\vec{r}', t - R) + s(\vec{r}', t + R) dV'
	\end{gathered}
\]

\subsection{О поле произвольно движущейся частицы}

Рассмотрим следующее уравнение:
\[
	\Delta u - \frac{\partial^2 u}{\partial t^2} = \delta(\vec{r} - \vec{a}(t)) = s(\vec{r}, t)
\]
По прежнему будем искать решение при нулевых начальных условиях.
\[
	s(\vec{r}, t) = \int\limits_{-\infty}^{\infty} \int\limits_V s(\vec{r}', t') \delta(t - t')\delta(\vec{r} - \vec{r}') dt' dV' 
\]
\[
	u(\vec{r}, t) = \int\limits_{-\infty}^{\infty} \int\limits_V s(\vec{r}', t') G(\vec{r}, \vec{r}', t, t') dV' dt' 
	= \int\limits_{-\infty}^{\infty} \int\limits_V \delta(\vec{r}' - \vec{a}(t')) G(\vec{r}, \vec{r}', t, t') dV' dt'
\]
\section{Колебания с произвольной возбуждающей силой}

Уравнение колебаний:
\[
	\ddot{x} + \omega_0^2 x = f(t)
\]
\[
	x(0) = x_0, \quad \dot{x}(0) = \dot{x}_0
\]
Решение будем искать в классе непрерывных функций. Рассмотрим вспомогательную задачу:
\[
	\ddot{y} + \omega_0^2 y = \eta(t - t')
\]
\[
	y(0) = 0, \quad \dot{y}(0) = 0 \quad t' > 0
\]
Её решение с непрерывной первой производной, как легко убедиться простой подстановкой:
\[
	y = 
	\begin{cases}
	0 & t<t' \\
	\frac{1 - \cos(\omega_0 (t - t'))}{\omega_0^2} & t \geqslant t'
	\end{cases}
\]
Тогда решением задачи:
\[
	\ddot{z} + \omega_0^2 z = \delta(t - t')
\]
\[
	z(0) = 0, \quad \dot{z}(0) = 0 \quad t' > 0
\]
будет
\[
	z(t, t') = 
	\begin{cases}
	0 & t<t' \\
	\frac{\sin(\omega_0 (t - t'))}{\omega_0} & t \geqslant t'
	\end{cases}
\]
А решение исходной задачи:
\[
	x(t) = x_0 \cos (\omega_0 t) + \frac{\dot{x}_0}{\omega_0} \sin (\omega_0 t) + \int\limits_{0}^{\infty} z(t, t') f(t') dt' = 
	x_0 \cos (\omega_0 t) + \frac{\dot{x}_0}{\omega_0} \sin (\omega_0 t) + \frac{1}{\omega_0} \int\limits_{0}^{t} \sin(\omega_0 (t - t')) f(t') dt'
\]
Есть одно замечание. Можно показать, что в случае задачи относительно $y$  $\dot{y}$ будет непрерывно, если непрерывно само решение? а $\dot{y}$ ограничено сверху на любом конечном промежутке (достаточно домножить на $\dot{y}$ и проинтегрировать в произвольном диапазоне), а для задачи относительно $z$, если решение непрерывно, то первая производная терпит разрыв. В то же время без этих требований построить решения затруднительно, особенно $y$.
\section{Упругое столкновение шаров}

\subsection{Нерелятивистский случай}

Пусть сталкиваются два шара и нормаль $\vec{n}$ в точке столкновения известна, поперечных мгновенных сил нет. Здесь всё просто:
\[
	\begin{aligned}
	& p_{1\perp} = p'_{1\perp} \\
	& p_{2\perp} = p'_{2\perp} \\
	& p_{1n} + p_{2n} = p'_{1n} + p'_{2n} \\
	& \frac{p_{1n}^2}{2m_1} + \frac{p_{2n}^2}{2m_2} = \frac{{p'}_{1n}^2}{2m_1} + \frac{{p'}_{2n}^2}{2m_2}
	\end{aligned} 
\]
\[
	\begin{aligned}
	& p_{1n} - p'_{1n} = p'_{2n} - p_{2n} \\
	& \frac{p_{1n}^2 - {p'}_{1n}^2}{m_1} = \frac{{p'}_{2n}^2 - p_{2n}^2}{m_2}
	\end{aligned}
	\Rightarrow
	\begin{aligned}
	& p_{1n} - p'_{1n} = p'_{2n} - p_{2n} \\
	& (p_{1n} - p'_{1n})\left(\frac{p_{1n} + p'_{1n}}{m_1} - \frac{p_{2n} + p'_{2n}}{m_2} \right) = 0
	\end{aligned}
\]
\[
	\frac{p_{1n} + p'_{1n}}{m_1} - \frac{p_{2n} + p'_{2n}}{m_2} = 0
\]
\[
	\frac{m_2}{m_1} (p_{1n} + p'_{1n}) = p_{2n} + p'_{2n}
\]
\[
	\left(1 + \frac{m_2}{m_1} \right) p'_{1n} + \left(\frac{m_2}{m_1} - 1\right) p_{1n} = 2 p_{2n}
\]
\[
	p'_{1n} = 
	\frac{2 m_1}{m_1 + m_2}  p_{2n} + \frac{m_1 - m_2}{m_1 + m_2} p_{1n} = 
	p_{1n} + \frac{2 (m_1 p_{2n} - m_2 p_{1n})}{m_1 + m_2} 
\]
Получаем два варианта ответов:
\[
	\left\{
	\begin{aligned}
	& p'_{1n} = p_{1n} \\
	& p'_{2n} = p_{2n}
	\end{aligned}
	\right.
	\qquad	
	\left\{
	\begin{aligned}
	& p'_{1n} = p_{1n} - \frac{2 (p_{1n} m_2 - p_{2n} m_1)}{m_1 + m_2} \\
	& p'_{2n} = p_{2n} + \frac{2 (p_{1n} m_2 - p_{2n} m_1)}{m_1 + m_2}
	\end{aligned}
	\right.
\]

\subsection{Релятивистский случай}

Пусть сталкиваются два релятивистских шара и нормаль $\vec{n}$ в точке столкновения известна, поперечных мгновенных сил нет. Тогда законы сохранения:
\[
	\begin{aligned}
	& p_{1\perp} = p'_{1\perp} \\
	& p_{2\perp} = p'_{2\perp} \\
	& p_{1n} + p_{2n} = p'_{1n} + p'_{2n} = 2 p \\
	& E_1 + E_2 = E_1' + E_2' = E
	\end{aligned} 
\]
Задача сводится к системе:
\[
	\begin{cases}
		x + y = 0; \\
		\sqrt{(x + p)^2 + a^2} + \sqrt{(y + p)^2 + b^2} = d
	\end{cases}
	\text{ где }
	\begin{cases}
	& x = p'_{1n} - p \\
	& y = p'_{2n} - p \\
	& a^2 = p_{1\perp}^2 + m_1^2 c^2 \\
	& b^2 = p_{2\perp}^2 + m_2^2 c^2 \\
	& d = E/c
	\end{cases}
\]
Или к одному уравнению:
\[
	\sqrt{(x + p)^2 + a^2} + \sqrt{(p - x)^2 + b^2} = d
\]
\[
	d^2 = 2(x^2 + p^2) + a^2 + b^2 + 2\sqrt{(x + p)^2 + a^2}\sqrt{(p - x)^2 + b^2}
\]
\[
	(d^2 - a^2 - b^2  - 2(x^2 + p^2))^2 = 4((x + p)^2 + a^2)((p - x)^2 + b^2)
\]
\[
	(d^2 - a^2 - b^2)^2  - 4(x^2 + p^2)(d^2 - a^2 - b^2) + 4 (x^2 + p^2)^2 = 4(p^2 - x^2)^2 + 4 a^2(p - x)^2 + 4 b^2(p + x)^2 + 4 a^2 b^2
\]
\[
	(d^2 - a^2 - b^2)^2  - 4(x^2 + p^2)(d^2 - a^2 - b^2) + 16 x^2 p^2 =  4 (a^2 + b^2)(p^2 + x^2) + 8 px (b^2 - a^2) + 4 a^2 b^2
\]
\[
	(d^2 - a^2 - b^2)^2  - 4(x^2 + p^2)d^2 + 16 x^2 p^2 =  8 px (b^2 - a^2) + 4 a^2 b^2
\]
\[
	4 x^2 (4 p^2 - d^2) - 8 px (b^2 - a^2) + (d^2 - a^2 - b^2)^2 - 4 p^2 d^2 - 4 a^2 b^2 = 0
\]
Это обычное квадратное уравнение. Его дискриминант $D$:
\[
	\begin{gathered}
	D/16 = 4 p^2 (b^2 - a^2)^2 - (4 p^2 - d^2) ((d^2 - a^2 - b^2)^2 - 4 p^2 d^2 - 4 a^2 b^2) =
	\\ = 4 p^2 [(b^2 - a^2)^2 - (d^2 - a^2 - b^2)^2 + 4 p^2 d^2 + 4 a^2 b^2] + d^2 ((d^2 - a^2 - b^2)^2 - 4 p^2 d^2 - 4 a^2 b^2) =
	\\ = 4 p^2 [(b^2 + a^2)^2 - (d^2 - a^2 - b^2)^2 + 4 p^2 d^2] + d^2 ((d^2 - a^2 - b^2)^2 - 4 p^2 d^2 - 4 a^2 b^2) =
	\\ = 4 p^2 [4 p^2 d^2 - d^4 + 2 d^2(a^2 + b^2)] + d^2 ((d^2 - a^2 - b^2)^2 - 4 p^2 d^2 - 4 a^2 b^2) =
	\\ = d^2 [16 p^4 + 8 p^2 (a^2 + b^2) + d^4 - 2 d^2(a^2 + b^2) - 8 p^2 d^2 + (a^2 - b^2)^2] =
	\\ = d^2 [(4p^2 - d^2)^2 + 2(4 p^2 - d^2)(a^2 + b^2) + (a^2 + b^2)^2 - 4 a^2 b^2] =
	\\ = d^2 [(4p^2 - d^2 + a^2 + b^2)^2 - 4 a^2 b^2]
	\end{gathered}
\]
\[
	x_{1, 2} = \frac{2 p (b^2 - a^2) \pm d\sqrt{(4p^2 - d^2 + a^2 + b^2)^2 - 4 a^2 b^2}}{2 (4 p^2 - d^2)}
\]
\[
	\begin{gathered}
	4p^2 - d^2 + a^2 + b^2 =
	\\ = (p_{1n} + p_{2n})^2 - \left(\sqrt{p_{1n}^2 + a^2} + \sqrt{p_{2n}^2 + b^2}\right)^2 + a^2 + b^2 =
	\\ = 2 p_{1n} p_{2n} - 2 \sqrt{p_{1n}^2 + a^2} \sqrt{p_{2n}^2 + b^2} = 2 p_{1n} p_{2n} - 2 E_1 E_2/c^2
	\end{gathered}
\]
\[
	\begin{gathered}
	(4p^2 - d^2 + a^2 + b^2)^2 - 4 a^2 b^2 = 
	\\ = 4 [2 p_{1n}^2 p_{2n}^2 - 2 p_{1n} p_{2n} \sqrt{p_{1n}^2 + a^2} \sqrt{p_{2n}^2 + b^2} + p_{1n}^2 b^2 + p_{2n}^2 a^2] =
	\\ = 4 [ p_{1n}^2 E_2^2 + p_{2n}^2 E_1^2 - 2 p_{1n} p_{2n} E_1 E_2] /c^2 =
	\\ = 4 [ p_{1n} E_2 - p_{2n} E_1]^2/c^2
	\end{gathered}
\]
\[
	b^2 - a^2 = (E_2^2 - E_1^2)/c^2 - p_{2n}^2 + p_{1n}^2
\]
\[
	\begin{gathered}
	c^2(2 p (b^2 - a^2) \pm d\sqrt{(4p^2 - d^2 + a^2 + b^2)^2 - 4 a^2 b^2}) =
	(p_{1n} + p_{2n}) ( (E_2^2 - E_1^2) - p_{2n}^2 c^2 + p_{1n}^2 c^2) \pm 2 (E_1 + E_2)[ p_{1n} E_2 - p_{2n} E_1] =
	\\ = p_{1n} E_2^2 - p_{1n} E_1^2 + p_{2n} E_2^2 - p_{2n} E_1^2 + (p_{1n} - p_{2n})(p_{1n} + p_{2n})^2 \pm
	\\ \pm 2 (p_{1n} - p_{2n}) E_1 E_2 \pm  2 p_{1n} E_2^2 \mp 2 p_{2n} E_1^2
	\end{gathered}
\]
Для $+$
\[
	\begin{gathered}
	3 p_{1n} E_2^2 - p_{1n} E_1^2 + p_{2n} E_2^2 - 3 p_{2n} E_1^2 + (p_{1n} - p_{2n})(p_{1n} + p_{2n})^2c^2 + 2 (p_{1n} - p_{2n}) E_1 E_2 =
	\\ = p_{1n} (3 E_2^2 + 2 E_1 E_2 - E_1^2) - p_{2n} (3 E_1^2 + 2 E_1 E_2 - E_2^2) + (p_{1n} - p_{2n})(p_{1n} + p_{2n})^2c^2 =
	\\ = p_{1n} (E_2 + E_1) (3 E_2 - E_1) - p_{2n} (E_1 + E_2) (3 E_1 - E_2) + (p_{1n} - p_{2n})(p_{1n} + p_{2n})^2 =
	\\ = 4 (p_{1n} E_2 - p_{2n} E_1) (E_1 + E_2) + (p_{1n} - p_{2n}) [(p_{1n} + p_{2n})^2c^2 - (E_1 + E_2)^2]
	\end{gathered}
\]
Для $-$
\[
	\begin{gathered}
	- p_{1n} E_2^2 - p_{1n} E_1^2 + p_{2n} E_2^2 + p_{2n} E_1^2 + (p_{1n} - p_{2n})(p_{1n} + p_{2n})^2c^2 - 2 (p_{1n} - p_{2n}) E_1 E_2 =
	\\ = - p_{1n} (E_2^2 + 2 E_1 E_2 + E_1^2) + p_{2n} (E_1^2 + 2 E_1 E_2 + E_2^2) + (p_{1n} - p_{2n})(p_{1n} + p_{2n})^2c^2 =
	\\ = (p_{1n} - p_{2n}) [(p_{1n} + p_{2n})^2c^2 - (E_1 + E_2)^2]
	\end{gathered}
\]
Отсюда следует:
\[
	\left[
	\begin{aligned}
		& x_1 = \frac{p_{1n} - p_{2n}}{2} + \frac{2 (p_{1n} E_2 - p_{2n} E_1) (E_1 + E_2)/c^2}{(p_{1n} + p_{2n})^2 - (E_1 + E_2)^2/c^2} \\
		& x_2 = \frac{p_{1n} - p_{2n}}{2}
	\end{aligned}
	\right.
\]
Таким образом существуют два решения задачи:
\[
	\left\{
	\begin{aligned}
	& p'_{1n} = p_{1n} \\
	& p'_{2n} = p_{2n}
	\end{aligned}
	\right.
	\qquad	
	\left\{
	\begin{aligned}
	& p'_{1n} = p_{1n} - \frac{2 (p_{1n} E_2 - p_{2n} E_1) (E_1 + E_2)/c^2}{(E_1 + E_2)^2/c^2 - (p_{1n} + p_{2n})^2} \\
	& p'_{2n} = p_{2n} + \frac{2 (p_{1n} E_2 - p_{2n} E_1) (E_1 + E_2)/c^2}{(E_1 + E_2)^2/c^2 - (p_{1n} + p_{2n})^2}
	\end{aligned}
	\right.
\]

\subsection{Условия столкновения}

Допустим, что 2 шара радиусов $R_1$ и $R_2$ движутся прямолинейно и равномерно со скоростями $\vec{v}_1$ и $\vec{v}_2$, координаты центров в момент времени $t_p$ равны $\vec{r}_1$ и $\vec{r}_2$ соответственно. В момент столкновения:
\[
	|\vec{r}_1  - \vec{r}_2 + (\vec{v}_1 - \vec{v}_2) (t - t_p)| = R_1 + R_2
\] 
Обозначим $\vec{r}_{12} = \vec{r}_1  - \vec{r}_2$, $\vec{v}_{12} = \vec{v}_1 - \vec{v}_2$, $\tau = t - t_p$, $R = R_1 + R_2$.
\[
	r_{12}^2 + v_{12}^2 \tau^2 + 2 \tau (\vec{v}_{12}\cdot\vec{r}_{12}) = R^2
\]
\[
	r_{12}^2 + v_{12}^2 \tau^2 + 2 \tau v_{12} r_{12} \cos \theta  = R^2
\]
\[
	\begin{gathered}
	\tau_{1,2} = - \frac{r_{12} \cos \theta \mp \sqrt{r_{12}^2 \cos^2 \theta - r_{12}^2  + R^2}}{v_{12}} =
	\\ = - \frac{r_{12} \cos \theta \mp \sqrt{R^2 - r_{12}^2 \sin^2 \theta}}{v_{12}} =
	\\ = r_{12} \frac{ - \cos \theta \pm \sqrt{R^2/r_{12}^2 - \sin^2 \theta}}{v_{12}}
	\end{gathered} 
\]
Но $\tau > 0$, $R < r_{12}$, если выбрать знак +:
\[
	\sqrt{R^2/r_{12}^2 - \sin^2 \theta} > \cos \theta
\]
Если $\cos \theta > 0$:
\[
	R^2/r_{12}^2 > 1
\]
Поэтому $\cos \theta < 0$. Если выбрать знак - также $\cos \theta < 0$:
\[
	- \cos \theta > \sqrt{R^2/r_{12}^2 - \sin^2 \theta}
\]
\[
	1 > R^2/r_{12}^2
\]
В результате получилось два момента столкновения, но только один из них соответствует действительности: тот который был раньше. Как следует из выше сказанного: 
\[
	\tau = \frac{ - \cos \theta - \sqrt{R^2/r_{12}^2 - \sin^2 \theta}}{v_{12}} r_{12} 
\]
Необходимые и достаточные условия столкновения:
\[
	\cos \theta < 0, \quad 1 > R^2/r_{12}^2 > \sin^2 \theta 
\]
Нормаль столкновения:
\[
	\vec{n} = \frac{\vec{r}_{12} + \vec{v}_{12} \tau}{|\vec{r}_{12} + \vec{v}_{12} \tau|}
\]
\section{О металлическом водороде}

Что такое металл? Кто-то наверное думает, что металлы это вещества из нижнего треугольника периодической системы элементов Менделеева. Это правда, но при условии, что температура в районе 273 К, давление в районе 1 атм, и мы исключили все полупроводники. Металл это прежде всего хороший проводник. Но что такое хороший проводник? С точки зрения классической теории это вещество с большой проводимостью. Есть чёткие границы, но они, как это ясно, размываются при изменении температуры и давления. С точки зрения квантовой теории это вещества, в которых есть зона проводимости. Мне трудно охарактеризовать её словами, поэтому вот рисунок:
\begin{center}
	\includegraphics[width = 0.6\textwidth]{images/tikz/for_H_metall}
\end{center}
\section{О кельтском камне в форме эллипсоида}

Кельтские камни представляют собой тела, тензор инерции которых в осях симметрии системы имеет ненулевые недиагональные компоненты. Простейшим примером кельтского камня будет выступать эллипсоид вращения плотность которого постоянна со вставками двух точечных масс не изменяющих положение центра масс, но тем не менее вносящих вклад в недиагональные компоненты.

Легко убедиться (от 10 минут до получаса, если уметь брать кратные интегралы), что момент инерции однородного эллипсоида вращения с полуосями $a$, $b$, $c$ в системе в которой оси совпадают с осями симметрии системы, 0 в центре симметрии, имеет вид:
\[
	\hat{J} = 
	\begin{pmatrix}
	\frac{1}{5}m(b^2 + c^2) & 0 & 0 \\
	0 & \frac{1}{5}m(a^2 + c^2) & 0 \\
	0 & 0 & \frac{1}{5}m(a^2 + b^2) 
	\end{pmatrix}
\]
С массой $M$ добавленной в точки $(x_a, y_a, 0)$,  $(-x_a, -y_a, 0)$ мы получим следующий тензор инерции:
\[
	\hat{J} = 
	\begin{pmatrix}
	\frac{1}{5}m(b^2 + c^2) + M y_a^2 & M x_a y_a & 0 \\
	M x_a y_a & \frac{1}{5}m(a^2 + c^2) + M x_a^2 & 0 \\
	0 & 0 & \frac{1}{5}m(a^2 + b^2) 
	\end{pmatrix}
\]
В общем случае тензор инерции будет иметь вид:
\[
	\hat{J} = 
	\begin{pmatrix}
	J_{xx} & J_{xy} & J_{xz} \\
	J_{yx} & J_{yy} & J_{yz} \\
	J_{zx} & J_{zy} & J_{zz} 
	\end{pmatrix}
\]
Тело при этом пусть будет эллиптической формы, центр масс на оси симметрии системы. Тело лежит на подложке, которая действует на него моментом сил трения $\vec{M}_{fr}$ в направлении перпендикулярном подложке и противоположно компоненте угловой скорости в данном направлении. Неподвижные оси $x', y', z'$ направим как показано на рисунке. Подвижные оси будут представлять собой оси симметрии эллипсоида $x, y, z$. В подвижной системе жёстко связанной с телом на систему будут действовать силы трения (будем считать, что они приводят только к наличию момента $\vec{M}_{fr}$ в направлении перпендикулярном подложке и противоположно проекции угловой скорости на данное направление), сила реакции опоры, направленная перпендикулярно подложке $\vec{N}$, сила инерции, так как центр масс движется вообще говоря ускоренно (но она постоянна по величине и не создаёт относительно центра масс момента, в чём также легко убедиться с помощью интегрирования), центробежная сила (приводит к знаменитым эйлеровым уравнениям -- момент равен $-\vec{\omega}\times\vec{L}$), а вот сила Кориолиса равна 0, так как скорость движения точек тела в подвижной системе координат связанной с телом равна 0. Уравнения динамики вращательного движения:
\[
	\hat{J} \frac{d\vec{\omega}}{dt} + \vec{\omega}\times(\hat{J}\vec{\omega}) = \vec{N}\times\vec{R} + \vec{M}_{fr}
\]
Введём эйлеровы углы $(\psi, \theta, \phi)$. Для полного счастья нам понадобятся орты подвижной системы координат, выраженные через неподвижные. С помощью вспомогательного единичного вектора $\vec{n}$, получим:
\[
	\begin{aligned}
	& \vec{n} = \cos \psi \vec{e}_x + \sin \psi \vec{e}_y \\
	& \vec{e}'_z = \vec{e}_z \cos \theta  + \vec{n}\times\vec{e}_z \sin \theta \\
	& \vec{e}'_x = \vec{n} \cos \phi  + \vec{e}_z\times\vec{n} \sin \phi \\
	& \vec{e}'_y = \vec{e}'_z \times \vec{e}'_x = - \vec{n} \sin \phi + \vec{e}_z\times\vec{n} \cos \phi
	\end{aligned}
\]
Или в страшной форме, после долгих и утомительных преобразований:
\[
	\begin{aligned}
	& \vec{e}_x =
	[\cos \phi \cos \psi - \sin \phi \sin \psi \cos \theta] \vec{e}'_x +
	[\cos \phi \sin \psi + \sin \phi \cos \psi \cos \theta] \vec{e}'_y +
	\sin \theta \sin \phi \vec{e}_z \\
	& \vec{e}_y =
	[- \sin \phi \cos \psi - \cos \phi \sin \psi \cos \theta] \vec{e}'_x +
	[- \sin \phi \sin \psi + \cos \phi \cos \psi \cos \theta] \vec{e}'_y +
	\sin \theta \cos \phi \vec{e}_z \\
	& \vec{e}_z =
	\sin \psi \sin \theta \vec{e}'_x -
	\sin \theta \cos \psi \vec{e}'_y +
	\cos \theta \vec{e}_z \\
	\end{aligned}
\]
Отсюда просто найти и обратное преобразование, но нам потребуется всего один вектор:
\[
	\vec{e}'_z =
	\sin \theta \sin \phi \vec{e}_x -
	\sin \theta \cos \phi \vec{e}_y +
	\cos \theta \vec{e}_z
\]
Зачем он нужен? Во-первых вдоль него направлен момент сил трения, а во-вторых с его помощью можно найти нормаль к эллипсоиду, который повернули на произвольные эйлеровы углы в точке соприкосновения с подложкой и как следствие найти эту точку.
Нормаль в точке эллипсоида $(x_a, y_a, z_a)$ (ещё 10-30 минут утомительных вычислений) коллинеарна вектору:
\[
	\left\{
		\frac{x_a}{a^2},  \frac{y_a}{b^2}, \frac{z_a}{c^2}
	\right\}
\]
Поэтому:
\[
	\begin{aligned}
	& A\sin \theta \sin \phi  = \frac{x_a}{a^2} \\
	& -A\sin \theta \cos \phi  = \frac{y_a}{b^2} \\
	& A\cos \theta  = \frac{z_a}{c^2} \\
	\end{aligned}
\]
Подставляем в уравнение эллипса:
\[
	A^2[a^2\sin^2 \theta \sin^2 \phi + b^2\sin^2 \theta \cos^2 \phi + c^2\cos^2 \theta] = 1
\]
\[
	A = \pm \frac{1}{\sqrt{a^2\sin^2 \theta \sin^2 \phi + b^2\sin^2 \theta \cos^2 \phi + c^2\cos^2 \theta}}
\]
Знак на самом деле вполне определён, но в данный момент мне придётся долго вспоминать чему он равен. Так, так, так... Кажется $z_a$ по картинке у нас отрицательное, значит и знак отрицательный. Теперь можно чётко записать, что такое $\vec{R}$:
\[
	\vec{R} = x_a \vec{e}_x + y_a \vec{e}_y + z_a \vec{e}_z
\]
Тем самым динамические уравнения полностью определены. Кинематические уравнения я подсмотрел в одном из курсов теоретической механики:
\[
	\begin{aligned}
	& \omega_x = \dot{\psi} \sin \theta \sin \phi + \dot{\theta} \cos \phi \\
	& \omega_y = \dot{\psi} \sin \theta \cos \phi - \dot{\theta} \sin \phi \\
	& \omega_z = \dot{\psi} \cos \theta + \dot{\phi}
	\end{aligned}
\]
В таком виде для нас они бесполезны, однако вполне сойдут в виде:
\[
	\begin{aligned}
	& \dot{\psi} = \frac{\omega_x \sin \phi + \omega_y \cos \phi}{\sin \theta} \\
	& \dot{\theta} = \omega_x \cos \phi - \omega_y \sin \phi \\
	& \dot{\phi} = \omega_z - \frac{\omega_x \sin \phi + \omega_y \cos \phi}{\sin \theta} \cos \theta \\
	\end{aligned}
\]
Начальные условия для системы:
\[
	\omega_x(0) = \omega_y(0) = 0
	\quad
	\omega_z(0) = \Omega 
	\quad
	\theta(0) = \phi(0) = \psi(0) = 0 
\]
Но легко видеть, что этот случай неинтересен и ни к чему особенному приводить не должен. Более интересный случай связан с небольшими отклонениями от 0 в начальных условиях. В принципе осталось разобраться с силой реакции опоры. Из-за ускоренного движения центра масс эллипсоида она может принимать, вообще говоря значения отличные от $mg$, однако если по сравнению с $g$ добавкой можно пренебречь, то осталось перейти к моделированию. Но мне лень.
\input{all/19.tex}
\section{Энергия, передаваемая от молекулы к молекуле в одноатомном идеальном газе}

Вопрос интересен тем, что позволяет оценить насколько велика по отношению к энергии молекулы энергия передаваемая другой молекуле. Будем считать, что молекула при соударении просто переходит из одной группы молекул с энергией в данном диапазоне в другую группу с энергией в другом диапазоне. Передаваемая в этом процессе энергия будет характеризоваться величиной:
\[
	\frac{p_1'^2}{2M} - \frac{p_1^2}{2M}
\] 
Функция распределения в равновесном состоянии для идеального газа:
\[
	f(p_1, p_1') = 
	A \exp \left(
	- \frac{p_1^2 + p_1'^2}{2MkT}
	\right)	p_1^2 p_1'^2
\]
Найдём нормировочную константу:
\[
	\int\limits_0^\infty \! \int\limits_0^\infty 
	A \exp \left(
	- \frac{p_1^2 + p_1'^2}{2MkT}
	\right)	p_1^2 p_1'^2 dp_1\,dp_1'
\]
\[
	\int\limits_0^\infty e^{- \alpha p^2} p^2 dp = 
	- \frac{\partial}{\partial \alpha} \int\limits_0^\infty e^{- \alpha p^2} dp
	= - \frac{1}{2} \frac{\partial}{\partial \alpha} \sqrt{\frac{\pi}{\alpha}}
	= \frac{\sqrt{\pi}}{4\alpha^{3/2}}
\]
\[
	\int\limits_0^\infty \! \int\limits_0^\infty 
	A \exp \left(
	- \frac{p_1^2 + p_1'^2}{2MkT}
	\right)	p_1^2 p_1'^2 dp_1\,dp_1'
	=
	A \frac{8M^3k^3T^3\pi}{16} = 1
\]
\[
	A = \frac{2}{M^3k^3T^3\pi}
\]
Очевидно, что среднее значение рассматриваемой характеристики равно 0. Найдём дисперсию:
\[
	\int\limits_0^\infty \! \int\limits_0^\infty 
	A \left(\frac{p_1'^2}{2M} - \frac{p_1^2}{2M}\right)^2\exp \left(
	- \frac{p_1^2 + p_1'^2}{2MkT}
	\right)	p_1^2 p_1'^2 dp_1\,dp_1' = (*)
\]
Найдём вспомогательные интегралы:
\[
	\int\limits_0^\infty e^{- \alpha p^2} p^4 dp = 
	\frac{1}{2} \frac{\partial^2}{\partial \alpha^2} \sqrt{\frac{\pi}{\alpha}} =
	\frac{3\sqrt{\pi}}{8\alpha^{5/2}}
\]
\[
	\int\limits_0^\infty e^{- \alpha p^2} p^6 dp = 
	\frac{1}{2} \frac{\partial^3}{\partial \alpha^3} \sqrt{\frac{\pi}{\alpha}} =
	\frac{15\sqrt{\pi}}{16\alpha^{7/2}}
\]
\[
	(*) = \frac{A}{4M^2} \frac{30\sqrt{\pi}}{16\alpha^{7/2}} \frac{\sqrt{\pi}}{4\alpha^{3/2}} -
	2\frac{A}{4M^2} \frac{3\sqrt{\pi}}{8\alpha^{5/2}}\frac{3\sqrt{\pi}}{8\alpha^{5/2}}
	=
	\frac{A}{4M^2}  \frac{(15 - 9)\pi}{32\alpha^{5}} = 
	\frac{16\alpha^{3}}{4M^2\pi} \frac{3\pi}{16\alpha^{5}} =
	\frac{3}{4M^2\alpha^{2}} = 
	\frac{3M^2k^2T^2}{M^2} = 3 k^2 T^2
\]
Стандартное отклонение:
\[
	\sigma = \sqrt{3} kT
\]
Данная величина превосходит среднее значение энергии отдельной молекулы одноатомного идеального газа! Где я всех обманул? Всё просто, нужно ещё знать вероятность столкновений молекул.
\section{Общая форма уравнений электродинамики}

Пока пространство-время обычное -- плоское. Интервал:
\[
	ds^2 = dx^i dx_i
\]

Будем исходить из того, что электромагнитное поле задаётся четырёхпотенциалом $A_i$. Силовая характеристика тензорная:
\[
	F_{ij} = \frac{\partial A_j}{\partial x^i} -  \frac{\partial A_i}{\partial x^j}
\]
Отсюда следует одно уравнение электродинамики:
\[
	\frac{\partial F_{ij}}{\partial x^k} + \frac{\partial F_{jk}}{\partial x^i} + \frac{\partial F_{ki}}{\partial x^j} = 0
\]
Или оно же, но через дуальный тензор:
\[
	\frac{\partial \tilde{F}^{ij}}{\partial x^j} = 0 
\]
А также то, что тензор $F_{ij}$ антисимметричный:
\[
	F_{ij} = - F_{ji}
\] 
Существуют четыре инварианта, которыми определяются уравнения электродинамики:
\[
	S_1 = F^{ij}F_{ij} \text{ характеризует поле}
\]
\[
	S_2 = \tilde{F}^{ij}F_{ij} \text{ характеризует поле}
\]
\[
	S_3 = j^i A_i \text{ характеризует взаимодействие поля с источниками}
\]
\[
	S_4 = A^i A_i \text{ ещё одна характеристика поля, в квантовой теории позволяет учесть массу переносчика взаимодействия}
\]
Лагранжева плотность:
\[
	L = L(S_1, S_2, S_3, S_4)
\]
Варьируем поле $A_i$ (далее везде отброшены слагаемые типа дивергенций, по которым можно проинтегрировать вариацию лагранжевой плотности):
\[
	\delta L = 
	\frac{\partial L}{\partial S_1} \delta S_1 + 
	\frac{\partial L}{\partial S_2} \delta S_2 + 
	\frac{\partial L}{\partial S_3} \delta S_3 + 
	\frac{\partial L}{\partial S_4} \delta S_4
\]
\begin{enumerate}
	\item 
	\[
		\begin{gathered}
		\frac{\partial L}{\partial S_1} \delta S_1 = 2 F^{ij} \frac{\partial L}{\partial S_1} \delta F_{ij} = 
		2 F^{ij} \frac{\partial L}{\partial S_1} \left(\frac{\partial \delta A_j}{\partial x^i} - \frac{\partial \delta A_i}{\partial x^j} \right)
		= \\ =
		[\text{раскрываем и меняем местами во втором слагаемом индексы $i, j$} ] = \\ =
		4 F^{ij} \frac{\partial L}{\partial S_1} \frac{\partial \delta A_j}{\partial x^i} = 
		4 \frac{\partial }{\partial x^i} \left(F^{ij} \frac{\partial L}{\partial S_1} \delta A_j \right) - 4 \frac{\partial}{\partial x^i} \left(F^{ij} \frac{\partial L}{\partial S_1} \right) \delta A_j =
		4 \frac{\partial}{\partial x^j} \left(F^{ij} \frac{\partial L}{\partial S_1} \right) \delta A_i
		\end{gathered}
	\]
	\item 
	\[
		\begin{gathered}
		\frac{\partial L}{\partial S_2} \delta S_2 = \frac{\partial L}{\partial S_2} \frac{1}{2}\left(\epsilon^{ijkl} F_{ij} \delta F_{kl} +  \epsilon^{ijkl} F_{kl} \delta F_{ij} \right)
		= \frac{\partial L}{\partial S_2} \frac{1}{2}\left(\epsilon^{klij} F_{kl} \delta F_{ij} +  \epsilon^{ijkl} F_{kl} \delta F_{ij} \right)
		= \frac{\partial L}{\partial S_2} \frac{1}{2}(\epsilon^{klij} + \epsilon^{ijkl}) F_{kl} \delta F_{ij} = 
		\\ =
		[\text{так как перестановка чётная}] 
		= 2 \frac{1}{2} \epsilon^{ijkl} F_{kl} \frac{\partial L}{\partial S_2} \delta F_{ij}
		= 2 \tilde{F}^{ij} \frac{\partial L}{\partial S_2} \left(\frac{\partial \delta A_j}{\partial x^i} - \frac{\partial \delta A_i}{\partial x^i} \right)
		= [\text{также как и выше}] = \\
		= 4 \frac{\partial}{\partial x^j} \left(\tilde{F}^{ij} \frac{\partial L}{\partial S_2} \right) \delta A_i
		\end{gathered}
	\]
	\item
	\[
		\frac{\partial L}{\partial S_3} \delta S_3 = j^i \frac{\partial L}{\partial S_3} \delta A_i
	\]
	\item 
	\[
		\frac{\partial L}{\partial S_4} \delta S_4 = 2 A^i \frac{\partial L}{\partial S_4}  \delta A_i
	\]
\end{enumerate}
Получаем:
\[
	\delta L = 
	\left(
	4 \frac{\partial}{\partial x^j} \left(F^{ij} \frac{\partial L}{\partial S_1} \right) + 
	4 \frac{\partial}{\partial x^j} \left(\tilde{F}^{ij} \frac{\partial L}{\partial S_1} \right) +
	\frac{\partial L}{\partial S_3} j^i + 
	\frac{\partial L}{\partial S_4} 2 A^i
	\right) \delta A_i
\]
Вспоминая выражение для действия и равенство нулю вариации действия получаем:
\[
	4 \frac{\partial}{\partial x^j} \left(F^{ij} \frac{\partial L}{\partial S_1} \right) + 
	4 \frac{\partial}{\partial x^j} \left(\tilde{F}^{ij} \frac{\partial L}{\partial S_2} \right) +
	\frac{\partial L}{\partial S_3} j^i + 
	\frac{\partial L}{\partial S_4} 2 A^i = 0
\]
Для электродинамики Максвелла:
\[
	L = - \frac{1}{4\mu_0} S_1 - S_3
\]
Следующее выражение имеет какое-то отношение к Шрёдингеру ($m$ -- масса фотона):
\[
	L = - \frac{1}{4\mu_0} S_1 - S_3 - \frac{m^2 c^2}{2 \hbar^2 \mu_0} S_4
\]
Лагранжева плотность Борна:
\[
	L = \frac{1}{\mu_0 \lambda^2} \left( 1 - \sqrt{1 + \frac{\lambda^2}{2}  S_1} \right) - S_3
\]
Лагранжева плотность Борна-Инфельда:
\[
	L = \frac{1}{\mu_0 \lambda^2} \left( 1 - \sqrt{1 + \frac{\lambda^2}{2}  S_1 - \frac{\lambda^4}{16}  S_2^2} \right) - S_3
\]
Для нелинейной теории Шрёдингера:
\[
	L = \frac{1}{2 \mu_0 \lambda^2} \ln \left(1 - \frac{\lambda^2}{2} S_1\right) - S_3
\]
\section{Решение уравнения Лапласа}

Рассмотрим задачу:
\[
	\Delta u = s(\vec{r})
\]

Граничные условия пока не приводятся, так как теорема единственности неизвестна.

\subsection{Теорема единственности}

Пусть $u$ и $v$ удовлетворяют уравнениям:
\[
	\Delta u = s(\vec{r}) \quad \Delta v = s(\vec{r})
\]
Рассмотрим интеграл:
\[
	\int\limits_V (u \Delta v - v \Delta u) dV = \int\limits_V (u - v) s(\vec{r}) dV
\]
\[
	u \Delta v = u \nabla \cdot \nabla v = \nabla \cdot (u \nabla v) - (\nabla u) \cdot (\nabla v)
\]
\[
	u \Delta v - v \Delta u = \div (u \nabla v - v \nabla u)
\]
\[
	\int\limits_V \div (u \nabla v - v \nabla u) dV = 
	\oint\limits_S \left(u \frac{\partial v}{\partial n} - v \frac{\partial u}{\partial n}\right) dS =
	\int\limits_V (u - v) s(\vec{r}) dV
\]
Следовательно для единственности решения достаточно задать на некоторой поверхности или совокупности поверхностей:
\[
	u = f(\vec{r}) \quad \frac{\partial u}{\partial n} = g(\vec{r}) \text{ на } S
\]
В самом деле в этом случае из интеграла следует:
\[
	\int\limits_V (u - v) s(\vec{r}) dV
\]
И из произвольности функции $s$ получаем:
\[
	u = v
\]

\subsection{Решение $N$-мерного уравнения Лапласа с $\delta$-образной правой частью}

\[
	\Delta G = \delta (\vec{r} - \vec{r}')
\]

В интегральной форме:
\[
	\oint\limits_S \grad G \cdot d\vec{S} = \int\limits_V \delta (\vec{r} - \vec{r}') dV
\]

Задача обладает сферической симметрией и относительно $\vec{R} = \vec{r} - \vec{r}'$, когда $S$ это сфера радиуса $R$:
\[
	S_N \frac{\partial G}{\partial R} = 1
\]
\[
	\frac{\partial G}{\partial R} = \frac{\Gamma(N/2)}{2 \pi^{N/2}} \frac{1}{R^{N - 1}}
\]
\[
	G = 
	\begin{cases}
	- \frac{\Gamma(N/2)}{2(N - 2) \pi^{N/2}} \frac{1}{R^{N - 2}} + C & N \ne 2 \\
	\frac{1}{2\pi} \ln R + C & N = 2
	\end{cases}
\]
При $N > 2$ будем считать $C = 0$.

\subsection{Общее решение уравнения Лапласа}

Уравнение:
\[
	\Delta u = s(\vec{r})
\]
При граничных условиях:
\[
	u = f(\vec{r}) \quad \frac{\partial u}{\partial n} = g(\vec{r}) \text{ на } S
\]
Рассмотрим интеграл:
\[
	\int\limits_V (u \Delta G - G \Delta u) dV = 
	\oint\limits_S \left(u \frac{\partial G}{\partial n} - G \frac{\partial u}{\partial n}\right) dS
\]
С учётом:
\[
	\Delta G(\vec{r}, \vec{r}') = \delta (\vec{r} - \vec{r}')
\]
Получаем:
\[
	u(\vec{r}') = \int\limits_V G(\vec{r}, \vec{r}') s(\vec{r}) dV + \oint\limits_S \left(f(\vec{r}) \frac{\partial G(\vec{r}, \vec{r}')}{\partial n} - G(\vec{r}, \vec{r}') g(\vec{r}) \right) dS
\]
\[
	\begin{aligned}
	&
	\int\limits_V G(\vec{r}, \vec{r}') s(\vec{r}) dV
	\text{ -- объёмный потенциал}
	\\
	& - \oint\limits_S G(\vec{r}, \vec{r}') g(\vec{r}) dS 
	\text{ -- потенциал простого слоя}
	\\
	& \oint\limits_S f(\vec{r}) \frac{\partial G(\vec{r}, \vec{r}')}{\partial n} dS
	\text{ -- потенциал двойного слоя}
	\end{aligned}
\]


\section{Функции Матьё}

%\subsection{Уравнение Матьё и преобразование Фурье}
%
%\[
%	y'' + (a - 2 q \cos (2x)) y = 0
%\]
%Выполним преобразование:
%\[
%	x = \omega x'
%\]
%Уравнение перейдёт в:
%\[
%	y'' + \omega^2 ( a - 2 q \cos (2\omega^2 x')) y = 0
%\]
%Будем искать решение в виде:
%\[
%	y(x') = \int\limits_{-\infty}^{\infty} Y(k) e^{ikx'} dk
%\]
%В результате получаем относительно $Y(k)$:
%\[
%	\int\limits_{-\infty}^{\infty} \left(- k^2 \omega^2 Y(k) e^{ikx'} + a Y(k) e^{ikx'} - 2 q \frac{e^{i2\omega x'} + e^{-i2\omega x'}}{2} Y(k) e^{ikx'} \right)  dk = (*)
%\]
%\[
%	\int\limits_{-\infty}^{\infty} Y(k) e^{i(k + 2\omega) x'} dk = \int\limits_{-\infty}^{\infty} Y(k - 2\omega) e^{ik x'} dk
%\]
%\[
%	\int\limits_{-\infty}^{\infty} Y(k) e^{i(k - 2\omega) x'} dk = \int\limits_{-\infty}^{\infty} Y(k + 2\omega) e^{ik x'} dk
%\]
%\[
%	(*) = \int\limits_{-\infty}^{\infty} \left((\omega^2 a - k^2) Y(k) e^{ikx'}  - \omega^2 q Y(k - 2 \omega) - \omega^2 q Y(k + 2 \omega) \right) e^{ikx'}  dk
%\]
%Убираем интеграл:
%\[
%	Y(k - 2\omega) + Y(k + 2\omega) = \frac{\omega^2 a - k^2}{\omega^2 q} Y(k)
%\]
%\[
%	k \to 0 \quad Y(- 2\omega) + Y(2\omega) = \frac{a}{q} Y(0)
%\]
%\[
%	\Rightarrow Y(k) = \frac{a}{2q} Y(0) + f(k)
%	\quad f(k) \text{ -- нечётная функция}
%\]
%\[
%	f(k - 2 \omega) + f(k + 2 \omega) = \frac{a^2}{2 q^2} Y(0) - \frac{a k^2}{2 \omega^2 q^2} Y(0) + \frac{\omega^2 a - k^2}{\omega^2 q} f(k)
%\]
%\[
%	\Rightarrow Y(0) = 0
%\]
%Теперь немного о странностях. Отсюда следует, что:
%\[
%	Y(4\omega) = \frac{a - 4}{q} Y(2\omega)
%\]
%Но иных решений кроме $Y(k) = 0$ мне найти так и не удалось. Единственное решение данной проблемы, которое приходит на ум, заключается в том, что преобразование Фурье в данном случае делать нельзя.

\subsection{Уравнение Матьё и его решение}

\[
	y'' + (a - 2 q \cos (2x)) y = 0
\]
Согласно теореме Флоке, его решение можно искать в виде:
\[
	e^{ikx} u(x)
\]
$k$ -- параметр, который пока не определён.
Где $u(x)$ -- периодическая функция с периодом коэффициента, в данном случае $\pi$.
%\[
%	u(x) = a_0 + \sum\limits_{n = 1}^\infty a_n \cos (2 n x) + b_n \sin (2 n x)
%\]
%Подставляем:
%\[
%	u'' + 2 ik u' + (a - k^2 - 2 q \cos (2x)) u = 0
%\]
%\[
%	\begin{gathered}
%	- \sum\limits_{n = 1}^\infty 4 n^2 a_n \cos (2 n x) + 4 n^2 b_n \sin (2 n x) + 2 i k \sum\limits_{n = 1}^\infty - 2 n a_n \sin (2 n x) + 2 n b_n \cos (2 n x) + 
%	\\ + (a - k^2 - 2 q \cos (2x)) a_0 +
%	(a - k^2) \sum\limits_{n = 1}^\infty a_n \cos (2 n x) + b_n \sin (2 n x) - 
%	\\ -
%	2 q \sum\limits_{n = 1}^\infty a_n (\cos (2 (n + 1) x) + \cos (2 (n - 1) x))  + b_n (\sin (2 (n + 1) x) + \sin (2 (n - 1) x))	
%	\end{gathered}
%\]
\[
	u(x) = \sum\limits_{n = -\infty}^\infty a_n e^{2nxi}
\]
Подставляем:
\[
	\begin{gathered}
	\sum\limits_{n = -\infty}^\infty a_n [a - (2n + k)^2 ] e^{(2n + k)xi} - 
	q a_n \left[e^{(2n + 2 + k)xi} + e^{(2n - 2 + k)xi}\right] =
	\\ =
	\sum\limits_{n = -\infty}^\infty a_n [a - (2n + k)^2 ] e^{(2n + k)xi} - 
	q (a_{n + 1} + a_{n - 1}) e^{(2n + k)xi}
	\end{gathered}
\]
Получаем систему:
\[
	a_n [a - (2n + k)^2 ] - q (a_{n + 1} + a_{n - 1})  = 0
\]

\section{О проводниках}

Классическая теория проводимости основана на уравнении движения электрона, записанном в следующем виде:
\[
	\frac{d\vec{p}}{dt} = - \nu \vec{p} + q\vec{E} + q\vec{v} \times \vec{B}
\]
Трудный вопрос будет ли работать это уравнение в релятивистском случае, поэтому оставим его пока в покое и рассмотрим нерелятивистский случай (тем более, что трудно представить себе среду с релятивистскими электронами и кристаллической формой). Поэтому:
\[
	\vec{p} = m \vec{v}
\]
\[
	\vec{j} = q n \vec{v}
\]
Для классического проводника:
\[
	\frac{d\vec{p}}{dt} = 0
\]
\[
	- \nu m \vec{v} + q \vec{E} + q\vec{v} \times \vec{B} = 0
\]
\[
	\nu m \vec{v} + q \vec{B} \times \vec{v} = q \vec{E} 
\]
Для простоты выкладок обозначим:
\[
	\lambda = \frac{\nu m}{q}
\]
\[
	\begin{pmatrix}
	\lambda & B_z & -B_y \\
	-B_z & \lambda & B_x \\
	B_y & -B_x & \lambda
	\end{pmatrix}
	\begin{pmatrix}
	v_x \\
	v_y \\
	v_z
	\end{pmatrix}
	= 
	\begin{pmatrix}
	E_x \\
	E_y \\
	E_z
	\end{pmatrix}
\]
Добавим ещё немного обозначений: $\hat{I}$ -- единичная матрица,
\[
	\hat{B} = 
	\begin{pmatrix}
	0 & B_z & -B_y \\
	-B_z & 0 & B_x \\
	B_y & -B_x & 0
	\end{pmatrix}
	\quad
	\hat{B}^2 = 
	\begin{pmatrix}
	- B_z^2 - B_y^2	& B_x B_y 			& B_x B_z \\
	B_x B_y 		& - B_z^2 - B_x^2 	& B_y B_z \\
	B_x B_z 		& B_y B_z  			& - B_x^2 - B_y^2
	\end{pmatrix}
	\quad
	B^2 = B_x^2 + B_y^2 + B_z^2
\]
Уравнение:
\[
	(\lambda \hat{I} + \hat{B}) \vec{v} = \vec{E}
\]
\[
	(\lambda \hat{I} + \hat{B})^{-1} = \frac{1}{\lambda(\lambda^2 + B^2)} 
	\begin{pmatrix}
	B_x^2 + \lambda^2   	 & B_x B_y - \lambda B_z 	& B_x B_z + \lambda B_y 	\\
	B_x B_y + \lambda B_z	 & B_y^2 + \lambda^2    	& B_y B_z - \lambda B_x 	\\
	B_x B_z - \lambda B_y	 & B_y B_z + \lambda B_x    & B_z^2 + \lambda^2		 	\\
	\end{pmatrix} =
	\frac{1}{\lambda(\lambda^2 + B^2)}  ((\lambda^2 + B^2) \hat{I} + \hat{B}^2 - \lambda \hat{B})
\]
\[
	\vec{v} = \frac{\vec{E}}{\lambda} + \frac{1}{\lambda(\lambda^2 + B^2)} (\hat{B}^2 - \lambda \hat{B}) \vec{E}
	=
	\frac{\vec{E}}{\lambda} + \frac{1}{\lambda(\lambda^2 + B^2)} (\vec{B}\times(\vec{B} \times \vec{E}) + \lambda \vec{E} \times \vec{B})
	= 
	\frac{\lambda\vec{E}}{\lambda^2 + B^2} + \frac{1}{\lambda(\lambda^2 + B^2)} (\vec{B} (\vec{E} \cdot \vec{B} ) + \lambda \vec{E} \times \vec{B})
\]
Отсюда следует один интересный вывод, если изложенный механизм действительно имеет место, то проводимость вдоль направления $\vec{E}$, когда $\vec{E} \perp \vec{B}$:
\[
	\sigma = \frac{qn \lambda}{\lambda^2 + B^2} = \frac{q^2 n}{m\nu \left(1 + \frac{m^2\nu^2 B^2}{q^2}\right)}
\]
и также существуют токи в направлении перпендикулярном $\vec{E}$ и $\vec{B}$.
Если $\vec{E} \parallel \vec{B}$:
\[
	\vec{v} = 
	\frac{\lambda\vec{E}}{\lambda^2 + B^2} + \frac{B^2 \vec{E}}{\lambda(\lambda^2 + B^2)} = \frac{\vec{E}}{\lambda}
\]
\[
	\sigma = \frac{qn}{\lambda} = \frac{q^2 n}{m\nu}
\]
Найдём параметр $\lambda$ для меди:
\[
	\lambda = \frac{qn}{\sigma} = qn\rho_e = -\frac{ k e N_A \rho\rho_e}{M} = -\frac{ 1\cdot 1{,}6\cdot 10^{-19} \cdot 6{,}02\cdot 10^{23} \cdot  1{,}75 \cdot 10^{-8} \cdot  8{,}9\cdot 10^3 }{63{,}5\cdot 10^{-3}} \approx - 2{,}4 \cdot 10^2 \text{ Тл}
\]
Лучшим чем медь проводником является серебро. Для него:
\[
	\lambda = -\frac{ 1\cdot 1{,}6\cdot 10^{-19} \cdot 6{,}02\cdot 10^{23} \cdot  1{,}5 \cdot 10^{-8} \cdot  8{,}9\cdot 10^3 }{63{,}5\cdot 10^{-3}} \approx - 2{,}0 \cdot 10^2 \text{ Тл}
\]
Для всех остальных веществ этот параметр существенно больше и как следствие:
\[
	\vec{v} \approx
	\frac{\vec{E}}{\lambda} + \frac{1}{\lambda^3} (\vec{B} (\vec{E} \cdot \vec{B} ) + \lambda \vec{E} \times \vec{B}) \approx \frac{\vec{E}}{\lambda} 
\]
\section{Уравнение Больцмана в $\tau$-приближении}

Уравнение Больцмана (Лиувилля) в $\tau$-приближении:
\[
	\frac{d f}{d t} = \frac{\partial f}{\partial t} + \vec{v} \cdot \frac{\partial f}{\partial \vec{r}} + \vec{F} \cdot \frac{\partial f}{\partial \vec{p}} = \frac{f_0 - f}{\tau}
\]
Вдоль характеристики:
\[
	\begin{aligned}
	& \frac{d \vec{r}}{d t} = \vec{v} \\
	& \frac{d \vec{p}}{d t} = \vec{F} \\
	& \frac{d f}{d t} = \frac{f_0 - f}{\tau}
	\end{aligned}
\]
Пусть 
\[
	f = GV
\]
\[
	\frac{d G}{d t} + \frac{G}{\tau} = 0 \quad \Rightarrow \quad G = A e^{-t/\tau}
\]
\[
	\frac{d V}{d t} = \frac{1}{A \tau} e^{t/\tau} f_0 \quad \Rightarrow \quad V = \frac{1}{A \tau} \int\limits_{a}^t e^{t'/\tau}f_0(t', \vec{r}'(t'), \vec{p}'(t')) dt'  + V_0 
\]
\[
	f(t, \vec{r}, \vec{p})  = GV = \left(\frac{1}{\tau} \int\limits_{a}^t e^{t'/\tau}f_0(t', \vec{r}'(t'), \vec{p}'(t')) dt'  + V_0 A \right) e^{-t/\tau}
\]
Решение должно быть конечно при любых $t$, в том числе и при $t \to -\infty$. Отсюда следует:
\[
	V_0 A = \frac{1}{\tau} \int\limits_{-\infty}^{a} e^{t'/\tau}f_0(t', \vec{r}'(t'), \vec{p}'(t')) dt'
\]
Окончательно:
\[
	f(t, \vec{r}, \vec{p}) = \frac{1}{\tau} \int\limits_{-\infty}^t \exp \left(\frac{t' - t}{\tau}\right) f_0(t', \vec{r}'(t'), \vec{p}'(t')) dt'
\]
где $\vec{p}(t')$, $\vec{r}(t')$ удовлетворяют системе:
\[
	\begin{aligned}
	& \frac{d \vec{r}'}{d t'} = \vec{v} \\
	& \frac{d \vec{p}'}{d t'} = \vec{F}
	\end{aligned} 
\]
с начальными условиями:
\[
	\vec{r}'(t) = \vec{r} \quad \vec{p}'(t) = \vec{p}
\]
Рассмотрим систему с гамильтонианом:
\[
	H = H(\vec{p}, \vec{r}) = E(\vec{p}) + U(\vec{r})
\]
то есть явно не зависящую от времени. Уравнения движения в гамильтоновой форме:
\[
	\begin{aligned}
	& \frac{d \vec{r}}{d t} = \frac{\partial H}{\partial \vec{p}} \\
	& \frac{d \vec{p}}{d t} = - \frac{\partial H}{\partial \vec{r}}
	\end{aligned}
\]
В этом случае энергия сохраняется вдоль траектории-характеристики:
\[
	\frac{d H}{d t} = 
	\frac{\partial H}{\partial t} + \frac{\partial H}{\partial \vec{p}} \cdot \frac{\partial H}{\partial \vec{r}} - \frac{\partial H}{\partial \vec{r}} \cdot \frac{\partial H}{\partial \vec{p}} = 0
\]
Выберем 
\[
	f_0 = \frac{1}{N} \exp \left(- \frac{E}{kT}\right)
\]
И найдём плотность тока:
\[
	\vec{j}(t) = \int \vec{v}(\vec{p}) f(t) d^3r\, d^3p = \int \frac{\vec{v}(\vec{p})}{N \tau} \int\limits_{-\infty}^t \exp \left(\frac{t' - t}{\tau}\right) \exp \left(- \frac{E(\vec{p'}(t'))}{kT}\right) dt'\, d^3r\, d^3p
\]
\section{Электромагнитное поле произвольно движущейся частицы}

\section{Решение неоднородного волнового уравнения (3d)}

\section{Система Молчанова-Селькова и всё что с ней связано}

В организмах как известно протекают различные химические реакции. Представление о мозге как о большой автоколебательной системе привело меня к поиску простейшей модели связанной автоколебательной системы. Случайно мне встретилась простейшая из автоколебательных систем -- система Молчанова-Селькова (речь шла о химических реакциях, но здесь приводятся только уравнения):
\[
	\begin{cases}
	\frac{\partial \xi}{\partial t} = \beta - \xi \eta^2 \\
	\frac{\partial \eta}{\partial t} = \xi \eta^2 - \xi
	\end{cases}
\]
Обычно $\beta$ -- внешний поток субстрата -- считается постоянным и ($\beta > 0$). Пока остановимся на этом случае.

\subsection{Устойчивость по Ляпунову}

\[
	\begin{cases}
	0 = \beta - \xi_0 \eta_0^2 \\
	0 = \xi_0 \eta_0^2 - \xi_0
	\end{cases}
	\quad 
	\Rightarrow
	\quad
	\begin{cases}
	\eta_0 = \pm 1 \\
	\xi_0 = \beta
	\end{cases}
\]
Определим теперь характер точек:
\[
	\xi = \xi_0 + \delta \xi \quad \eta = \eta_0 + \delta \eta
\]
\[
	\begin{cases}
	\frac{\partial \delta\xi}{\partial t} = - 2 \xi_0 \eta_0 \delta \eta - \eta_0^2 \delta \xi = \mp 2 \beta \delta \eta - \delta \xi\\
	\frac{\partial \delta\eta}{\partial t} = 2 \xi_0 \eta_0 \delta \eta + (\eta_0^2 - 1) \delta \xi = \pm 2 \beta \delta \eta
	\end{cases}
\]
\[
	\delta \eta = A_\eta e^{\pm 2 \beta t} \quad \delta \xi = A_\xi e^{-t} - A_\eta e^{\pm 2 \beta t}
\]
Откуда видно, что точка $(1, \beta)$ является седловой, точка $(-1, \beta)$ представляет собой устойчивый узел.
\section{Электродинамические системы}

\subsection{Однородно-заряженные объекты}

\subsubsection{Однородно-заряженная линия}

\[
	2\pi r h E = \frac{\lambda}{\varepsilon_0} h 
	\quad \Rightarrow \quad
	E = \frac{\lambda}{2 \pi \varepsilon_0 r}
\]
С потенциалом в данном случае есть ряд проблем:
\[
	\varphi - \varphi_0 = - \int\limits_{r_0}^{r} \vec{E} \cdot d\vec{r} =
	- \frac{\lambda}{2\pi\varepsilon_0} \ln \frac{r}{r_0}
\]
Чему равно $\varphi_0$ и $r_0$ не имеет значения.

\subsubsection{Однородно-заряженный отрезок}

\[
	\begin{gathered}
	\varphi 
	= 
	\frac{\lambda}{4\pi\varepsilon_0} \int\limits_{-l}^{l} \frac{dx}{\sqrt{(x - r \cos \theta)^2 + r^2 \sin^2 \theta}} 
	=
	\frac{\lambda}{4\pi\varepsilon_0} \mathrm{\,arsh\,} \frac{x - r \cos \theta}{r \sin \theta} \Big|_{-l}^{l} 
	= \\ =
	\frac{\lambda}{4\pi\varepsilon_0} \ln \frac{(l - r \cos \theta) + \sqrt{(l - r \cos \theta)^2 + r^2 \sin^2 \theta}}{- (l + r \cos \theta) + \sqrt{(l + r \cos \theta)^2 + r^2 \sin^2 \theta}}
	= \\ =
	\frac{\lambda}{4\pi\varepsilon_0} \ln \frac{\left[(l - r \cos \theta) + \sqrt{(l - r \cos \theta)^2 + r^2 \sin^2 \theta}\right]\left[(l + r \cos \theta) + \sqrt{(l + r \cos \theta)^2 + r^2 \sin^2 \theta}\right]}{r^2 \sin^2\theta}
	= \\ =
	\frac{\lambda}{4\pi\varepsilon_0} \ln \frac{(l - r \cos \theta + r_+)(l + r \cos \theta + r_-)}{r^2 \sin^2\theta}
	\end{gathered}
\]
При больших $l$ легко видеть, что выражение с точностью до константы переходит к предыдущему случаю.

\[
	\begin{gathered}
	E_r = - \frac{\partial \varphi}{\partial r} = 
	- \frac{\lambda}{4\pi\varepsilon_0} 
	\left[
	\frac{- \cos \theta + \cfrac{r - l \cos \theta}{r_+}}{l - r \cos \theta + r_+} 
	+
	\frac{\cos \theta + \cfrac{r + l \cos \theta}{r_-}}{l + r \cos \theta + r_-} 
	-
	2 \frac{1}{r}
	\right] 
	= \\ =
	- \frac{\lambda}{4\pi\varepsilon_0} 
	\left[
	\frac{r \sin^2 \theta}{l - r \cos \theta + r_+} \frac{1}{r_+} - \frac{\cos \theta}{r_+}
	+
	\frac{r \sin^2 \theta}{l + r \cos \theta + r_-} \frac{1}{r_-} + \frac{\cos \theta}{r_-}
	-
	2 \frac{1}{r}
	\right] 
	= \\ =
	- \frac{\lambda}{4\pi\varepsilon_0} 
	\left[
	\frac{r \sin^2 \theta}{r^2 \sin^2 \theta} \frac{r_+ - l + r \cos \theta}{r_+} - \frac{\cos \theta}{r_+}
	+
	\frac{r \sin^2 \theta}{r^2 \sin^2 \theta} \frac{r_- - l - r \cos \theta}{r_-} + \frac{\cos \theta}{r_-}
	-
	2 \frac{1}{r}
	\right] 
	= \\ =
	- \frac{\lambda}{4\pi\varepsilon_0} 
	\left[
	\frac{1}{r} \frac{- l}{r_+}
	+
	\frac{1}{r} \frac{- l}{r_-}
	\right]
	= 
	\frac{\lambda l}{4\pi\varepsilon_0 r} 
	\left[
	\frac{1}{r_+}
	+
	\frac{1}{r_-}
	\right]
	=
	\frac{q}{8\pi\varepsilon_0 r} 
	\left[
	\frac{1}{r_+}
	+
	\frac{1}{r_-}
	\right]
	\end{gathered}
\]

\[
	\begin{gathered}
	E_\theta = -\frac{1}{r} \frac{\partial \varphi}{\partial \theta} = 
	- \frac{\lambda}{4\pi\varepsilon_0 r} 
	\left[
	\frac{r \sin \theta + \cfrac{r l \sin \theta}{r_+}}{l - r \cos \theta + r_+} 
	+
	\frac{- r \sin \theta + \cfrac{- r l \sin \theta}{r_-}}{l + r \cos \theta + r_-} 
	-
	2 \frac{\cos \theta}{\sin \theta}
	\right] 
	= \\ =
	- \frac{\lambda}{4\pi\varepsilon_0 r} 
	\left[
	\frac{r (r_+ + l) \sin \theta}{r^2 \sin^2 \theta} \frac{r_+ - l + r \cos \theta}{r_+}
	- \frac{r (r_- + l) \sin \theta}{r^2 \sin^2 \theta} \frac{r_- - l - r \cos \theta}{r_-}
	- 2 \frac{\cos \theta}{\sin \theta}
	\right] 
	= \\ =
	- \frac{\lambda}{4\pi\varepsilon_0 r} 
	\left[
	\frac{r_+ + l}{r\sin \theta} \frac{r_+ - l + r \cos \theta}{r_+}
	- \frac{r_- + l}{r\sin \theta} \frac{r_- - l - r \cos \theta}{r_-}
	- 2 \frac{\cos \theta}{\sin \theta}
	\right] 
	= \\ =
	- \frac{\lambda}{4\pi\varepsilon_0 r} 
	\left[
	\frac{r_+^2 - l^2 + l r \cos \theta}{r\sin \theta} \frac{1}{r_+}
	- \frac{r_-^2 - l^2 - l r \cos \theta}{r\sin \theta} \frac{1}{r_-}
	\right] 
	= \\ =
	- \frac{\lambda}{4\pi\varepsilon_0 r} 
	\left[
	\frac{r^2 - l r \cos \theta}{r\sin \theta} \frac{1}{r_+}
	- \frac{r^2 + l r \cos \theta}{r\sin \theta} \frac{1}{r_-}
	\right] 
	= 
	- \frac{\lambda}{4\pi\varepsilon_0 r \sin \theta} 
	\left[
	\frac{r - l \cos \theta}{r_+}
	- \frac{r + l \cos \theta}{r_-}
	\right] 
	= \\ =
	- \frac{\lambda}{4\pi\varepsilon_0 r \sin \theta} 
	\left[
	\sqrt{1 - \frac{r^2}{r_+^2} \sin^2 \theta}
	- \sqrt{1 - \frac{r^2}{r_-^2} \sin^2 \theta}
	\right] 
	= 
	- \frac{\lambda}{4\pi\varepsilon_0 r \sin \theta} 
	\left[
	\frac{\frac{r^2 (r_+^2 - r_-^2) \sin^2 \theta}{r_-^2 r_+^2}}{\sqrt{1 - \frac{r^2}{r_+^2} \sin^2 \theta}
	+ \sqrt{1 - \frac{r^2}{r_-^2} \sin^2 \theta}}
	\right] 
	= \\ =
	\frac{4 \lambda r l \sin \theta \cos \theta}{4\pi\varepsilon_0} 
	\left[
	\frac{1}{r_- r_+ [r (r_- + r_+) + (r_+ - r_-) l \cos \theta]}
	\right] 
	\end{gathered}
\]

\[
	\begin{gathered}
	E^2 = E_\theta^2 + E_r^2 =
	\frac{\lambda^2}{16\pi^2\varepsilon_0^2 r^2} 
	\left\{\left[
	\frac{1}{r_+^2}
	+
	\frac{1}{r_-^2}
	+
	\frac{2}{r_- r_+}
	\right] l^2 +
	\frac{1}{\sin^2 \theta}\left[
	\frac{(r - l \cos \theta)^2}{r_+^2}
	+ \frac{(r + l \cos \theta)^2}{r_-^2}
	- 2 \frac{r^2 - l^2 \cos^2 \theta}{r_+ r_-}
	\right]
	\right\}
	= \\ =
	\frac{\lambda^2}{16\pi^2\varepsilon_0^2 r^2} 
	\left\{\left[
	\frac{1}{r_+^2}
	+
	\frac{1}{r_-^2}
	+
	\frac{2}{r_- r_+}
	\right] l^2 +
	\frac{1}{\sin^2 \theta}\left[
	2 - \frac{r^2 \sin^2 \theta}{r_+^2}
	- \frac{r^2 \sin^2 \theta}{r_-^2}
	- 2 \frac{r^2 - l^2 \cos^2 \theta}{r_+ r_-}
	\right]
	\right\}
	= \\ =
	\frac{\lambda^2}{16\pi^2\varepsilon_0^2 r^2} 
	\left\{
	\left(\frac{1}{r_+} + \frac{1}{r_-} \right)^2 l^2 +
	\frac{1}{\sin^2 \theta}\left[
	2 - r^2 \sin^2 \theta \left(\frac{1}{r_+} + \frac{1}{r_-} \right)^2
	- 2 \frac{r^2 - l^2}{r_+ r_-} \cos^2 \theta
	\right]
	\right\}
	= \\ =
	\frac{\lambda^2}{16\pi^2\varepsilon_0^2 r^2} 
	\left\{
	\left(\frac{1}{r_+} + \frac{1}{r_-} \right)^2 (l^2 - r^2) +
	\frac{1}{\sin^2 \theta}\left[
	2 
	- 2 \frac{r^2 - l^2}{r_+ r_-} (1 - \sin^2 \theta)
	\right]
	\right\}
	= \\ =
	\frac{\lambda^2}{16\pi^2\varepsilon_0^2 r^2} 
	\left\{
	\left(\frac{1}{r_+^2} + \frac{1}{r_-^2} \right) (l^2 - r^2) +
	\frac{1}{\sin^2 \theta}\left[
	2 
	- 2 \frac{r^2 - l^2}{r_+ r_-}
	\right]
	\right\}
	\end{gathered}
\]
Учтём:
\[
	r_+ r_- = \sqrt{(r^2 + l^2) - 4 r^2 l^2 \cos^2 \theta} = \sqrt{(r^2 - l^2) + 4 r^2 l^2 \sin^2 \theta}
\]
В результате:
\[
	1 - \frac{r^2 - l^2}{r_+ r_-} = \frac{4 r^2 l^2 \sin^2 \theta}{r_+ r_- + r^2 - l^2} \frac{1}{r_+ r_-}
\]
\[
	\begin{gathered}
	E^2 = 
	\frac{\lambda^2}{16\pi^2\varepsilon_0^2 r^2} 
	\left\{
	\left(\frac{1}{r_+^2} + \frac{1}{r_-^2} \right) (l^2 - r^2) +
	\frac{4 r^2 l^2}{r_+ r_- + r^2 - l^2} \frac{1}{r_+ r_-}
	\right\}
	\end{gathered}
\]
\subsubsection{Однородно-заряженный луч}

Воспользуемся результатами для отрезка, но несимметричного как выше, а от 0 до $l$:
\[
\begin{gathered}
\varphi 
= 
\frac{\lambda}{4\pi\varepsilon_0} \int\limits_{0}^{l} \frac{dx}{\sqrt{(x - r \cos \theta)^2 + r^2 \sin^2 \theta}} 
=
\frac{\lambda}{4\pi\varepsilon_0} \mathrm{\,arsh\,} \frac{x - r \cos \theta}{r \sin \theta} \Big|_{0}^{l} 
= \\ =
\frac{\lambda}{4\pi\varepsilon_0} \ln \frac{(l - r \cos \theta) + \sqrt{(l - r \cos \theta)^2 + r^2 \sin^2 \theta}}{r(1 - \cos \theta)}
\end{gathered}
\]
Вообще нужно бы устремить $l$ к $\infty$, но тогда и потенциал устремится к бесконечности, поэтому сначала найдём $E_r$, $E_\theta$:
\[
	E_r = - \frac{\partial \varphi}{\partial r} = 
	- \frac{\lambda}{4\pi\varepsilon_0} \left[
		\frac{- \cos \theta + \frac{- l \cos \theta + r}{\sqrt{(l - r \cos \theta)^2 + r^2 \sin^2 \theta}}}{l - r \cos \theta + \sqrt{(l - r \cos \theta)^2 + r^2 \sin^2 \theta}} - \frac{1}{r}
	\right]
\]
\[
	E_\theta = - \frac{1}{r}\frac{\partial \varphi}{\partial \theta} = 
	- \frac{\lambda}{4\pi\varepsilon_0 r} \left[
	\frac{r \sin \theta + \frac{l r \sin \theta}{\sqrt{(l - r \cos \theta)^2 + r^2 \sin^2 \theta}}}{l - r \cos \theta + \sqrt{(l - r \cos \theta)^2 + r^2 \sin^2 \theta}} - \frac{\sin \theta}{1 - \cos \theta}
	\right]
\]
Тогда при $l \to \infty$:
\[
	E_r = \frac{\lambda}{4\pi\varepsilon_0 r}
\]
\[
	E_\theta = \frac{\lambda}{4\pi\varepsilon_0 r} \frac{\sin \theta}{1 - \cos \theta} = \frac{\lambda}{4\pi\varepsilon_0 r} \mathrm{\,ctg\,} \frac{\theta}{2}
\]
Обозначим через $\rho$ расстояние до оси $x$ и найдём выражение для $\vec{E}$ в этих координатах:
\[
	E_\rho = E_r \sin \theta + E_\theta \cos \theta = \frac{\lambda}{4\pi \varepsilon_0 r} \frac{\sin \theta}{1 - \cos \theta} =
	\frac{\lambda}{4\pi \varepsilon_0 \rho} (1 + \cos \theta) = \frac{\lambda}{4\pi \varepsilon_0 \rho} \left(1 + \frac{x}{\sqrt{x^2 + \rho^2}}\right)
\]
\[
	E_x = E_r \cos \theta - E_\theta \sin \theta = - \frac{\lambda}{4\pi \varepsilon_0 r} =
	- \frac{\lambda}{4\pi \varepsilon_0 \sqrt{x^2 + \rho^2}} 
\]

\subsubsection{Однородно-заряженный круговой виток}

В качестве системы координат выберем сферическую систему $(r, \theta, \alpha)$. Найдём потенциал, считая, что радиус окружности $R$:
\[
	\begin{gathered}
	\varphi = \frac{1}{4\pi\varepsilon_0} \int\limits_{0}^{2\pi} \frac{\lambda R d\alpha}{\sqrt{(R \cos \alpha - r \sin \theta)^2 + R^2 \sin^2 \alpha + r^2 \cos^2 \theta}} = 
	\frac{\lambda R}{4\pi\varepsilon_0} \int\limits_{0}^{2\pi} \frac{d\alpha}{\sqrt{R^2 + r^2 - 2 r R \sin \theta \cos \alpha}}
	\end{gathered}
\]

\subsubsection{Однородно-заряженная прямоугольная пластинка}
\subsubsection{Однородно-заряженная круглая пластинка}
\subsubsection{Однородно-заряженная полукруглая пластинка}
\subsubsection{Однородно-заряженная плоскость}
\subsubsection{Однородно-заряженная полуплоскость}
\subsubsection{Однородно-заряженная сфера}
\subsubsection{Однородно-заряженная полусфера}
\subsubsection{Однородно-заряженный шар}
\subsubsection{Однородно-заряженный полушар}

\section{Принцип наименьшего действия для электромагнитных полей преобразованнных по Фурье по времени}

В класической форме действие имеет вид:
\[
	S = \int (\rho \varphi - \vec{j}\cdot\vec{A}) dV dt- \frac{1}{2\mu_0c^2} \int (E^2 - B^2 c^2) dV dt
\]
Кто не верит может проверить, главное не забывать дополнительное условие:
\[
	\vec{E} = - \frac{\partial \vec{A}}{\partial t} - \nabla \varphi
	\qquad
	\vec{B} = \rot \vec{A}
\]
Выполним преобразование Фурье:
\[
\begin{aligned}
	& \rho = \int\limits_{-\infty}^{\infty} \rho_\omega e^{i\omega t} d\omega \\
	& \varphi = \int\limits_{-\infty}^{\infty} \varphi_\omega e^{i\omega t} d\omega \\
	& \vec{E} = \int\limits_{-\infty}^{\infty} \vec{E}_\omega e^{i\omega t} d\omega
\end{aligned}
\qquad
\begin{aligned}
	& \vec{j} = \int\limits_{-\infty}^{\infty} \vec{j}_\omega e^{i\omega t} d\omega \\
	& \vec{A} = \int\limits_{-\infty}^{\infty} \vec{A}_\omega e^{i\omega t} d\omega \\
	& \vec{B} = \int\limits_{-\infty}^{\infty} \vec{B}_\omega e^{i\omega t} d\omega
\end{aligned}
\]
При этом:
\[
	\vec{E}_\omega = - i \omega \vec{A}_\omega - \nabla \varphi_\omega
	\qquad
	\vec{B}_\omega = \rot \vec{A}_\omega
\]
Индекс $\omega$ означает и то, что это преобразованная компонента, и одновременно зависимость от $\omega$. Но все поля действительны. Какой кошмар! Что делать, кто виноват? Действительно кошмар, ведь теперь они подчиняются дополнительным условиям. Чтобы их получить рассмотрим тождество:
\[
	\rho = (1 - \alpha) \int\limits_{-\infty}^{\infty} \rho_\omega e^{i\omega t} d\omega + \alpha \int\limits_{-\infty}^{\infty} \rho^*_\omega e^{- i\omega t} d\omega = 
	\int\limits_{-\infty}^{\infty} [(1 - \alpha) \rho_\omega + \alpha \rho^*_{- \omega}] e^{- i\omega t} d\omega
	= \int\limits_{-\infty}^{\infty} \rho_\omega e^{- i\omega t} d\omega
\]
Данное тождество выполняется при любом параметре $\alpha$, отсюда автоматически следует:
\[
	\rho^*_{- \omega} = \rho_{\omega}
\]
Аналогичные условия справедливы и для всех остальных функций и нужны для правильного варьирования. Учтём теперь, что для произведения функций справедливо:
\[
	\int\limits_{-\infty}^{\infty} f^*(t) g(t) dt = 
	\int\limits_{-\infty}^{\infty} f^*_\omega(\omega)  \int\limits_{-\infty}^{\infty} g(t) e^{- i\omega t} dt d\omega =
	2\pi \int\limits_{-\infty}^{\infty} f^*_\omega(\omega) g_\omega(\omega) d\omega
\]
Получаем действие в виде:
\[
	S = 2\pi \left[ \int (\rho^*_\omega \varphi_\omega - \vec{j}^*_\omega\cdot\vec{A}_\omega) dV d\omega- \frac{1}{2\mu_0c^2} \int (\vec{E}^*_\omega\cdot\vec{E}_\omega - c^2 \vec{B}^*_\omega \cdot \vec{B}_\omega) dV d\omega \right]
\]
Как тяжела была бы жизнь, если бы не было закона сохранения заряда:
\[
	i\omega \rho_\omega + \div \vec{j}_\omega = 0
\]
Подставляем $\rho$ отсюда в действие и учитываем тождество:
\[
	\div u \vec{a} = \vec{a} \cdot \nabla u + u \div \vec{a}
\]
Наслаждаемся результатом:
\[
	S = \int\limits_{-\infty}^{\infty}\oint \frac{1}{i\omega} \varphi_\omega \vec{j}^*_\omega \cdot d\vec{S} d\omega +
	\int \frac{\vec{j}^*_\omega}{i\omega}  \cdot (- \nabla\varphi_\omega - i\omega\vec{A}_\omega) dV d\omega - \frac{1}{2\mu_0c^2} \int \left(\vec{E}^*_\omega\cdot\vec{E}_\omega - \frac{c^2}{\omega^2} \rot\vec{E}^*_\omega \cdot \rot\vec{E}_\omega\right) dV d\omega
\]

\[
\boxed{
	S = \int\limits_{-\infty}^{\infty}\oint \frac{1}{i\omega} \varphi_\omega \vec{j}^*_\omega \cdot d\vec{S} d\omega +
	\int \frac{1}{i\omega}  \vec{j}^*_\omega \cdot \vec{E}_\omega dV d\omega - \frac{1}{2\mu_0c^2} \int \left(\vec{E}^*_\omega\cdot\vec{E}_\omega - \frac{c^2}{\omega^2} \rot\vec{E}^*_\omega \cdot \rot\vec{E}_\omega\right) dV d\omega
}
\]
Покажем как отсюда получить уравнения Максвелла для резонаторов. Варьируем по неизвестному полю $\vec{E}_\omega$. Нас будет интересовать только третий кусок действия:
\[
	\begin{gathered}
	\delta \int \left(\vec{E}^*_\omega\cdot\vec{E}_\omega - \frac{c^2}{\omega^2} \rot\vec{E}^*_\omega \cdot \rot\vec{E}_\omega\right) dV d\omega 
	= \\ =
	\int \left(\delta\vec{E}^*_\omega\cdot\vec{E}_\omega + \vec{E}^*_\omega\cdot\delta\vec{E}_\omega - \frac{c^2}{\omega^2} \rot\delta \vec{E}^*_\omega \cdot \rot\vec{E}_\omega - \frac{c^2}{\omega^2} \rot \vec{E}^*_\omega \cdot \rot\delta\vec{E}_\omega \right) dV d\omega
	= \\ =
	\int \left(\delta\vec{E}^*_{-\omega}\cdot\vec{E}_{-\omega} + \vec{E}^*_\omega\cdot\delta\vec{E}_\omega - \frac{c^2}{\omega^2} \rot\delta \vec{E}^*_{-\omega} \cdot \rot\vec{E}_{-\omega} - \frac{c^2}{\omega^2} \rot \vec{E}^*_\omega \cdot \rot\delta\vec{E}_\omega \right) dV d\omega
	= \\ =
	2 \int \left(\vec{E}^*_\omega\cdot\delta\vec{E}_\omega - \frac{c^2}{\omega^2} \rot \vec{E}^*_\omega \cdot \rot\delta\vec{E}_\omega \right) dV d\omega
	\end{gathered}
\]
Здесь учтены условия, указанные выше. Воспользуемся тождеством:
\[
	\rot \vec{a} \cdot \rot \vec{b} = \div (\vec{b} \times \rot \vec{a}) + \vec{b} \cdot \rot \rot \vec{a}
\]
Получаем уравнения для резонаторов с идеально-проводящими стенками:
\[
	\frac{1}{i\omega} \vec{j}^*_\omega - \frac{1}{\mu_0 c^2} \left(\vec{E}^*_\omega - \frac{c^2}{\omega^2} \rot\rot\vec{E}^*_\omega \right) = 0
\]
\[
	\mu_0 \vec{j}^*_\omega - \frac{i\omega}{c^2} \vec{E}^*_\omega - \rot\vec{B}^*_\omega = 0
\]
\[
	 \rot\vec{B}_\omega = \mu_0 \vec{j}_\omega + \frac{i\omega}{c^2} \vec{E}_\omega
\]
\section{Странные вещи}

\subsection{Тождество для д...а}

\[
	f(x)\delta(x - x_0) + [F(x) - F(x_0)] \delta'(x - x_0) = 0, \qquad F(x) = \int f(x) dx
\]

\textit{Док-во:}

Продифференцируем тождество:
\[
	F(x) \delta(x - x_0) = F(x_0) \delta(x - x_0)
\]
Просто, не правда ли? В чём подвох? При представлении $\delta$-функции в форме гауссовского предела доказательство упирается в весьма необычные трудности. В частности требуется доказать следующее равенство:
\[
	\lim\limits_{a \to 0} \frac{1}{a}\left(1 - 2 \frac{(x - x_0)^2}{a^2}\right) e^{- \cfrac{(x - x_0)^2}{a^2}} = \lim\limits_{a \to 0} \frac{1}{a} e^{- \cfrac{(x - x_0)^2}{a^2}}
\]
И вот вопрос: как это сделать строго? Казалось бы, что сложного, рассмотрим два случая $x = x_0$ и $x \ne x_0$, равенство выполняется? - Да! Но мне это доказательство не даёт покоя. Кажется, что я что-то упускаю. Тем более, что из тождества следует:
\[
	\delta'(x -x_0) = \frac{1}{x_0 - x} \delta(x -x_0)
\]

\subsection{Ещё одно странное тождество}

\[
	(f(x) \eta(x - a))''_{xx} = f''(x) \eta(x - a) + f'(a) \delta(x - a) + f(a) \delta'(x - a)
\]

\textit{Док-во:}

Снова всё тривиально:
\[
	(f(x) \eta(x - a))''_{xx} = (f'(x) \eta(x - a) + f'(x) \delta(x - a))'_x = (f'(x) \eta(x - a) + f'(a) \delta(x - a))'_x =
	f''(x) \eta(x - a) + f'(a) \delta(x - a) + f(a) \delta'(x - a)
\]

\subsection{Функция отличная от нуля в одной единственной точке}

\[
	f(x) = 2A\eta(-|x-a|) = \begin{cases}
	A, & x = a\\
	0, & x\ne a
	\end{cases}
\]
Наибольший интерес представляет её производная:
\[
	f'(x) = -2A\delta(-|x-a|)\sign(x -a) = 0
\]
Не правда ли замечательно! Теперь решение уравнения:
\[
	y' = 0,\qquad y(0) = y_0
\]
будет выглядеть так:
\[
	y = y_0+\sum\limits_{i, a_i \ne 0} 2 C_i \eta(-|x-a_i|)
\]
То есть в классе функций с разрывами решений бесконечно много! 

Как мне подсказал Antony я вообще говоря не прав, так как здесь имеет место неопределённость вида $\infty\times 0$. Представление в форме предела последовательности для $\delta-$функции, приводит тем не менее к результатам изложенным выше, но при одном дополнительном условии, что пределы можно менять местами, что скорее всего также неправильно. Ещё один способ мог бы быть связан с представлением решения в форме предела:
\[
	\lim\limits_{b \to 0} \eta(x - a - b) - \eta(x - a)
\]
Но как следует из определения $\eta-$функции это тождественно 0. И вопрос о правильности полученных результатов остаётся открытым. Вдогонку к данной проблеме можно рассмотреть следующее интересное тождество:
\[
	\sign^2(x) = |\sign(x)|
\]
Продифференцируем его:
\[
	4 \sign(x) \delta(x) = \sign(\sign(x)) 2\delta(x) = 2 \sign(x) \delta(x) 
\]
Если бы можно было сократить всё лишнее было бы:
\[
	4 = 2
\]
Но к счастью сократить ничего нельзя. Однако, отсюда следует:
\[
	 2 \sign(x) \delta(x) = 0
\]

\subsection{Функция для вычисления площади пересечения треугольника с полуплоскостью}

Рассмотрим треугольник, один из катетов, которого лежит на оси абсцисс, другой параллелен оси ординат и прямую, параллельную оси ординат, делящую плоскость на две полуплоскости, как показано на рисунке:
\begin{center}
	\includegraphics{images/png/triangle_func.png}
	%\usetikzlibrary{patterns}
\begin{tikzpicture}

\draw[thick,-stealth]  (0,-1) -- (0,6);
\draw[thick,-stealth] (-2,0) -- (6,0);
\draw[very thick, pattern=vertical lines] (-1,0) -- (1.7, 4) -- (3,0) -- (-1, 0);
\draw[very thick, dashed] (2, -1) -- (2, 5);
\path[pattern=horizontal lines] (2, -1) rectangle (-1.8, 5);

\node[below, fill = white] at (-1,-0.1) {$x_A$};
\node[above, fill = white] at (1.1,4.1) {$B(x_B, y_B)$};
\node[below] at (3,-0.1) {$x_C$};
\node[below right] at (2,-0.1) {$x$};

\end{tikzpicture}
\end{center}
Площадь области с двойной штриховкой:
\[
	S_+(x, x_A, x_B, x_C, y_B) = 
	\begin{cases}
	0, & x < x_A; \\
	\frac{1}{2} y_B \frac{(x - x_A)^2}{x_B - x_A}, & x_A < x < x_B; \\
	\frac{1}{2} y_B \left[(x_B - x_A) + \frac{2 x_C - x - x_B}{x_C - x_B} (x - x_B)\right] , & x_B < x < x_C; \\
	\frac{1}{2} y_B (x_C - x_A), & x_C < x.
	\end{cases}
\]
Другая полуплоскость, даёт область с одинарной штриховкой:
\[
S_-(x, x_A, x_B, x_C, y_B) = 
\begin{cases}
\frac{1}{2} y_B (x_C - x_A), & x < x_A; \\
\frac{1}{2} y_B \left[(x_C - x_B) + \frac{x_B + x - 2 x_A}{x_B - x_A} (x_B - x)\right], & x_A < x < x_B; \\
\frac{1}{2} y_B \frac{(x_C - x)^2}{x_C - x_B} , & x_B < x < x_C; \\
0, & x_C < x.
\end{cases}
\]
Первую из этих функций можно представить в виде:
\[
	\begin{aligned}
	& S_+(x, x_A, x_B, x_C, y_B) =  
	\frac{1}{2} y_B \frac{(x - x_A)^2}{x_B - x_A} \eta(x - x_A) \eta(x_B - x)  + \\ & +
	\frac{1}{2} y_B \left[(x_B - x_A) + \frac{2 x_C - x - x_B}{x_C - x_B} (x - x_B)\right] \eta(x - x_B) \eta(x_C - x) +
	\frac{1}{2} y_B (x_C - x_A) \eta(x - x_C)
	\end{aligned}
\]
Здесь следует отметить, что представление возможно только если $x_A < x_B < x_C$. Именно для этого случая записаны все выражения выше.
\section{Весёлые интегралы}

\begin{enumerate}
	\item 
	\[
		\int \frac{dx}{\sqrt{x^2 + 1}} = \mathrm{arsh\,}x = \ln |x + \sqrt{x^2 + 1}|
	\]
	\item
	\[
		\int \sqrt{x^2 + 1} dx = 
		x \sqrt{x^2 + 1} - \int \frac{x^2}{\sqrt{x^2 + 1}}dx =
		x \sqrt{x^2 + 1} - \int \sqrt{x^2 + 1}dx + \int \frac{dx}{\sqrt{x^2 + 1}}
	\]
	\[
		\Rightarrow \quad
		\int \sqrt{x^2 + 1} dx = \frac{x \sqrt{x^2 + 1} + \ln |x + \sqrt{x^2 + 1}|}{2}
	\]
	\item
	\[
		\int \frac{\sqrt{x}}{\sqrt{x-1}} dx = 
		\int 2 \sqrt{x} d\sqrt{x-1} =
		[\sqrt{x - 1} = p]
		= 2 \int \sqrt{p^2 + 1} dp = 
		\sqrt{x} \sqrt{x-1}+ \ln |\sqrt{x - 1} + \sqrt{x}|
	\]
	\item
	\[
		\int \frac{x\,dx}{\sqrt{px^2 - ax - b}} = 
		\frac{1}{\sqrt{p}} \int \frac{\Big(x - \cfrac{a}{2p} + \cfrac{a}{2p}\Big)\,dx}{\sqrt{\Big(x - \cfrac{a}{2p}\Big)^2 - \Big(\cfrac{a^2}{4p^2} + \cfrac{b}{p}\Big)}}
		= \frac{1}{\sqrt{p}} \sqrt{\Big(x - \cfrac{a}{2p}\Big)^2 - \Big(\cfrac{a^2}{4p^2} + \cfrac{b}{p}\Big)} + 
		\frac{1}{\sqrt{p}} \cfrac{a}{2p} \,\mathrm{arch\,} \frac{x - \cfrac{a}{2p}}{\sqrt{\cfrac{a^2}{4p^2} + \cfrac{b}{p}}}
	\]
\end{enumerate}
\section{Весёлые интегралы в комплексной плоскости}

\begin{center}
\begin{longtable}{|p{0.03\textwidth}|p{0.3\textwidth}|p{0.6\textwidth}|}
	\hline
	\endfirsthead
	\hline
	\endhead
	\endfoot
	
	\hline
	\endlastfoot
	
	1 
	&
	
	$f(z)$ ограничена по модулю:
	\[
	|f(z)| < M
	\]
	
	\centering\includegraphics[width = 0.3 \textwidth]{images/png/countour_int1.png}
	&
	\[
	\begin{gathered}
	\oint\limits_C \frac{e^{-ia z}}{z} f(z) dz = 
	\int\limits_{C_R} + \int\limits_{R}^{r} + \int\limits_{C_r} + \int\limits_{-r}^{-R}
	\end{gathered}
	\]
	\[
	\begin{gathered}
	\left|
	\int\limits_{C_R} \frac{e^{-ia z}}{z} f(z) dz 
	\right| 
	= \left| 
	\int\limits_{\pi}^{2\pi} i e^{-ia R e^{i\varphi}} f(R e^{i\varphi}) d\varphi
	\right|
	\leqslant 
	\int\limits_{\pi}^{2\pi} e^{a R \sin \varphi} M d \varphi = \\ =
	M e^{a R \sin \varphi^*} \pi\Big|_{\varphi^* \in (\pi, 2\pi)} \underset{R \to \infty}{\to} 0
	\end{gathered}
	\]
	\[
	\begin{gathered}
	\lim\limits_{r \to 0} \int\limits_{C_r} \frac{e^{-ia z}}{z} f(z) dz =
	\lim\limits_{r \to 0} \int\limits_{2\pi}^{\pi} i e^{-ia R e^{i\varphi}} f(r e^{i\varphi}) d\varphi =
	- i f(0) \pi
	\end{gathered}
	\]
	\[
	\begin{gathered}
	\fint\limits_{-\infty}^{\infty} \frac{e^{-ia z}}{z} f(z) dz =
	- i f(0) \pi
	\end{gathered}
	\]
	Можно показать, что при изменении знака $a$ изменится контур (будет вверху), направление обхода и как следствие знак результата.
	\[
	\begin{gathered}
	\fint\limits_{-\infty}^{\infty} \frac{e^{ia z}}{z} f(z) dz =
	i f(0) \pi \sign a
	\end{gathered}
	\]
	Отсюда можно получить более общее выражение:
	\[
	\begin{gathered}
	\fint\limits_{-\infty}^{\infty} \frac{e^{ia z}}{z - z_0} f(z) dz =
	i e^{iaz_0} f(z_0) \pi \sign a
	\end{gathered}
	\]
	\\ 
\end{longtable}
\end{center}
\section{Весёлые диффуры}

\begin{enumerate}
	\item
		\[
			\begin{aligned}
				& g'' = \frac{a}{2g^2} + \frac{b}{g^3} \\
				& \Rightarrow g'^2 = g'^2_0 + \frac{a}{g_0} + \frac{b}{g^2_0} - \frac{a}{g} - \frac{b}{g^2} =
				p - \frac{a}{g} - \frac{b}{g^2} \\
				& \int\limits_{g_0}^{g} \frac{g\,dg}{\sqrt{pg^2 - ag - b}} = t - t_0
			\end{aligned}
		\]
\end{enumerate}

\end{document}